%===================================================================================
\chapter{Description of the NVECTOR module}\label{s:nvector}
%===================================================================================
\index{NVECTOR@\texttt{NVECTOR} module}
% This is a shared SUNDIALS TEX file with description of
% the generic nvector abstraction
%
The {\sundials} solvers are written in a data-independent manner.
They all operate on generic vectors (of type \Id{N\_Vector}) through a set of
operations defined by the particular {\nvector} implementation.
Users can provide their own specific implementation of the {\nvector}
module, or use one of the implementations provided with {\sundials}.
The generic {\nvector} is described below and the implementations
provided with {\sundials} are described in the following sections.

% ====================================================================
\section{The NVECTOR API}
\label{s:nvector_api}
% ====================================================================

The generic {\nvector} API can be broken down into groups of functions:
the core vector operations, the fused vector operations, the vector array
operations, the local reduction operations, the exchange operations, and
finally some utility functions. All but the last group are defined by a
particular {\nvector} implementation. The utility functions are defined
by the generic {\nvector} itself.

% ====================================================================
\subsection{NVECTOR core functions}\label{ss:nvecops}

\ucfunctionf{N\_VGetVectorID}
{
  id = N\_VGetVectorID(w);
}
{
  Returns the vector type identifier for the vector \id{w}. It is used to determine the
  vector implementation type (e.g.~serial, parallel,\ldots) from the abstract
  \id{N\_Vector} interface.
}
{
  \begin{args}[w]
  \item[w] (\id{N\_Vector}) a {\nvector} object
  \end{args}
}
{
  This function returns an \id{N\_Vector\_ID}. Possible values are given in Table
  \ref{t:vectorIDs}.
}
{}

\ucfunctionf{N\_VClone}
{
  v = N\_VClone(w);
}
{
  Creates a new \id{N\_Vector} of the same type as an existing vector \id{w} and sets the
  {\em ops} field. It does not copy the vector, but rather allocates storage for the new vector.
}
{
  \begin{args}[w]
  \item[w] (\id{N\_Vector}) a {\nvector} object
  \end{args}
}
{
  This function returns an \id{N\_Vector} object. If an error occurs, then this
  routine will return \id{NULL}.
}
{}

\ucfunctionf{N\_VCloneEmpty}
{
  v = N\_VCloneEmpty(w);
}
{
  Creates a new \id{N\_Vector} of the same type as an existing vector \id{w} and sets the
  {\em ops} field. It does not allocate storage for data.
}
{
  \begin{args}[w]
  \item[w] (\id{N\_Vector}) a {\nvector} object
  \end{args}
}
{
  This function returns an \id{N\_Vector} object. If an error occurs, then this
  routine will return \id{NULL}.
}
{}

\ucfunctionf{N\_VDestroy}
{
  N\_VDestroy(v);
}
{
  Destroys the \id{N\_Vector} \id{v} and frees memory allocated for its
  internal data.
}
{
  \begin{args}[v]
  \item[v] (\id{N\_Vector}) a {\nvector} object to destroy
  \end{args}
}
{}
{}

\ucfunctionfl{N\_VSpace}
{
  N\_VSpace(v, \&lrw, \&liw);
}
{
  Returns storage requirements for one \id{N\_Vector}.
  \id{lrw} contains the number of realtype words and \id{liw}
  contains the number of integer words, This function is advisory
  only, for use in determining a user's total space requirements;
  it could be a dummy function in a user-supplied
  {\nvector} module if that information is not of interest.
}
{
  \begin{args}[v]
  \item[v] (\id{N\_Vector}) a {\nvector} object
  \item[lrw] (\id{sunindextype*}) out parameter containing the number of realtype words
  \item[liw] (\id{sunindextype*}) out parameter containing the number of integer words
  \end{args}
}
{}
{}
{
  integer(c\_long) :: lrw(1), liw(1)\\
  call FN\_VSpace\_Serial(v, lrw, liw)
}

\ucfunctionf{N\_VGetArrayPointer}
{
  vdata = N\_VGetArrayPointer(v);
}
{
  Returns a pointer to a \id{realtype} array from the \id{N\_Vector} \id{v}.
  Note that this assumes that the internal data in \id{N\_Vector} is
  a contiguous array of \id{realtype} and is accessible from the CPU.

  This routine is only used in the
  solver-specific interfaces to the dense and banded (serial) linear
  solvers, the sparse linear solvers (serial and threaded), and in the
  interfaces to the banded (serial) and band-block-diagonal (parallel)
  preconditioner modules provided with {\sundials}.
}
{
  \begin{args}[v]
  \item[v] (\id{N\_Vector}) a {\nvector} object
  \end{args}
}
{
  \id{realtype*}
}
{}

\ucfunctionf{N\_VGetDeviceArrayPointer}
{
  vdata = N\_VGetDeviceArrayPointer(v);
}
{
  Returns a device pointer to a \id{realtype} array from the \id{N\_Vector}
  \id{v}. Note that this assumes that the internal data in \id{N\_Vector} is a
  contiguous array of \id{realtype} and is accessible from the device (e.g.,
  GPU).

  This operation is \textit{optional} except when using the GPU-enabled direct
  linear solvers.
}
{
  \begin{args}[v]
  \item[v] (\id{N\_Vector}) a {\nvector} object
  \end{args}
}
{
  \id{realtype*}
}
{
  Currently, only the GPU-enabled {\sundials} vectors provide this operation.
  All other SUNDIALS vectors will return \id{NULL}.
}

\ucfunctionf{N\_VSetArrayPointer}
{
  N\_VSetArrayPointer(vdata, v);
}
{
  Overwrites the pointer to the data in an \id{N\_Vector} with a given \id{realtype*}.
  Note that this assumes that the internal data in \id{N\_Vector} is a contiguous
  array of \id{realtype}. This routine is only used in the interfaces to the dense
  (serial) linear solver, hence need not exist in a user-supplied {\nvector} module
  for a parallel environment.
}
{
  \begin{args}[v]
  \item[v] (\id{N\_Vector}) a {\nvector} object
  \end{args}
}
{}
{}

\ucfunctionf{N\_VGetCommunicator}
{
  N\_VGetCommunicator(v);
}
{

  Returns a pointer to the \id{MPI\_Comm} object associated with the
  vector (if applicable). For MPI-unaware vector implementations, this
  should return \id{NULL}.
}
{
  \begin{args}[v]
  \item[v] (\id{N\_Vector}) a {\nvector} object
  \end{args}
}
{
  A \id{void *} pointer to the \id{MPI\_Comm} object if the vector is MPI-aware,
  otherwise \id{NULL}.
}
{}

\ucfunctionf{N\_VGetLength}
{
  N\_VGetLength(v);
}
{
  Returns the global length (number of `active' entries) in the
  {\nvector} \id{v}.  This value should be cumulative across all
  processes if the vector is used in a parallel environment.  If \id{v}
  contains additional storage, e.g., for parallel communication, those
  entries should not be included.
}
{
  \begin{args}[v]
  \item[v] (\id{N\_Vector}) a {\nvector} object
  \end{args}
}
{
  \id{sunindextype}
}
{}

\ucfunctionf{N\_VLinearSum}
{
  N\_VLinearSum(a, x, b, y, z);
}
{
  Performs the operation $z = a x + b y$, where $a$ and $b$ are \id{realtype}
  scalars and $x$ and $y$ are of type \id{N\_Vector}:
  $z_i = a x_i + b y_i, \: i=0,\ldots,n-1$.
}
{
  \begin{args}[a]
  \item[a] (\id{realtype}) constant that scales \id{x}
  \item[x] (\id{N\_Vector}) a {\nvector} object
  \item[b] (\id{realtype}) constant that scales \id{y}
  \item[y] (\id{N\_Vector}) a {\nvector} object
  \item[z] (\id{N\_Vector}) a {\nvector} object containing the result
  \end{args}
}
{
  The output vector \id{z} can be the same as either of the input vectors (\id{x} or \id{y}).
}
{}

\ucfunctionf{N\_VConst}
{
  N\_VConst(c, z);
}
{
  Sets all components of the \id{N\_Vector} \id{z} to \id{realtype} \id{c}:
  $z_i = c,\: i=0,\ldots,n-1$.
}
{
  \begin{args}[c]
  \item[c] (\id{realtype}) constant to set all components of \id{z} to
  \item[z] (\id{N\_Vector}) a {\nvector} object containing the result
  \end{args}
}
{}
{}

\ucfunctionf{N\_VProd}
{
  N\_VProd(x, y, z);
}
{
  Sets the \id{N\_Vector} \id{z} to be the component-wise product of the
  \id{N\_Vector} inputs \id{x} and \id{y}: $z_i = x_i y_i,\: i=0,\ldots,n-1$.
}
{
  \begin{args}[x]
  \item[x] (\id{N\_Vector}) a {\nvector} object
  \item[y] (\id{N\_Vector}) a {\nvector} object
  \item[z] (\id{N\_Vector}) a {\nvector} object containing the result
  \end{args}
}
{}
{}

\ucfunctionf{N\_VDiv}
{
  N\_VDiv(x, y, z);
}
{
  Sets the \id{N\_Vector} \id{z} to be the component-wise ratio of the
  \id{N\_Vector} inputs \id{x} and \id{y}:
  $z_i = x_i / y_i,\: i=0,\ldots,n-1$. The $y_i$ may not be tested
  for $0$ values. It should only be called with a \id{y} that is
  guaranteed to have all nonzero components.
}
{
  \begin{args}[x]
  \item[x] (\id{N\_Vector}) a {\nvector} object
  \item[y] (\id{N\_Vector}) a {\nvector} object
  \item[z] (\id{N\_Vector}) a {\nvector} object containing the result
  \end{args}
}
{}
{}

\ucfunctionf{N\_VScale}
{
  N\_VScale(c, x, z);
}
{
  Scales the \id{N\_Vector} \id{x} by the \id{realtype} scalar \id{c}
  and returns the result in \id{z}: $z_i = c x_i , \: i=0,\ldots,n-1$.
}
{
  \begin{args}[c]
  \item[c] (\id{realtype}) constant that scales the vector \id{x}
  \item[x] (\id{N\_Vector}) a {\nvector} object
  \item[z] (\id{N\_Vector}) a {\nvector} object containing the result
  \end{args}
}
{}
{}

\ucfunctionf{N\_VAbs}
{
  N\_VAbs(x, z);
}
{
  Sets the components of the \id{N\_Vector} \id{z} to be the absolute
  values of the components of the \id{N\_Vector} \id{x}:
  $z_i = | x_i | , \: i=0,\ldots,n-1$.
}
{
  \begin{args}[x]
  \item[x] (\id{N\_Vector}) a {\nvector} object
  \item[z] (\id{N\_Vector}) a {\nvector} object containing the result
  \end{args}
}
{}
{}

\ucfunctionf{N\_VInv}
{
  N\_VInv(x, z);
}
{

  Sets the components of the \id{N\_Vector} \id{z} to be the inverses
  of the components of the \id{N\_Vector} \id{x}:
  $z_i = 1.0 /  x_i  , \: i=0,\ldots,n-1$. This routine
  may not check for division by $0$. It should be called only with an
  \id{x} which is guaranteed to have all nonzero components.
}
{
  \begin{args}[x]
  \item[x] (\id{N\_Vector}) a {\nvector} object to
  \item[z] (\id{N\_Vector}) a {\nvector} object containing the result
  \end{args}
}
{}
{}

\ucfunctionf{N\_VAddConst}
{
  N\_VAddConst(x, b, z);
}
{
  Adds the \id{realtype} scalar \id{b} to all components of \id{x}
  and returns the result in the \id{N\_Vector} \id{z}:
  $z_i = x_i + b , \: i=0,\ldots,n-1$.
}
{
  \begin{args}[x]
  \item[x] (\id{N\_Vector}) a {\nvector} object
  \item[b] (\id{realtype}) constant added to all components of \id{x}
  \item[z] (\id{N\_Vector}) a {\nvector} object containing the result
  \end{args}
}
{}
{}

\ucfunctionf{N\_VDotProd}
{
  d = N\_VDotProd(x, y);
}
{
  Returns the value of the ordinary dot product of \id{x} and \id{y}:
  $d=\sum_{i=0}^{n-1} x_i y_i$.
}
{
  \begin{args}[x]
  \item[x] (\id{N\_Vector}) a {\nvector} object
    with \id{y}
  \item[y] (\id{N\_Vector}) a {\nvector} object
    with \id{x}
  \end{args}
}
{
  \id{realtype}
}
{}

\ucfunctionf{N\_VMaxNorm}
{
  m = N\_VMaxNorm(x);
}
{
  Returns the maximum norm of the \id{N\_Vector} \id{x}:
  $m = \max_{i} | x_i |$.
}
{
  \begin{args}[x]
  \item[x] (\id{N\_Vector}) a {\nvector} object
  \end{args}
}
{
  \id{realtype}
}
{}

\ucfunctionf{N\_VWrmsNorm}
{
  m = N\_VWrmsNorm(x, w)
}
{
  Returns the weighted root-mean-square norm of the \id{N\_Vector} \id{x} with
  \id{realtype} weight vector \id{w}:
  $m = \sqrt{\left( \sum_{i=0}^{n-1} (x_i w_i)^2 \right) / n}$.
}
{
  \begin{args}[x]
  \item[x] (\id{N\_Vector}) a {\nvector} object
  \item[w] (\id{N\_Vector}) a {\nvector} object containing weights
  \end{args}
}
{
  \id{realtype}
}
{}

\ucfunctionf{N\_VWrmsNormMask}
{
  m = N\_VWrmsNormMask(x, w, id);
}
{
  Returns the weighted root mean square norm of the \id{N\_Vector} \id{x} with
  \id{realtype} weight vector \id{w} built using only
  the elements of \id{x} corresponding to
  positive elements of the \id{N\_Vector} \id{id}:
  $m = \sqrt{\left( \sum_{i=0}^{n-1} (x_i w_i H(id_i))^2 \right) / n}$,
  where
  $
  H(\alpha) =
  \begin{cases}
  1 & \alpha > 0 \\
  0 & \alpha \leq 0
  \end{cases}
  $
}
{
  \begin{args}[x]
  \item[x] (\id{N\_Vector}) a {\nvector} object
  \item[w] (\id{N\_Vector}) a {\nvector} object containing weights
  \item[id] (\id{N\_Vector}) mask vector
  \end{args}
}
{
  \id{realtype}
}
{}

\ucfunctionf{N\_VMin}
{
  m = N\_VMin(x);
}
{
  Returns the smallest element of the \id{N\_Vector} \id{x}:
  $m = \min_i x_i $.
}
{
  \begin{args}[x]
  \item[x] (\id{N\_Vector}) a {\nvector} object
  \end{args}
}
{
  \id{realtype}
}
{}

\ucfunctionf{N\_VWL2Norm}
{
  m = N\_VWL2Norm(x, w);
}
{
  Returns the weighted Euclidean $\ell_2$ norm of the \id{N\_Vector} \id{x}
  with \id{realtype} weight vector \id{w}:
  $m = \sqrt{\sum_{i=0}^{n-1} (x_i w_i)^2}$.
}
{
  \begin{args}[x]
  \item[x] (\id{N\_Vector}) a {\nvector} object
  \item[w] (\id{N\_Vector}) a {\nvector} object containing weights
  \end{args}
}
{
  \id{realtype}
}
{}

\ucfunctionf{N\_VL1Norm}
{
  m = N\_VL1Norm(x);
}
{
  Returns the $\ell_1$ norm of the \id{N\_Vector} \id{x}:
  $m = \sum_{i=0}^{n-1} | x_i |$.
}
{
  \begin{args}[x]
  \item[x] (\id{N\_Vector}) a {\nvector} object to obtain the norm of
  \end{args}
}
{
  \id{realtype}
}
{}

\ucfunctionf{N\_VCompare}
{
  N\_VCompare(c, x, z);
}
{
  Compares the components of the \id{N\_Vector} \id{x} to the \id{realtype}
  scalar \id{c} and returns an \id{N\_Vector} \id{z} such that:
  $z_i = 1.0$ if $| x_i | \ge c$ and $z_i = 0.0$ otherwise.
}
{
  \begin{args}[c]
  \item[c] (\id{realtype}) constant that each component of \id{x} is compared to
  \item[x] (\id{N\_Vector}) a {\nvector} object
  \item[z] (\id{N\_Vector}) a {\nvector} object containing the result
  \end{args}
}
{}
{}

\ucfunctionf{N\_VInvTest}
{
  t = N\_VInvTest(x, z);
}
{
  Sets the components of the \id{N\_Vector} \id{z} to be the inverses
  of the components of the \id{N\_Vector} \id{x}, with prior testing
  for zero values: $z_i = 1.0 /  x_i  , \: i=0,\ldots,n-1$.
}
{
  \begin{args}[x]
  \item[x] (\id{N\_Vector}) a {\nvector} object
  \item[z] (\id{N\_Vector}) an output {\nvector} object
  \end{args}
}
{
  Returns a \id{booleantype} with value \id{SUNTRUE} if all components
  of \id{x} are nonzero (successful inversion) and returns
  \id{SUNFALSE} otherwise.
}
{}

\ucfunctionf{N\_VConstrMask}
{
  t = N\_VConstrMask(c, x, m);
}
{
  Performs the following constraint tests:
  $x_i > 0$ if $c_i=2$,
  $x_i \ge 0$ if $c_i=1$,
  $x_i \le 0$ if $c_i=-1$,
  $x_i < 0$ if $c_i=-2$.
  There is no constraint on $x_i$ if $c_i=0$. This routine returns a boolean
  assigned to \id{SUNFALSE} if any element failed the constraint test and
  assigned to \id{SUNTRUE} if all passed.  It also sets a mask vector \id{m},
  with elements equal to $1.0$ where the constraint test failed, and $0.0$
  where the test passed. This routine is used only for constraint checking.
}
{
  \begin{args}[c]
  \item[c] (\id{realtype}) scalar constraint value
  \item[x] (\id{N\_Vector}) a {\nvector} object
  \item[m] (\id{N\_Vector}) output mask vector
  \end{args}
}
{
  Returns a \id{booleantype} with value \id{SUNFALSE} if any element failed the
  constraint test, and \id{SUNTRUE} if all passed.
}
{}

\ucfunctionf{N\_VMinQuotient}
{
  minq = N\_VMinQuotient(num, denom);
}
{
  This routine returns the minimum of the quotients obtained
  by term-wise dividing \id{num}$_i$ by \id{denom}$_i$.
  A zero element in \id{denom} will be skipped.
  If no such quotients are found, then the large value
  \Id{BIG\_REAL} (defined in the header file \id{sundials\_types.h})
  is returned.
}
{
  \begin{args}[x]
  \item[num] (\id{N\_Vector}) a {\nvector} object used as the numerator
  \item[denom] (\id{N\_Vector}) a {\nvector} object used as the denominator
  \end{args}
}
{
  \id{realtype}
}
{}


% ====================================================================
\subsection{NVECTOR fused functions}\label{ss:nvecfusedops}

Fused and vector array operations are intended to increase data reuse, reduce
parallel communication on distributed memory systems, and lower the number of
kernel launches on systems with accelerators. If a particular {\nvector}
implementation defines a fused or vector array operation as \id{NULL}, the
generic {\nvector} module will automatically call standard vector operations as
necessary to complete the desired operation.  In all
{\sundials}-provided {\nvector} implementations, all fused and vector
array operations are disabled by default.  However, these
implementations provide additional user-callable functions to enable/disable
any or all of the fused and vector array operations. See the following sections
for the implementation specific functions to enable/disable operations.


\ucfunctionfl{N\_VLinearCombination}
{
  ier = N\_VLinearCombination(nv, c, X, z);
}
{
  This routine computes the linear combination of $n_v$ vectors with $n$
  elements:
  \begin{equation*}
    z_i = \sum_{j=0}^{n_v-1} c_j x_{j,i}, \quad i=0,\ldots,n-1,
  \end{equation*}
  where $c$ is an array of $n_v$ scalars, $X$ is an array of $n_v$ vectors,
  and $z$ is the output vector.
}
{
  \begin{args}[nv]
  \item[nv] (\id{int}) the number of vectors in the linear combination
  \item[c] (\id{realtype*}) an array of $n_v$ scalars used to scale
    the corresponding vector in \id{X}
  \item[X] (\id{N\_Vector*}) an array of $n_v$ {\nvector} objects
    to be scaled and combined
  \item[z] (\id{N\_Vector}) a {\nvector} object containing the result
  \end{args}
}
{
  Returns an \id{int} with value \id{0} for success and a non-zero value otherwise.
}
{
  If the output vector $z$ is one of the vectors in $X$, then it \textit{must} be
  the first vector in the vector array.
}
{
  real(c\_double) :: c(nv)\\
  type(c\_ptr), target :: X(nv)\\
  type(N\_Vector), pointer :: z\\
  ierr = FN\_VLinearCombination(nv, c, X, z)
}

\ucfunctionfl{N\_VScaleAddMulti}
{
  ier = N\_VScaleAddMulti(nv, c, x, Y, Z);
}
{
  This routine scales and adds one vector to $n_v$ vectors with $n$ elements:
  \begin{equation*}
    z_{j,i} = c_j x_i + y_{j,i}, \quad j=0,\ldots,n_v-1 \quad i=0,\ldots,n-1,
  \end{equation*}
  where $c$ is an array of $n_v$ scalars, $x$ is the vector to be scaled and
  added to each vector in the vector array of $n_v$ vectors $Y$, and $Z$ is a
  vector array of $n_v$ output vectors.
}
{
  \begin{args}[nv]
  \item[nv] (\id{int}) the number of scalars and vectors in \id{c}, \id{Y}, and \id{Z}
  \item[c] (\id{realtype*}) an array of $n_v$ scalars
  \item[x] (\id{N\_Vector}) a {\nvector} object to be scaled and added to each
    vector in \id{Y}
  \item[Y] (\id{N\_Vector*}) an array of $n_v$ {\nvector} objects where each vector
    $j$ will have the vector \id{x} scaled by \id{c\_j} added to it
  \item[Z] (\id{N\_Vector}) an output array of $n_v$ {\nvector} objects
  \end{args}
}
{
  Returns an \id{int} with value \id{0} for success and a non-zero value otherwise.
}
{}
{
  real(c\_double) :: c(nv)\\
  type(c\_ptr), target :: Y(nv), Z(nv)\\
  type(N\_Vector), pointer :: x\\
  ierr = FN\_VScaleAddMulti(nv, c, x, Y, Z)
}

\ucfunctionfl{N\_VDotProdMulti}
{
  ier = N\_VDotProdMulti(nv, x, Y, d);
}
{
  This routine computes the dot product of a vector with $n_v$ other vectors:
  \begin{equation*}
    d_j = \sum_{i=0}^{n-1} x_i y_{j,i}, \quad j=0,\ldots,n_v-1,
  \end{equation*}
  where $d$ is an array of $n_v$ scalars containing the dot products of the
  vector $x$ with each of the $n_v$ vectors in the vector array $Y$.
}
{
  \begin{args}[nv]
  \item[nv] (\id{int}) the number of vectors in \id{Y}
  \item[x] (\id{N\_Vector}) a {\nvector} object to be used in a dot product
    with each of the vectors in \id{Y}
  \item[Y] (\id{N\_Vector*}) an array of $n_v$ {\nvector} objects to use
    in a dot product with \id{x}
  \item[d] (\id{realtype*}) an output array of $n_v$ dot products
  \end{args}
}
{
  Returns an \id{int} with value \id{0} for success and a non-zero value otherwise.
}
{}
{
  real(c\_double) :: d(nv)\\
  type(c\_ptr), target :: Y(nv)\\
  type(N\_Vector), pointer :: x\\
  ierr = FN\_VDotProdMulti(nv, x, Y, d)
}


% ====================================================================
\subsection{NVECTOR vector array functions}\label{ss:nvecarrayops}


\ucfunctionf{N\_VLinearSumVectorArray}
{
  ier = N\_VLinearSumVectorArray(nv, a, X, b, Y, Z);
}
{
  This routine computes the linear sum of two vector arrays containing $n_v$
  vectors of $n$ elements:
  \begin{equation*}
    z_{j,i} = a x_{j,i} + b y_{j,i}, \quad i=0,\ldots,n-1 \quad j=0,\ldots,n_v-1,
  \end{equation*}
  where $a$ and $b$ are scalars and $X$, $Y$, and $Z$ are arrays of $n_v$ vectors.
}
{
  \begin{args}[nv]
  \item[nv] (\id{int}) the number of vectors in the vector arrays
  \item[a] (\id{realtype}) constant to scale each vector in \id{X} by
  \item[X] (\id{N\_Vector*}) an array of $n_v$ {\nvector} objects
  \item[Y] (\id{N\_Vector*}) an array of $n_v$ {\nvector} objects
  \item[Z] (\id{N\_Vector*}) an output array of $n_v$ {\nvector} objects
  \end{args}
}
{
  Returns an \id{int} with value \id{0} for success and a non-zero value otherwise.
}
{}

\ucfunctionf{N\_VScaleVectorArray}
{
  ier = N\_VScaleVectorArray(nv, c, X, Z);
}
{
  This routine scales each vector of $n$ elements in a vector array of $n_v$
  vectors by a potentially different constant:
  \begin{equation*}
    z_{j,i} = c_j x_{j,i}, \quad i=0,\ldots,n-1 \quad j=0,\ldots,n_v-1,
  \end{equation*}
  where $c$ is an array of $n_v$ scalars and $X$ and $Z$ are arrays of $n_v$
  vectors.
}
{
  \begin{args}[nv]
  \item[nv] (\id{int}) the number of vectors in the vector arrays
  \item[c] (\id{realtype}) constant to scale each vector in \id{X} by
  \item[X] (\id{N\_Vector*}) an array of $n_v$ {\nvector} objects
  \item[Z] (\id{N\_Vector*}) an output array of $n_v$ {\nvector} objects
  \end{args}
}
{
  Returns an \id{int} with value \id{0} for success and a non-zero value otherwise.
}
{}

\ucfunctionf{N\_VConstVectorArray}
{
  ier = N\_VConstVectorArray(nv, c, X);
}
{
  This routine sets each element in a vector of $n$ elements in a vector array of
  $n_v$ vectors to the same value:
  \begin{equation*}
    z_{j,i} = c, \quad i=0,\ldots,n-1 \quad j=0,\ldots,n_v-1,
  \end{equation*}
  where $c$ is a scalar and $X$ is an array of $n_v$ vectors.
}
{
  \begin{args}[nv]
  \item[nv] (\id{int}) the number of vectors in \id{X}
  \item[c] (\id{realtype}) constant to set every element in every
    vector of \id{X} to
  \item[X] (\id{N\_Vector*}) an array of $n_v$ {\nvector} objects
  \end{args}
}
{
  Returns an \id{int} with value \id{0} for success and a non-zero value otherwise.
}
{}

\ucfunctionf{N\_VWrmsNormVectorArray}
{
  ier = N\_VWrmsNormVectorArray(nv, X, W, m);
}
{
  This routine computes the weighted root mean square norm of $n_v$ vectors with
  $n$ elements:
  \begin{equation*}
    m_j = \left( \frac1n \sum_{i=0}^{n-1} \left(x_{j,i} w_{j,i}\right)^2\right)^{1/2}, \quad j=0,\ldots,n_v-1,
  \end{equation*}
  where $m$ contains the $n_v$ norms of the vectors in the vector array $X$ with
  corresponding weight vectors $W$.
}
{
  \begin{args}[nv]
  \item[nv] (\id{int}) the number of vectors in the vector arrays
  \item[X] (\id{N\_Vector*}) an array of $n_v$ {\nvector} objects
  \item[W] (\id{N\_Vector*}) an array of $n_v$ {\nvector} objects
  \item[m] (\id{realtype*}) an output array of $n_v$ norms
  \end{args}
}
{
  Returns an \id{int} with value \id{0} for success and a non-zero value otherwise.
}
{}

\ucfunctionf{N\_VWrmsNormMaskVectorArray}
{
  ier = N\_VWrmsNormMaskVectorArray(nv, X, W, id, m);
}
{
  This routine computes the masked weighted root mean square norm of $n_v$
  vectors with $n$ elements:
  \begin{equation*}
    m_j = \left( \frac1n \sum_{i=0}^{n-1} \left(x_{j,i} w_{j,i}
    H(id_i)\right)^2 \right)^{1/2}, \quad j=0,\ldots,n_v-1,
  \end{equation*}
  $H(id_i)=1$ for $id_i > 0$ and is zero otherwise, $m$ contains the $n_v$
  norms of the vectors in the vector array $X$ with corresponding weight
  vectors $W$ and mask vector $id$.
}
{
  \begin{args}[nv]
  \item[nv] (\id{int}) the number of vectors in the vector arrays
  \item[X] (\id{N\_Vector*}) an array of $n_v$ {\nvector} objects
  \item[W] (\id{N\_Vector*}) an array of $n_v$ {\nvector} objects
  \item[id] (\id{N\_Vector}) the mask vector
  \item[m] (\id{realtype*}) an output array of $n_v$ norms
  \end{args}
}
{
  Returns an \id{int} with value \id{0} for success and a non-zero value otherwise.
}
{}

\ucfunction{N\_VScaleAddMultiVectorArray}
{
  ier = N\_VScaleAddMultiVectorArray(nv, ns, c, X, YY, ZZ);
}
{
  This routine scales and adds a vector in a vector array of $n_v$ vectors to
  the corresponding vector in $n_s$ vector arrays:
  \begin{equation*}
    z_{k,j,i} = c_k x_{j,i} + y_{k,j,i}, \quad i=0,\ldots,n-1 \quad j=0,\ldots,nv-1, \quad k=0,\ldots,ns-1
  \end{equation*}
  where $c$ is an array of $n_s$ scalars, $X$ is a vector array of $n_v$ vectors
  to be scaled and added to the corresponding vector in each of the $n_s$ vector
  arrays in the array of vector arrays $YY$ and stored in the output array of vector
  arrays $ZZ$.
}
{
  \begin{args}[nv]
  \item[nv] (\id{int}) the number of vectors in the vector arrays
  \item[ns] (\id{int}) the number of scalars in \id{c} and vector arrays
    in \id{YY} and \id{ZZ}
  \item[c] (\id{realtype*}) an array of $n_s$ scalars
  \item[X] (\id{N\_Vector*}) an array of $n_v$ {\nvector} objects
  \item[YY] (\id{N\_Vector**}) an array of $n_s$ {\nvector} arrays
  \item[ZZ] (\id{N\_Vector**}) an output array of $n_s$ {\nvector}
    arrays
  \end{args}
}
{
  Returns an \id{int} with value \id{0} for success and a non-zero value otherwise.
}
{}

\ucfunction{N\_VLinearCombinationVectorArray}
{
  ier = N\_VLinearCombinationVectorArray(nv, ns, c, XX, Z);
}
{
  This routine computes the linear combination of $n_s$ vector arrays containing
  $n_v$ vectors with $n$ elements:
  \begin{equation*}
  z_{j,i} = \sum_{k=0}^{n_s-1} c_k x_{k,j,i}, \quad i=0,\ldots,n-1 \quad j=0,\ldots,n_v-1,
  \end{equation*}
  where $c$ is an array of $n_s$ scalars (type \id{realtype*}), $XX$
  (type \id{N\_Vector**}) is an array of $n_s$ vector arrays each containing $n_v$
  vectors to be summed into the output vector array of $n_v$ vectors $Z$ (type
  \id{N\_Vector*}). If the output vector array $Z$ is one of the vector arrays in
  $XX$, then it \textit{must} be the first vector array in $XX$.
}
{
 \begin{args}[nv]
  \item[nv] (\id{int}) the number of vectors in the vector arrays
  \item[ns] (\id{int}) the number of scalars in \id{c} and vector arrays
    in \id{YY} and \id{ZZ}
  \item[c] (\id{realtype*}) an array of $n_s$ scalars
  \item[XX] (\id{N\_Vector**}) an array of $n_s$ {\nvector} arrays
  \item[Z] (\id{N\_Vector*}) an output array {\nvector} objects
  \end{args}
}
{
  Returns an \id{int} with value \id{0} for success and a non-zero value otherwise.
}
{}

% ====================================================================
\subsection{NVECTOR local reduction functions}\label{ss:nveclocalops}

Local reduction operations are intended to reduce parallel
communication on distributed memory systems, particularly when
{\nvector} objects are combined together within a \\
{\nvecmpimanyvector} object (see Section \ref{ss:nvec_mpimanyvector}).  If a
particular {\nvector} implementation defines a local reduction
operation as \id{NULL}, the {\nvecmpimanyvector} module will
automatically call standard vector reduction operations as necessary
to complete the desired operation. All {\sundials}-provided {\nvector}
implementations include these local reduction operations, which may be
used as templates for user-defined {\nvector} implementations.


\ucfunctionf{N\_VDotProdLocal}
{
  d = N\_VDotProdLocal(x, y);
}
{
  This routine computes the MPI task-local portion of the ordinary dot product of \id{x} and \id{y}:
  \begin{equation*}
    d=\sum_{i=0}^{n_{local}-1} x_i y_i,
  \end{equation*}
  where $n_{local}$ corresponds
  to the number of components in the vector on this MPI task (or
  $n_{local}=n$ for MPI-unaware applications).
}
{
  \begin{args}[x]
  \item[x] (\id{N\_Vector}) a {\nvector} object
  \item[y] (\id{N\_Vector}) a {\nvector} object
  \end{args}
}
{
  \id{realtype}
}
{}

\ucfunctionf{N\_VMaxNormLocal}
{
  m = N\_VMaxNormLocal(x);
}
{
  This routine computes the MPI task-local portion of the maximum norm of the \id{N\_Vector} \id{x}:
  \begin{equation*}
    m = \max_{0\le i< n_{local}} | x_i |,
  \end{equation*}
  where $n_{local}$ corresponds
  to the number of components in the vector on this MPI task (or
  $n_{local}=n$ for MPI-unaware applications).
}
{
  \begin{args}[x]
  \item[x] (\id{N\_Vector}) a {\nvector} object
  \end{args}
}
{
  \id{realtype}
}
{}

\ucfunctionf{N\_VMinLocal}
{
  m = N\_VMinLocal(x);
}
{
  This routine computes the smallest element of the MPI task-local portion of
  the \id{N\_Vector} \id{x}:
  \begin{equation*}
    m = \min_{0\le i< n_{local}} x_i,
  \end{equation*}
  where $n_{local}$ corresponds
  to the number of components in the vector on this MPI task (or
  $n_{local}=n$ for MPI-unaware applications).
}
{
  \begin{args}[x]
  \item[x] (\id{N\_Vector}) a {\nvector} object
  \end{args}
}
{
  \id{realtype}
}
{}

\ucfunctionf{N\_VL1NormLocal}
{
  n = N\_VL1NormLocal(x);
}
{
  This routine computes the MPI task-local portion of the $\ell_1$ norm of the \id{N\_Vector} \id{x}:
  \begin{equation*}
  n = \sum_{i=0}^{n_{local}-1} | x_i |,
  \end{equation*}
  where $n_{local}$ corresponds
  to the number of components in the vector on this MPI task (or
  $n_{local}=n$ for MPI-unaware applications).
}
{
  \begin{args}[x]
  \item[x] (\id{N\_Vector}) a {\nvector} object
  \end{args}
}
{
  \id{realtype}
}
{}

\ucfunctionf{N\_VWSqrSumLocal}
{
  s = N\_VWSqrSumLocal(x,w);
}
{
  This routine computes the MPI task-local portion of the weighted
  squared sum of the \id{N\_Vector} \id{x} with weight vector \id{w}:
  \begin{equation*}
    s = \sum_{i=0}^{n_{local}-1} (x_i w_i)^2,
  \end{equation*}
  where $n_{local}$ corresponds
  to the number of components in the vector on this MPI task (or
  $n_{local}=n$ for MPI-unaware applications).
}
{
  \begin{args}[x]
  \item[x] (\id{N\_Vector}) a {\nvector} object
  \item[w] (\id{N\_Vector}) a {\nvector} object containing weights
  \end{args}
}
{
  \id{realtype}
}
{}

\ucfunctionf{N\_VWSqrSumMaskLocal}
{
  s = N\_VWSqrSumMaskLocal(x,w,id);
}
{
  This routine computes the MPI task-local portion of the weighted
  squared sum of the \id{N\_Vector} \id{x} with weight
  vector \id{w} built using only the elements of \id{x} corresponding to
  positive elements of the \id{N\_Vector} \id{id}:
  \begin{equation*}
    m = \sum_{i=0}^{n_{local}-1} (x_i w_i H(id_i))^2, \quad \text{where} \quad H(\alpha)
  = \begin{cases} 1 & \alpha > 0 \\ 0 & \alpha \leq 0 \end{cases}
  \end{equation*}
  and
  $n_{local}$ corresponds to the number of components in the vector on
  this MPI task (or $n_{local}=n$ for MPI-unaware applications).
}
{
  \begin{args}[x]
  \item[x] (\id{N\_Vector}) a {\nvector} object
  \item[w] (\id{N\_Vector}) a {\nvector} object containing weights
  \item[id] (\id{N\_Vector}) a {\nvector} object used as a mask
  \end{args}
}
{
  \id{realtype}
}
{}

\ucfunctionf{N\_VInvTestLocal}
{
  t = N\_VInvTestLocal(x, z);
}
{
  Sets the MPI task-local components of the \id{N\_Vector} \id{z} to
  be the inverses of the components of the \id{N\_Vector} \id{x}, with
  prior testing for zero values:
  \begin{equation*}
  z_i = 1.0 /  x_i  , \: i=0,\ldots,n_{local}-1,
  \end{equation*}
  where $n_{local}$
  corresponds to the number of components in the vector on this MPI task
  (or $n_{local}=n$ for MPI-unaware applications).
}
{
  \begin{args}[x]
  \item[x] (\id{N\_Vector}) a {\nvector} object
  \item[z] (\id{N\_Vector}) an output {\nvector} object
  \end{args}
}
{
  Returns a \id{booleantype} with the value \id{SUNTRUE} if all task-local
  components of \id{x} are nonzero (successful inversion) and with the
  value \id{SUNFALSE} otherwise.
}
{}

\ucfunctionf{N\_VConstrMaskLocal}
{
  t = N\_VConstrMaskLocal(c,x,m);
}
{
  Performs the following constraint tests:
  {\begin{align*}
  x_i > 0        & \quad \text{if} \quad c_i=2, \\
  x_i \ge 0      & \quad \text{if} \quad c_i=1, \\
  x_i \le 0      & \quad \text{if} \quad c_i=-1, \\
  x_i < 0        & \quad \text{if} \quad c_i=-2, \text{and} \\
  \text{no test} & \quad \text{if} \quad c_i=0,
  \end{align*}}%
  for all MPI task-local components of the vectors.
  It sets a mask vector \id{m}, with elements equal to $1.0$ where
  the constraint test failed, and $0.0$ where the test passed. This
  routine is used only for constraint checking.
}
{
  \begin{args}[c]
  \item[c] (\id{realtype}) scalar constraint value
  \item[x] (\id{N\_Vector}) a {\nvector} object
  \item[m] (\id{N\_Vector}) output mask vector
  \end{args}
}
{
  Returns a \id{booleantype} with the value \id{SUNFALSE} if any
  task-local element failed the constraint test and the value
  \id{SUNTRUE} if all passed.
}
{}

\ucfunctionf{N\_VMinQuotientLocal}
{
  minq = N\_VMinQuotientLocal(num,denom);
}
{
  This routine returns the minimum of the quotients obtained
  by term-wise dividing \id{num}$_i$ by \id{denom}$_i$, for all MPI
  task-local components of the vectors.  A zero element in \id{denom}
  will be skipped. If no such quotients are found, then the large value
  \Id{BIG\_REAL} (defined in the header file \id{sundials\_types.h})
  is returned.
}
{
  \begin{args}[denom]
  \item[num] (\id{N\_Vector}) a {\nvector} object used as the numerator
  \item[denom] (\id{N\_Vector}) a {\nvector} object used as the denominator
  \end{args}
}
{
  \id{realtype}
}
{}


% ====================================================================
\subsection{NVECTOR exchange operations}\label{ss:nvecexchangeops}

The following vector exchange operations are also \textit{optional} and are
intended only for use when interfacing with the XBraid library for
parallel-in-time integration. In that setting these operations are required but
are otherwise unused by SUNDIALS packages and may be set to \id{NULL}. For each
operation, we give the function signature, a description of the expected
behavior, and an example of the function usage.

\ucfunctionf{N\_VBufSize}
{
  flag = N\_VBufSize(N\_Vector x, sunindextype *size);
}
{
  This routine returns the buffer size need to exchange in the data in the
  vector \id{x} between computational nodes.
}
{
  \begin{args}[size]
  \item[x] (\id{N\_Vector}) a {\nvector} object
  \item[size] (\id{sunindextype*}) the size of the message buffer
  \end{args}
}
{
  Returns an \id{int} with value \id{0} for success and a non-zero value otherwise.
}
{}

\ucfunctionf{N\_VBufPack}
{
  flag = N\_VBufPack(N\_Vector x, void *buf);
}
{
  This routine fills the exchange buffer \id{buf} with the vector data in \id{x}.
}
{
  \begin{args}[buf]
  \item[x] (\id{N\_Vector}) a {\nvector} object
  \item[buf] (\id{sunindextype*}) the exchange buffer to pack
  \end{args}
}
{
  Returns an \id{int} with value \id{0} for success and a non-zero value otherwise.
}
{}

\ucfunctionf{N\_VBufUnpack}
{
  flag = N\_VBufUnpack(N\_Vector x, void *buf);
}
{
  This routine unpacks the data in the exchange buffer \id{buf} into the vector
  \id{x}.
}
{
  \begin{args}[buf]
  \item[x] (\id{N\_Vector}) a {\nvector} object
  \item[buf] (\id{sunindextype*}) the exchange buffer to unpack
  \end{args}
}
{
  Returns an \id{int} with value \id{0} for success and a non-zero value otherwise.
}
{}



% ====================================================================
\subsection{NVECTOR utility functions}\label{ss:nvecutils}

To aid in the creation of custom {\nvector} modules the generic {\nvector}
module provides three  utility functions \id{N\_VNewEmpty}, \id{N\_VCopyOps}
and \id{N\_VFreeEmpty}. When used in custom {\nvector} constructors and clone
routines these functions will ease the introduction of any new optional vector
operations to the {\nvector} API by ensuring only required operations need to
be set and all operations are copied when cloning a vector.

To aid the use of arrays of {\nvector} objects, the generic {\nvector} module
also provides the utility functions \ID{N\_VCloneVectorArray},
\ID{N\_VCloneVectorArrayEmpty}, and \ID{N\_VDestroyVectorArray}.


\ucfunctionf{N\_VNewEmpty}
{
  v = N\_VNewEmpty();
}
{
  The function \Id{N\_VNewEmpty} allocates a new generic {\nvector} object and
  initializes its content pointer and the function pointers in the operations
  structure to \id{NULL}.
}
{}
{
  This function returns an \id{N\_Vector} object. If an error occurs when
  allocating the object, then this routine will return \id{NULL}.
}
{}
{}

\ucfunctionf{N\_VCopyOps}
{
  retval = N\_VCopyOps(w, v);
}
{
  The function \Id{N\_VCopyOps} copies the function pointers in the \id{ops}
  structure of \id{w} into the \id{ops} structure of \id{v}.
}
{
  \begin{args}[w]
  \item[w] (\id{N\_Vector}) the vector to copy operations from
  \item[v] (\id{N\_Vector}) the vector to copy operations to
  \end{args}
}
{
  This returns \id{0} if successful and a non-zero value if either of the inputs
  are \id{NULL} or the \id{ops} structure of either input is \id{NULL}.
}
{}

\ucfunctionf{N\_VFreeEmpty}
{
  N\_VFreeEmpty(v);
}
{
  This routine frees the generic \id{N\_Vector} object, under the assumption that any
  implementation-specific data that was allocated within the underlying content structure
  has already been freed. It will additionally test whether the ops pointer is \id{NULL},
  and, if it is not, it will free it as well.
}
{
  \begin{args}[v]
  \item[v] (\id{N\_Vector})
  \end{args}
}
{}
{}

\ucfunction{N\_VCloneEmptyVectorArray}
{
  vecarray = N\_VCloneEmptyVectorArray(count, w);
}
{
  Creates an array of \id{count} variables of type \id{N\_Vector},
  each of the same type as the existing \id{N\_Vector} w. It achieves
  this by calling the implementation-specific \id{N\_VCloneEmpty} operation.
}
{
  \begin{args}[count]
  \item[count] (\id{int}) the size of the vector array
  \item[w] (\id{N\_Vector}) the vector to clone
  \end{args}
}
{
  Returns an array of \id{count} \id{N\_Vector} objects if successful, or
  \id{NULL} if an error occurred while cloning.
}
{}

\ucfunction{N\_VCloneVectorArray}
{
  vecarray = N\_VCloneVectorArray(count, w);
}
{
  Creates an array of \id{count} variables of type \id{N\_Vector},
  each of the same type as the existing \id{N\_Vector} w. It achieves
  this by calling the implementation-specific \id{N\_VClone} operation.
}
{
  \begin{args}[count]
  \item[count] (\id{int}) the size of the vector array
  \item[w] (\id{N\_Vector}) the vector to clone
  \end{args}
}
{
  Returns an array of \id{count} \id{N\_Vector} objects if successful, or
  \id{NULL} if an error occurred while cloning.
}
{}

\ucfunction{N\_VDestroyVectorArray}
{
   N\_VDestroyVectorArray(count, w);
}
{
  Destroys (frees) an array of variables of type \id{N\_Vector}. It
  depends on the implementation-specific \id{N\_VDestroy} operation.
}
{
  \begin{args}[count]
  \item[vs] (\id{N\_Vector*}) the array of vectors to destroy
  \item[count] (\id{int}) the size of the vector array
  \end{args}
}
{}
{}

\ucfunctionf{N\_VNewVectorArray}
{
  vecarray = N\_VNewVectorArray(count);
}
{
  Returns an empty \id{N\_Vector} array large enough to hold \id{count}
  \id{N\_Vector} objects. This function is primarily meant for users of
  the Fortran 2003 interface.
}
{
  \begin{args}[count]
  \item[count] (\id{int}) the size of the vector array
  \end{args}
}
{
  Returns a \id{N\_Vector*} if successful, Returns \id{NULL} if an error occurred.
}
{
  Users of the Fortran 2003 interface to the \id{N\_VManyVector} or
  \id{N\_VMPIManyVector} will need this to create an array to hold
  the subvectors. Note that this function does restrict the the max
  number of subvectors usable with the \id{N\_VManyVector} and
  \id{N\_VMPIManyVector} to the max size of an \id{int} despite the
  ManyVector implementations accepting a subvector count larger than
  this value.
}

\ucfunctionf{N\_VGetVecAtIndexVectorArray}
{
  v = N\_VGetVecAtIndexVectorArray(vecs, index);
}
{
  Returns the \id{N\_Vector} object stored in the vector array at the
  provided index. This function is primarily meant for users of the
  Fortran 2003 interface.
}
{
  \begin{args}[count]
  \item[vecs] (\id{N\_Vector}*) the array of vectors to index
  \item[index] (\id{int}) the index of the vector to return
  \end{args}
}
{
  Returns the \id{N\_Vector} object stored in the vector array at the
  provided index. Returns \id{NULL} if an error occurred.
}
{}

\ucfunctionf{N\_VSetVecAtIndexVectorArray}
{
  N\_VSetVecAtIndexVectorArray(vecs, index, v);
}
{
  Sets the \id{N\_Vector} object stored in the vector array at the
  provided index. This function is primarily meant for users of the
  Fortran 2003 interface.
}
{
  \begin{args}[count]
  \item[vecs] (\id{N\_Vector}*) the array of vectors to index
  \item[index] (\id{int}) the index of the vector to return
  \item[v] (\id{N\_Vector}) the vector to store at the index
  \end{args}
}
{}
{}


% ====================================================================
\subsection{NVECTOR identifiers}
\label{ss:nvecIDs}

Each {\nvector} implementation included in {\sundials} has a
unique identifier specified in enumeration and shown in Table \ref{t:vectorIDs}.

\begin{table}
\centering
\caption{Vector Identifications associated with vector kernels supplied with \id{\sundials}.}
\label{t:vectorIDs}
\medskip
\begin{tabular}{|l|l|c|}
\hline
{\bf Vector ID} & {\bf Vector type} & {\bf ID Value} \\
\hline
SUNDIALS\_NVEC\_SERIAL        & Serial                                        & 0 \\
SUNDIALS\_NVEC\_PARALLEL      & Distributed memory parallel (MPI)             & 1 \\
SUNDIALS\_NVEC\_OPENMP        & OpenMP shared memory parallel                 & 2 \\
SUNDIALS\_NVEC\_PTHREADS      & PThreads shared memory parallel               & 3 \\
SUNDIALS\_NVEC\_PARHYP        & {\hypre} ParHyp parallel vector               & 4 \\
SUNDIALS\_NVEC\_PETSC         & {\petsc} parallel vector                      & 5 \\
SUNDIALS\_NVEC\_CUDA          & {\cuda} vector                                & 6 \\
SUNDIALS\_NVEC\_HIP           & {\hip} vector                                 & 7 \\
SUNDIALS\_NVEC\_SYCL          & {\sycl} vector                                & 8 \\
SUNDIALS\_NVEC\_RAJA          & {\raja} vector                                & 9 \\
SUNDIALS\_NVEC\_OPENMPDEV     & OpenMP vector with device offloading          & 10 \\
SUNDIALS\_NVEC\_TRILINOS      & {\trilinos} Tpetra vector                     & 11 \\
SUNDIALS\_NVEC\_MANYVECTOR    & ``ManyVector'' vector                         & 12 \\
SUNDIALS\_NVEC\_MPIMANYVECTOR & MPI-enabled ``ManyVector'' vector             & 13 \\
SUNDIALS\_NVEC\_MPIPLUSX      & MPI+X vector                                  & 14 \\
SUNDIALS\_NVEC\_CUSTOM        & User-provided custom vector                   & 15 \\
\hline
\end{tabular}
\end{table}


% ====================================================================
\subsection{The generic NVECTOR module implementation}
\label{ss:nvec_impl_details}

The generic \ID{N\_Vector} type is a pointer to a structure that has an
implementation-dependent {\em content} field containing the
description and actual data of the vector, and an {\em ops} field
pointing to a structure with generic vector operations.
The type \id{N\_Vector} is defined as
%%
%%
\begin{verbatim}
typedef struct _generic_N_Vector *N_Vector;

struct _generic_N_Vector {
    void *content;
    struct _generic_N_Vector_Ops *ops;
};
\end{verbatim}
%%
%%
The \id{\_generic\_N\_Vector\_Ops} structure is essentially a list of pointers to
the various actual vector operations, and is defined as
%%
\begin{verbatim}
struct _generic_N_Vector_Ops {
  N_Vector_ID  (*nvgetvectorid)(N_Vector);
  N_Vector     (*nvclone)(N_Vector);
  N_Vector     (*nvcloneempty)(N_Vector);
  void         (*nvdestroy)(N_Vector);
  void         (*nvspace)(N_Vector, sunindextype *, sunindextype *);
  realtype*    (*nvgetarraypointer)(N_Vector);
  realtype*    (*nvgetdevicearraypointer)(N_Vector);
  void         (*nvsetarraypointer)(realtype *, N_Vector);
  void*        (*nvgetcommunicator)(N_Vector);
  sunindextype (*nvgetlength)(N_Vector);
  void         (*nvlinearsum)(realtype, N_Vector, realtype, N_Vector, N_Vector);
  void         (*nvconst)(realtype, N_Vector);
  void         (*nvprod)(N_Vector, N_Vector, N_Vector);
  void         (*nvdiv)(N_Vector, N_Vector, N_Vector);
  void         (*nvscale)(realtype, N_Vector, N_Vector);
  void         (*nvabs)(N_Vector, N_Vector);
  void         (*nvinv)(N_Vector, N_Vector);
  void         (*nvaddconst)(N_Vector, realtype, N_Vector);
  realtype     (*nvdotprod)(N_Vector, N_Vector);
  realtype     (*nvmaxnorm)(N_Vector);
  realtype     (*nvwrmsnorm)(N_Vector, N_Vector);
  realtype     (*nvwrmsnormmask)(N_Vector, N_Vector, N_Vector);
  realtype     (*nvmin)(N_Vector);
  realtype     (*nvwl2norm)(N_Vector, N_Vector);
  realtype     (*nvl1norm)(N_Vector);
  void         (*nvcompare)(realtype, N_Vector, N_Vector);
  booleantype  (*nvinvtest)(N_Vector, N_Vector);
  booleantype  (*nvconstrmask)(N_Vector, N_Vector, N_Vector);
  realtype     (*nvminquotient)(N_Vector, N_Vector);
  int          (*nvlinearcombination)(int, realtype*, N_Vector*, N_Vector);
  int          (*nvscaleaddmulti)(int, realtype*, N_Vector, N_Vector*, N_Vector*);
  int          (*nvdotprodmulti)(int, N_Vector, N_Vector*, realtype*);
  int          (*nvlinearsumvectorarray)(int, realtype, N_Vector*, realtype,
                                         N_Vector*, N_Vector*);
  int          (*nvscalevectorarray)(int, realtype*, N_Vector*, N_Vector*);
  int          (*nvconstvectorarray)(int, realtype, N_Vector*);
  int          (*nvwrmsnomrvectorarray)(int, N_Vector*, N_Vector*, realtype*);
  int          (*nvwrmsnomrmaskvectorarray)(int, N_Vector*, N_Vector*, N_Vector,
                                            realtype*);
  int          (*nvscaleaddmultivectorarray)(int, int, realtype*, N_Vector*,
                                             N_Vector**, N_Vector**);
  int          (*nvlinearcombinationvectorarray)(int, int, realtype*, N_Vector**,
                                                 N_Vector*);
  realtype     (*nvdotprodlocal)(N_Vector, N_Vector);
  realtype     (*nvmaxnormlocal)(N_Vector);
  realtype     (*nvminlocal)(N_Vector);
  realtype     (*nvl1normlocal)(N_Vector);
  booleantype  (*nvinvtestlocal)(N_Vector, N_Vector);
  booleantype  (*nvconstrmasklocal)(N_Vector, N_Vector, N_Vector);
  realtype     (*nvminquotientlocal)(N_Vector, N_Vector);
  realtype     (*nvwsqrsumlocal)(N_Vector, N_Vector);
  realtype     (*nvwsqrsummasklocal(N_Vector, N_Vector, N_Vector);
  int          (*nvbufsize)(N_Vector, sunindextype *);
  int          (*nvbufpack)(N_Vector, void*);
  int          (*nvbufunpack)(N_Vector, void*);
};
\end{verbatim}

The generic {\nvector} module defines and implements the vector operations
acting on an \id{N\_Vector}. These routines are nothing but wrappers for
the vector operations defined by a particular {\nvector} implementation,
which are accessed through the {\em ops} field of the \id{N\_Vector}
structure. To illustrate this point we show below the implementation of a
typical vector operation from the generic {\nvector} module, namely \id{N\_VScale},
which performs the scaling of a vector \id{x} by a scalar \id{c}:
%%
%%
\begin{verbatim}
void N_VScale(realtype c, N_Vector x, N_Vector z)
{
   z->ops->nvscale(c, x, z);
}
\end{verbatim}
%%
%%
Section \ref{ss:nvecops} defines a complete list of all standard vector operations
defined by the generic {\nvector} module. Sections \ref{ss:nvecfusedops},
\ref{ss:nvecarrayops} and \ref{ss:nveclocalops} list \textit{optional} fused,
vector array and local reduction operations, respectively.


The Fortran 2003 interface provides a \id{bind(C)} derived-type for the
\id{\_generic\_N\_Vector} and the \id{\_generic\_N\_Vector\_Ops} structures.
Their definition is given below.
%%
%%
\begin{verbatim}
 type, bind(C), public :: N_Vector
  type(C_PTR), public :: content
  type(C_PTR), public :: ops
 end type N_Vector

 type, bind(C), public :: N_Vector_Ops
  type(C_FUNPTR), public :: nvgetvectorid
  type(C_FUNPTR), public :: nvclone
  type(C_FUNPTR), public :: nvcloneempty
  type(C_FUNPTR), public :: nvdestroy
  type(C_FUNPTR), public :: nvspace
  type(C_FUNPTR), public :: nvgetarraypointer
  type(C_FUNPTR), public :: nvsetarraypointer
  type(C_FUNPTR), public :: nvgetcommunicator
  type(C_FUNPTR), public :: nvgetlength
  type(C_FUNPTR), public :: nvlinearsum
  type(C_FUNPTR), public :: nvconst
  type(C_FUNPTR), public :: nvprod
  type(C_FUNPTR), public :: nvdiv
  type(C_FUNPTR), public :: nvscale
  type(C_FUNPTR), public :: nvabs
  type(C_FUNPTR), public :: nvinv
  type(C_FUNPTR), public :: nvaddconst
  type(C_FUNPTR), public :: nvdotprod
  type(C_FUNPTR), public :: nvmaxnorm
  type(C_FUNPTR), public :: nvwrmsnorm
  type(C_FUNPTR), public :: nvwrmsnormmask
  type(C_FUNPTR), public :: nvmin
  type(C_FUNPTR), public :: nvwl2norm
  type(C_FUNPTR), public :: nvl1norm
  type(C_FUNPTR), public :: nvcompare
  type(C_FUNPTR), public :: nvinvtest
  type(C_FUNPTR), public :: nvconstrmask
  type(C_FUNPTR), public :: nvminquotient
  type(C_FUNPTR), public :: nvlinearcombination
  type(C_FUNPTR), public :: nvscaleaddmulti
  type(C_FUNPTR), public :: nvdotprodmulti
  type(C_FUNPTR), public :: nvlinearsumvectorarray
  type(C_FUNPTR), public :: nvscalevectorarray
  type(C_FUNPTR), public :: nvconstvectorarray
  type(C_FUNPTR), public :: nvwrmsnormvectorarray
  type(C_FUNPTR), public :: nvwrmsnormmaskvectorarray
  type(C_FUNPTR), public :: nvscaleaddmultivectorarray
  type(C_FUNPTR), public :: nvlinearcombinationvectorarray
  type(C_FUNPTR), public :: nvdotprodlocal
  type(C_FUNPTR), public :: nvmaxnormlocal
  type(C_FUNPTR), public :: nvminlocal
  type(C_FUNPTR), public :: nvl1normlocal
  type(C_FUNPTR), public :: nvinvtestlocal
  type(C_FUNPTR), public :: nvconstrmasklocal
  type(C_FUNPTR), public :: nvminquotientlocal
  type(C_FUNPTR), public :: nvwsqrsumlocal
  type(C_FUNPTR), public :: nvwsqrsummasklocal
  type(C_FUNPTR), public :: nvbufsize
  type(C_FUNPTR), public :: nvbufpack
  type(C_FUNPTR), public :: nvbufunpack
 end type N_Vector_Ops
\end{verbatim}

% =====================================================================
\subsection{Implementing a custom NVECTOR}
\label{ss:nvector_custom_implmentation}

A particular implementation of the {\nvector} module must:

\begin{itemize}
\item Specify the {\em content} field of \id{N\_Vector}.
\item Define and implement the vector operations.
  Note that the names of these routines should be unique to that implementation in order
  to permit using more than one {\nvector} module (each with different \id{N\_Vector}
  internal data representations) in the same code.
\item Define and implement user-callable constructor and destructor
  routines to create and free an \id{N\_Vector} with
  the new {\em content} field and with {\em ops} pointing to the
  new vector operations.
\item Optionally, define and implement additional user-callable routines
  acting on the newly defined \id{N\_Vector} (e.g., a routine to print
  the content for debugging purposes).
\item Optionally, provide accessor macros as needed for that particular implementation to
  be used to access different parts in the {\em content} field of the newly defined \id{N\_Vector}.
\end{itemize}

It is recommended that a user-supplied {\nvector} implementation returns the
\id{SUNDIALS\_NVEC\_CUSTOM} identifier from the \id{N\_VGetVectorID} function.

To aid in the creation of custom {\nvector} modules the generic {\nvector}
module provides two utility functions \id{N\_VNewEmpty} and \id{N\_VCopyOps}.
When used in custom {\nvector} constructors and clone routines these functions
will ease the introduction of any new optional vector operations to the
{\nvector} API by ensuring only required operations need to be set and all
operations are copied when cloning a vector.


\subsubsection{Support for complex-valued vectors}

While {\sundials} itself is written under an assumption of real-valued
data, it does provide limited support for complex-valued problems.
However, since none of the built-in {\nvector} modules supports
complex-valued data, users must provide a custom {\nvector}
implementation for this task.  Many of the {\nvector} routines
described in Sections \ref{ss:nvecops}-\ref{ss:nveclocalops} above
naturally extend to complex-valued vectors; however, some do not.  To
this end, we provide the following guidance:

\begin{itemize}
\item \id{N\_VMin} and \id{N\_VMinLocal} should return the minimum of
  all \emph{real} components of the vector, i.e.,  $m = \min_i
  \operatorname{real}(x_i) $.

\item \id{N\_VConst} (and similarly \id{N\_VConstVectorArray}) should
  set the real components of the vector to the input constant, and set
  all imaginary components to zero, i.e.,
  $z_i = c + 0 j,\: i=0,\ldots,n-1$.

\item \id{N\_VAddConst} should only update the real components of the
  vector with the input constant, leaving all imaginary components
  unchanged.

\item \id{N\_VWrmsNorm}, \id{N\_VWrmsNormMask}, \id{N\_VWSqrSumLocal}
  and \id{N\_VWSqrSumMaskLocal} should assume that all entries of the
  weight vector \id{w} and the mask vector \id{id} are real-valued.

\item \id{N\_VDotProd} should mathematically return a complex number
  for complex-valued vectors; as this is not possible with
  {\sundials}' current \id{realtype}, this routine should
  be set to \id{NULL} in the custom {\nvector} implementation.

\item \id{N\_VCompare}, \id{N\_VConstrMask}, \id{N\_VMinQuotient},
  \id{N\_VConstrMaskLocal} and \id{N\_VMinQuotientLocal}
  are ill-defined due to the lack of a clear ordering in the
  complex plane.  These routines should be set to \id{NULL}
  in the custom {\nvector} implementation.

\end{itemize}

While many {\sundials} solver modules may be utilized on
complex-valued data, others cannot.  Specifically, although both
{\sunnonlinsolnewton} and {\sunnonlinsolfixedpoint} may be used with
any of the IVP solvers ({\cvode}, {\cvodes}, {\ida}, {\idas} and
{\arkode}) for complex-valued problems, the Anderson-acceleration
feature {\sunnonlinsolfixedpoint} cannot be used due to its reliance
on \id{N\_VDotProd}.  By this same logic, the Anderson acceleration
feature within {\kinsol} also will not work with complex-valued
vectors.

Similarly, although each package's linear solver interface (e.g.,
{\cvls}) may be used on complex-valued problems, none of the built-in
{\sunmatrix} or {\sunlinsol} modules work.  Hence a complex-valued
user should provide a custom {\sunlinsol} (and optionally a custom
{\sunmatrix}) implementation for solving linear systems, and then
attach this module as normal to the package's linear solver
interface.

Finally, constraint-handling features of each package cannot be used
for complex-valued data, due to the issue of
ordering in the complex plane discussed above with
\id{N\_VCompare}, \id{N\_VConstrMask}, \id{N\_VMinQuotient},
\id{N\_VConstrMaskLocal} and \id{N\_VMinQuotientLocal}.

We provide a simple example of a complex-valued example problem,
including a custom complex-valued Fortran 2003 {\nvector} module, in the
files
\newline\noindent\id{examples/arkode/F2003\_custom/ark\_analytic\_complex\_f2003.f90},
\newline\noindent\id{examples/arkode/F2003\_custom/fnvector\_complex\_mod.f90}, and
\newline\noindent\id{examples/arkode/F2003\_custom/test\_fnvector\_complex\_mod.f90}.



%---------------------------------------------------------------------------
\section{NVECTOR functions used by IDAS}

In Table \ref{t:nvecuse} below, we list the vector functions used in the 
{\nvector} module used by the {\idas} package.
The table also shows, for each function, which of the code modules uses
the function. The {\idas} column shows function usage within the main
integrator module, while the remaining columns show function usage
within the {\idas} linear solvers interface, the {\idabbdpre}
preconditioner module, and the {\idaa} module.

At this point, we should emphasize that the {\idas} user does not need to know 
anything about the usage of vector functions by the {\idas} code modules in order 
to use {\idas}. The information is presented as an implementation detail for the 
interested reader.

\begin{table}[htb]
\centering
\caption{List of vector functions usage by {\idas} code modules}\label{t:nvecuse}
\medskip
\begin{tabular}{|r|c|c|c|c|c|} \hline
                                            & 
\begin{sideways}{\idas}      \end{sideways} & 
\begin{sideways}{\idals}     \end{sideways} & 
\begin{sideways}{\idabbdpre} \end{sideways} &
\begin{sideways}{\idaa}      \end{sideways}  \\ \hline\hline
%                                      IDAS   LS   BBD   IDAA
\id{N\_VGetVectorID}                  &     &     &     &     \\ \hline
\id{N\_VClone}                        & \cm & \cm & \cm & \cm \\ \hline
\id{N\_VCloneEmpty}                   &     &  1  &     &     \\ \hline
\id{N\_VDestroy}                      & \cm & \cm & \cm & \cm \\ \hline
\id{N\_VCloneVectorArray}             & \cm &     &     & \cm \\ \hline
\id{N\_VDestroyVectorArray}           & \cm &     &     & \cm \\ \hline
\id{N\_VSpace}                        & \cm &  2  &     &     \\ \hline
\id{N\_VGetArrayPointer}              &     &  1  & \cm &     \\ \hline
\id{N\_VSetArrayPointer}              &     &  1  &     &     \\ \hline
\id{N\_VLinearSum}                    & \cm & \cm &     & \cm \\ \hline
\id{N\_VConst}                        & \cm & \cm &     & \cm \\ \hline
\id{N\_VProd}                         & \cm &     &     &     \\ \hline
\id{N\_VDiv}                          & \cm &     &     &     \\ \hline
\id{N\_VScale}                        & \cm & \cm & \cm & \cm \\ \hline
\id{N\_VAbs}                          & \cm &     &     &     \\ \hline
\id{N\_VInv}                          & \cm &     &     &     \\ \hline
\id{N\_VAddConst}                     & \cm &     &     &     \\ \hline
\id{N\_VDotProd}                      &     & \cm &     &     \\ \hline
\id{N\_VMaxNorm}                      & \cm &     &     &     \\ \hline
\id{N\_VWrmsNorm}                     & \cm & \cm &     &     \\ \hline
\id{N\_VMin}                          & \cm &     &     &     \\ \hline
\id{N\_VMinQuotient}                  & \cm &     &     &     \\ \hline
\id{N\_VConstrMask}                   & \cm &     &     &     \\ \hline
\id{N\_VWrmsNormMask}                 & \cm &     &     &     \\ \hline
\id{N\_VCompare}                      & \cm &     &     &     \\ \hline
\hline
\id{N\_VLinearCombination}            & \cm &     &     &     \\ \hline 
\id{N\_VScaleAddMulti}                & \cm &     &     &     \\ \hline 
\id{N\_VDotProdMulti}                 &     &  3  &     &     \\ \hline 
\hline
\id{N\_VLinearSumVectorArray}         & \cm &     &     &     \\ \hline 
\id{N\_VScaleVectorArray}             & \cm &     &     &     \\ \hline 
\id{N\_VConstVectorArray}             & \cm &     &     &     \\ \hline 
\id{N\_VWrmsNormVectorArray}          & \cm &     &     &     \\ \hline 
\id{N\_VWrmsNormMaskVectorArray}      & \cm &     &     &     \\ \hline 
\id{N\_VScaleAddMultiVectorArray}     & \cm &     &     &     \\ \hline 
\id{N\_VLinearCombinationVectorArray} & \cm &     &     &     \\ \hline 
\end{tabular}
\end{table}

Special cases (numbers match markings in table):
\begin{enumerate}
\item These routines are only required if an internal
  difference-quotient routine for constructing dense or band
  Jacobian matrices is used.
\item This routine is optional, and is only used in estimating
  space requirements for {\idas} modules for user feedback.
\item The optional function \id{N\_VDotProdMulti} is only used
  when Classical Gram-Schmidt is enabled with {\spgmr} or
  {\spfgmr}. The remaining operations from Tables \ref{t:nvecfusedops}
  and \ref{t:nvecarrayops} not listed above are unused and a
  user-supplied {\nvector} module for {\idas} could omit these
  operations.
\end{enumerate}

Of the functions listed in Table \ref{t:nvecops}, \id{N\_VWL2Norm}, 
\id{N\_VL1Norm}, \id{N\_VInvTest}, \id{N\_VGetCommunicator}, and
\id{N\_VGetLength} are {\em not} used by 
{\idas}. Therefore a user-supplied {\nvector} module for {\idas} could
omit these functions. 

%---------------------------------------------------------------------------
% nvector module sections
%---------------------------------------------------------------------------

% This is a shared SUNDIALS TEX file with a description of the
%% serial nvector implementation
%%
\section{The NVECTOR\_SERIAL implementation}\label{ss:nvec_ser}

The serial implementation of the {\nvector} module provided with {\sundials},
{\nvecs}, defines the {\em content} field of \id{N\_Vector} to be a structure
containing the length of the vector, a pointer to the beginning of a contiguous
data array, and a boolean flag {\em own\_data} which specifies the ownership
of {\em data}.
%%
\begin{verbatim}
struct _N_VectorContent_Serial {
  sunindextype length;
  booleantype own_data;
  realtype *data;
};
\end{verbatim}
%%
%%--------------------------------------------
%%

The header file to include when using this module is \id{nvector\_serial.h}.
The installed module library to link to is
\id{libsundials\_nvecserial.\textit{lib}}
where \id{\em.lib} is typically \id{.so} for shared libraries and \id{.a}
for static libraries.


% ====================================================================
\subsection{NVECTOR\_SERIAL accessor macros}
\label{ss:nvec_ser_macros}
% ====================================================================

The following macros are provided to access the content of an {\nvecs}
vector. The suffix \id{\_S} in the names denotes the serial version.
%%
\begin{itemize}

\item \ID{NV\_CONTENT\_S}

  This routine gives access to the contents of the serial
  vector \id{N\_Vector}.

  The assignment \id{v\_cont} $=$ \id{NV\_CONTENT\_S(v)} sets
  \id{v\_cont} to be a pointer to the serial \id{N\_Vector} content
  structure.

  Implementation:

  \verb|#define NV_CONTENT_S(v) ( (N_VectorContent_Serial)(v->content) )|

\item \ID{NV\_OWN\_DATA\_S}, \ID{NV\_DATA\_S}, \ID{NV\_LENGTH\_S}


  These macros give individual access to the parts of
  the content of a serial \id{N\_Vector}.

  The assignment \id{v\_data = NV\_DATA\_S(v)} sets \id{v\_data} to be
  a pointer to the first component of the data for the \id{N\_Vector} \id{v}.
  The assignment \id{NV\_DATA\_S(v) = v\_data} sets the component array of \id{v} to
  be \id{v\_data} by storing the pointer \id{v\_data}.

  The assignment \id{v\_len = NV\_LENGTH\_S(v)} sets \id{v\_len} to be
  the length of \id{v}. On the other hand, the call \id{NV\_LENGTH\_S(v) = len\_v}
  sets the length of \id{v} to be \id{len\_v}.

  Implementation:

  \verb|#define NV_OWN_DATA_S(v) ( NV_CONTENT_S(v)->own_data )|

  \verb|#define NV_DATA_S(v) ( NV_CONTENT_S(v)->data )|

  \verb|#define NV_LENGTH_S(v) ( NV_CONTENT_S(v)->length )|

\item \ID{NV\_Ith\_S}

  This macro gives access to the individual components of the data
  array of an \id{N\_Vector}.

  The assignment \id{r = NV\_Ith\_S(v,i)} sets \id{r} to be the value of
  the \id{i}-th component of \id{v}. The assignment \id{NV\_Ith\_S(v,i) = r}
  sets the value of the \id{i}-th component of \id{v} to be \id{r}.

  Here $i$ ranges from $0$ to $n-1$ for a vector of length $n$.

  Implementation:

  \verb|#define NV_Ith_S(v,i) ( NV_DATA_S(v)[i] )|

\end{itemize}


% ====================================================================
\subsection{NVECTOR\_SERIAL functions}
\label{ss:nvec_ser_functions}
% ====================================================================

The {\nvecs} module defines serial implementations of all vector operations listed
in Tables \ref{ss:nvecops}, \ref{ss:nvecfusedops}, \ref{ss:nvecarrayops}
and \ref{ss:nveclocalops}. Their
names are obtained from those in these tables by appending the suffix \id{\_Serial}
(e.g. \id{N\_VDestroy\_Serial}).
All the standard vector operations listed in \ref{ss:nvecops} with the suffix
\id{\_Serial} appended are callable via the {\F} 2003 interface by prepending an
`F' (e.g. \id{FN\_VDestroy\_Serial}).

The module {\nvecs} provides the following additional user-callable routines:
%%--------------------------------------
\sunmodfunf{N\_VNew\_Serial}
{
  This function creates and allocates memory for a serial \id{N\_Vector}.
  Its only argument is the vector length.
}
{
  N\_Vector N\_VNew\_Serial(sunindextype vec\_length);
}
%%--------------------------------------
\sunmodfunf{N\_VNewEmpty\_Serial}
{
  This function creates a new serial \id{N\_Vector} with an empty (\id{NULL})
  data array.
}
{
  N\_Vector N\_VNewEmpty\_Serial(sunindextype vec\_length);
}
%%--------------------------------------
\sunmodfunf{N\_VMake\_Serial}
{
  This function creates and allocates memory for a serial vector
  with user-provided data array.

  (This function does {\em not} allocate memory for \id{v\_data} itself.)
}
{
  N\_Vector N\_VMake\_Serial(sunindextype vec\_length, realtype *v\_data);
}
%%--------------------------------------
\sunmodfunf{N\_VCloneVectorArray\_Serial}
{
  This function creates (by cloning) an array of \id{count} serial vectors.
}
{
  N\_Vector *N\_VCloneVectorArray\_Serial(int count, N\_Vector w);
}
%%--------------------------------------
\sunmodfunf{N\_VCloneVectorArrayEmpty\_Serial}
{
  This function creates (by cloning) an array of \id{count} serial vectors,
  each with an empty (\id{NULL}) data array.
}
{
  N\_Vector *N\_VCloneVectorArrayEmpty\_Serial(int count, N\_Vector w);
}
%%--------------------------------------
\sunmodfunf{N\_VDestroyVectorArray\_Serial}
{
  This function frees memory allocated for the array of \id{count} variables of type
  \id{N\_Vector} created with \id{N\_VCloneVectorArray\_Serial} or with \newline
  \id{N\_VCloneVectorArrayEmpty\_Serial}.
}
{
  void N\_VDestroyVectorArray\_Serial(N\_Vector *vs, int count);
}
%%--------------------------------------
\sunmodfunf{N\_VPrint\_Serial}
{
  This function prints the content of a serial vector to \id{stdout}.
}
{
  void N\_VPrint\_Serial(N\_Vector v);
}
%%--------------------------------------
\sunmodfunf{N\_VPrintFile\_Serial}
{
  This function prints the content of a serial vector to \id{outfile}.
}
{
  void N\_VPrintFile\_Serial(N\_Vector v, FILE *outfile);
}
%%--------------------------------------

By default all fused and vector array operations are disabled in the {\nvecs}
module. The following additional user-callable routines are provided to
enable or disable fused and vector array operations for a specific vector. To
ensure consistency across vectors it is recommended to first create a vector
with \id{N\_VNew\_Serial}, enable/disable the desired operations for that vector
with the functions below, and create any additional vectors from that vector
using \id{N\_VClone}. This guarantees the new vectors will have the same
operations enabled/disabled as cloned vectors inherit the same enable/disable
options as the vector they are cloned from while vectors created with
\id{N\_VNew\_Serial} will have the default settings for the {\nvecs} module.
%%--------------------------------------
\sunmodfunf{N\_VEnableFusedOps\_Serial}
{
  This function enables (\id{SUNTRUE}) or disables (\id{SUNFALSE}) all fused and
  vector array operations in the serial vector. The return value is \id{0} for
  success and \id{-1} if the input vector or its \id{ops} structure are \id{NULL}.
}
{
  int N\_VEnableFusedOps\_Serial(N\_Vector v, booleantype tf);
}
%%--------------------------------------
\sunmodfunf{N\_VEnableLinearCombination\_Serial}
{
  This function enables (\id{SUNTRUE}) or disables (\id{SUNFALSE}) the linear
  combination fused operation in the serial vector. The return value is \id{0} for
  success and \id{-1} if the input vector or its \id{ops} structure are \id{NULL}.
}
{
  int N\_VEnableLinearCombination\_Serial(N\_Vector v, booleantype tf);
}
%%--------------------------------------
\sunmodfunf{N\_VEnableScaleAddMulti\_Serial}
{
  This function enables (\id{SUNTRUE}) or disables (\id{SUNFALSE}) the scale and
  add a vector to multiple vectors fused operation in the serial vector. The
  return value is \id{0} for success and \id{-1} if the input vector or its
  \id{ops} structure are \id{NULL}.
}
{
  int N\_VEnableScaleAddMulti\_Serial(N\_Vector v, booleantype tf);
}
%%--------------------------------------
\sunmodfunf{N\_VEnableDotProdMulti\_Serial}
{
  This function enables (\id{SUNTRUE}) or disables (\id{SUNFALSE}) the multiple
  dot products fused operation in the serial vector. The return value is \id{0}
  for success and \id{-1} if the input vector or its \id{ops} structure are
  \id{NULL}.
}
{
  int N\_VEnableDotProdMulti\_Serial(N\_Vector v, booleantype tf);
}
%%--------------------------------------
\sunmodfunf{N\_VEnableLinearSumVectorArray\_Serial}
{
  This function enables (\id{SUNTRUE}) or disables (\id{SUNFALSE}) the linear sum
  operation for vector arrays in the serial vector. The return value is \id{0} for
  success and \id{-1} if the input vector or its \id{ops} structure are \id{NULL}.
}
{
  int N\_VEnableLinearSumVectorArray\_Serial(N\_Vector v, booleantype tf);
}
%%--------------------------------------
\sunmodfunf{N\_VEnableScaleVectorArray\_Serial}
{
  This function enables (\id{SUNTRUE}) or disables (\id{SUNFALSE}) the scale
  operation for vector arrays in the serial vector. The return value is \id{0} for
  success and \id{-1} if the input vector or its \id{ops} structure are \id{NULL}.
}
{
  int N\_VEnableScaleVectorArray\_Serial(N\_Vector v, booleantype tf);
}
%%--------------------------------------
\sunmodfunf{N\_VEnableConstVectorArray\_Serial}
{
  This function enables (\id{SUNTRUE}) or disables (\id{SUNFALSE}) the const
  operation for vector arrays in the serial vector. The return value is \id{0} for
  success and \id{-1} if the input vector or its \id{ops} structure are \id{NULL}.
}
{
  int N\_VEnableConstVectorArray\_Serial(N\_Vector v, booleantype tf);
}
%%--------------------------------------
\sunmodfunf{N\_VEnableWrmsNormVectorArray\_Serial}
{
  This function enables (\id{SUNTRUE}) or disables (\id{SUNFALSE}) the WRMS norm
  operation for vector arrays in the serial vector. The return value is \id{0} for
  success and \id{-1} if the input vector or its \id{ops} structure are \id{NULL}.
}
{
  int N\_VEnableWrmsNormVectorArray\_Serial(N\_Vector v, booleantype tf);
}
%%--------------------------------------
\sunmodfunf{N\_VEnableWrmsNormMaskVectorArray\_Serial}
{
  This function enables (\id{SUNTRUE}) or disables (\id{SUNFALSE}) the masked WRMS
  norm operation for vector arrays in the serial vector. The return value is
  \id{0} for success and \id{-1} if the input vector or its \id{ops} structure are
  \id{NULL}.
}
{
  int N\_VEnableWrmsNormMaskVectorArray\_Serial(N\_Vector v, booleantype tf);
}
%%--------------------------------------
\sunmodfun{N\_VEnableScaleAddMultiVectorArray\_Serial}
{
  This function enables (\id{SUNTRUE}) or disables (\id{SUNFALSE}) the scale and
  add a vector array to multiple vector arrays operation in the serial vector. The
  return value is \id{0} for success and \id{-1} if the input vector or its
  \id{ops} structure are \id{NULL}.
}
{
  int N\_VEnableScaleAddMultiVectorArray\_Serial(N\_Vector v,
  \newlinefill{int N\_VEnableScaleAddMultiVectorArray\_Serial}
  booleantype tf);
}
%%--------------------------------------
\sunmodfun{N\_VEnableLinearCombinationVectorArray\_Serial}
{
  This function enables (\id{SUNTRUE}) or disables (\id{SUNFALSE}) the linear
  combination operation for vector arrays in the serial vector. The return value
  is \id{0} for success and \id{-1} if the input vector or its \id{ops} structure
  are \id{NULL}.
}
{
  int N\_VEnableLinearCombinationVectorArray\_Serial(N\_Vector v,
  \newlinefill{int N\_VEnableLinearCombinationVectorArray\_Serial}
  booleantype tf);
}
%%
%%------------------------------------
%%
\paragraph{\bf Notes}

\begin{itemize}

\item
  When looping over the components of an \id{N\_Vector} \id{v}, it is
  more efficient to first obtain the component array via
  \id{v\_data = NV\_DATA\_S(v)} and then access \id{v\_data[i]} within the
  loop than it is to use \id{NV\_Ith\_S(v,i)} within the loop.

\item
  {\warn}\id{N\_VNewEmpty\_Serial}, \id{N\_VMake\_Serial},
  and \id{N\_VCloneVectorArrayEmpty\_Serial} set the field
  {\em own\_data} $=$ \id{SUNFALSE}.
  \id{N\_VDestroy\_Serial} and \id{N\_VDestroyVectorArray\_Serial}
  will not attempt to free the pointer {\em data} for any \id{N\_Vector} with
  {\em own\_data} set to \id{SUNFALSE}. In such a case, it is the user's responsibility to
  deallocate the {\em data} pointer.

\item
  {\warn}To maximize efficiency, vector operations in the {\nvecs} implementation
  that have more than one \id{N\_Vector} argument do not check for
  consistent internal representation of these vectors. It is the user's
  responsibility to ensure that such routines are called with \id{N\_Vector}
  arguments that were all created with the same internal representations.

\end{itemize}


% ====================================================================
\subsection{NVECTOR\_SERIAL Fortran interfaces}
\label{ss:nvec_ser_fortran}
% ====================================================================

The {\nvecs} module provides a {\F} 2003 module as well as {\F} 77
style interface functions for use from {\F} applications.

\subsubsection*{FORTRAN 2003 interface module}
The \ID{fnvector\_serial\_mod} {\F} module defines interfaces to all
{\nvecs} {\CC} functions using the intrinsic \id{iso\_c\_binding}
module which provides a standardized mechanism for interoperating with {\CC}. As
noted in the {\CC} function descriptions above, the interface functions are
named after the corresponding {\CC} function, but with a leading `F'. For
example, the function \id{N\_VNew\_Serial} is interfaced as
\id{FN\_VNew\_Serial}.

The {\F} 2003 {\nvecs} interface module can be accessed with the \id{use}
statement, i.e. \id{use fnvector\_serial\_mod}, and linking to the library
\id{libsundials\_fnvectorserial\_mod}.{\em lib} in addition to the {\CC} library.
For details on where the library and module file
\id{fnvector\_serial\_mod.mod} are installed see Appendix \ref{c:install}.
We note that the module is accessible from the {\F} 2003 {\sundials} integrators
\textit{without} separately linking to the
\id{libsundials\_fnvectorserial\_mod} library.

\subsubsection*{FORTRAN 77 interface functions}
For solvers that include a {\F} 77 interface module, the {\nvecs} module
also includes a {\F}-callable function \id{FNVINITS(code, NEQ, IER)},
to initialize this {\nvecs} module.  Here \id{code} is an input solver id
(1 for {\cvode}, 2 for {\ida}, 3 for {\kinsol}, 4 for {\arkode}); NEQ is
the problem size (declared so as to match C type \id{long int}); and
IER is an error return flag equal 0 for success and -1 for failure.

% This is a shared SUNDIALS TEX file with description of
% the MPI parallel nvector implementation
%
\section{The NVECTOR\_PARALLEL implementation}\label{ss:nvec_par}

The {\nvecp} implementation of the {\nvector} module provided with
{\sundials} is based on {\mpi}.  It defines the {\em content}
field of \id{N\_Vector} to be a structure containing the global and local lengths
of the vector, a pointer to the beginning of a contiguous local data array,
an {\mpi} communicator, and a boolean flag {\em own\_data} indicating ownership of
the data array {\em data}.
%%
\begin{verbatim}
struct _N_VectorContent_Parallel {
  sunindextype local_length;
  sunindextype global_length;
  booleantype own_data;
  realtype *data;
  MPI_Comm comm;
};
\end{verbatim}
%%
%%--------------------------------------------

The header file to include when using this module is \id{nvector\_parallel.h}.
The installed module library to link to is
\id{libsundials\_nvecparallel.\textit{lib}}
where \id{\em.lib} is typically \id{.so} for shared libraries and \id{.a}
for static libraries.


% ====================================================================
\subsection{NVECTOR\_PARALLEL accessor macros}
\label{ss:nvec_par_macros}
% ====================================================================

The following macros are provided to access the content of a {\nvecp}
vector. The suffix \id{\_P} in the names denotes the distributed memory
parallel version.
\begin{itemize}

\item
  \ID{NV\_CONTENT\_P}

  This macro gives access to the contents of the parallel
  vector \id{N\_Vector}.

  The assignment \id{v\_cont = NV\_CONTENT\_P(v)} sets
  \id{v\_cont} to be a pointer to the \id{N\_Vector} content
  structure of type \id{struct \_N\_VectorContent\_Parallel}.

  Implementation:

  \verb|#define NV_CONTENT_P(v) ( (N_VectorContent_Parallel)(v->content) )|

\item
  \ID{NV\_OWN\_DATA\_P}, \ID{NV\_DATA\_P},
  \ID{NV\_LOCLENGTH\_P}, \ID{NV\_GLOBLENGTH\_P}

  These macros give individual access to the parts of
  the content of a parallel \id{N\_Vector}.

  The assignment \id{v\_data = NV\_DATA\_P(v)} sets \id{v\_data} to be
  a pointer to the first component of the local data for the \id{N\_Vector} \id{v}.
  The assignment \id{NV\_DATA\_P(v) = v\_data} sets the component array of
  \id{v} to be \id{v\_data} by storing the pointer \id{v\_data}.

  The assignment \id{v\_llen = NV\_LOCLENGTH\_P(v)} sets \id{v\_llen} to be
  the length of the local part of \id{v}.
  The call \id{NV\_LENGTH\_P(v) = llen\_v} sets
  the local length of \id{v} to be \id{llen\_v}.

  The assignment \id{v\_glen = NV\_GLOBLENGTH\_P(v)} sets \id{v\_glen} to
  be the global length of the vector \id{v}.
  The call \id{NV\_GLOBLENGTH\_P(v) = glen\_v} sets the global
  length of \id{v} to be \id{glen\_v}.

  Implementation:

  \verb|#define NV_OWN_DATA_P(v)   ( NV_CONTENT_P(v)->own_data )|

  \verb|#define NV_DATA_P(v)       ( NV_CONTENT_P(v)->data )|

  \verb|#define NV_LOCLENGTH_P(v)  ( NV_CONTENT_P(v)->local_length )|

  \verb|#define NV_GLOBLENGTH_P(v) ( NV_CONTENT_P(v)->global_length )|

\item \ID{NV\_COMM\_P}

  This macro provides access to the {\mpi} communicator used by the {\nvecp}
  vectors.

  Implementation:

  \verb|#define NV_COMM_P(v) ( NV_CONTENT_P(v)->comm )|

\item \ID{NV\_Ith\_P}

  This macro gives access to the individual components of the local data
  array of an \id{N\_Vector}.

  The assignment \id{r = NV\_Ith\_P(v,i)} sets \id{r} to be the value of
  the \id{i}-th component of the local part of \id{v}.
  The assignment \id{NV\_Ith\_P(v,i) = r}
  sets the value of the \id{i}-th component of the local part of \id{v}
  to be \id{r}.

  Here $i$ ranges from $0$ to $n-1$, where $n$ is the local length.

  Implementation:

  \verb|#define NV_Ith_P(v,i) ( NV_DATA_P(v)[i] )|

\end{itemize}


% ====================================================================
\subsection{NVECTOR\_PARALLEL functions}
\label{ss:nvec_par_functions}
% ====================================================================

The {\nvecp} module defines parallel implementations of all vector operations listed
in Tables \ref{ss:nvecops}, \ref{ss:nvecfusedops}, \ref{ss:nvecarrayops},
and \ref{ss:nveclocalops}. Their names
are obtained from those in these tables by appending the suffix \id{\_Parallel}
(e.g. \id{N\_VDestroy\_Parallel}).
The module {\nvecp} provides the following additional
user-callable routines:
%%--------------------------------------
\sunmodfunf{N\_VNew\_Parallel}
{
  This function creates and allocates memory for a parallel vector.
}
{
  N\_Vector N\_VNew\_Parallel(MPI\_Comm comm, sunindextype
  local\_length,
  \newlinefill{N\_Vector N\_VNew\_Parallel}
  sunindextype global\_length);
}
%%--------------------------------------
\sunmodfunf{N\_VNewEmpty\_Parallel}
{
  This function creates a new parallel \id{N\_Vector} with an empty
  (\id{NULL}) data array.
}
{
  N\_Vector N\_VNewEmpty\_Parallel(MPI\_Comm comm, sunindextype
  local\_length,
  \newlinefill{N\_Vector N\_VNewEmpty\_Parallel}
  sunindextype global\_length);
}
%%--------------------------------------
\sunmodfunf{N\_VMake\_Parallel}
{
  This function creates and allocates memory for a parallel vector
  with user-provided data array. This function does {\em not} allocate memory
  for \id{v\_data} itself.
}
{
  N\_Vector N\_VMake\_Parallel(MPI\_Comm comm,
  sunindextype local\_length,
  \newlinefill{N\_Vector N\_VMake\_Parallel}
  sunindextype global\_length,
  realtype *v\_data);
}
%%--------------------------------------
\sunmodfunf{N\_VCloneVectorArray\_Parallel}
{
  This function creates (by cloning) an array of \id{count} parallel vectors.
}
{
  N\_Vector *N\_VCloneVectorArray\_Parallel(int count, N\_Vector w);
}
%%--------------------------------------
\sunmodfunf{N\_VCloneVectorArrayEmpty\_Parallel}
{
  This function creates (by cloning) an array of \id{count} parallel vectors,
  each with an empty (\id{NULL}) data array.
}
{
  N\_Vector *N\_VCloneVectorArrayEmpty\_Parallel(int count, N\_Vector w);
}
%%--------------------------------------
\sunmodfunf{N\_VDestroyVectorArray\_Parallel}
{
  This function frees memory allocated for the array of \id{count}  variables of
  type \id{N\_Vector} created with \id{N\_VCloneVectorArray\_Parallel} or with \\
  \id{N\_VCloneVectorArrayEmpty\_Parallel}.
}
{
  void N\_VDestroyVectorArray\_Parallel(N\_Vector *vs, int count);
}
%%--------------------------------------
\sunmodfunf{N\_VGetLocalLength\_Parallel}
{
  This function returns the local vector length.
}
{
  sunindextype N\_VGetLocalLength\_Parallel(N\_Vector v);
}
%%--------------------------------------
\sunmodfunf{N\_VPrint\_Parallel}
{
  This function prints the local content of a parallel vector to \id{stdout}.
}
{
  void N\_VPrint\_Parallel(N\_Vector v);
}
%%--------------------------------------
\sunmodfunf{N\_VPrintFile\_Parallel}
{
  This function prints the local content of a parallel vector to \id{outfile}.
}
{
  void N\_VPrintFile\_Parallel(N\_Vector v, FILE *outfile);
}
%%--------------------------------------
By default all fused and vector array operations are disabled in the {\nvecp}
module. The following additional user-callable routines are provided to
enable or disable fused and vector array operations for a specific vector. To
ensure consistency across vectors it is recommended to first create a vector
with \id{N\_VNew\_Parallel}, enable/disable the desired operations for that vector
with the functions below, and create any additional vectors from that vector
using \id{N\_VClone} with that vector. This guarantees the new vectors will
have the same operations enabled/disabled as cloned vectors inherit the same
enable/disable options as the vector they are cloned from while vectors created with
\id{N\_VNew\_Parallel} will have the default settings for the {\nvecp} module.
%%--------------------------------------
\sunmodfunf{N\_VEnableFusedOps\_Parallel}
{
  This function enables (\id{SUNTRUE}) or disables (\id{SUNFALSE}) all fused and
  vector array operations in the parallel vector. The return value is \id{0} for
  success and \id{-1} if the input vector or its \id{ops} structure are \id{NULL}.
}
{
  int N\_VEnableFusedOps\_Parallel(N\_Vector v, booleantype tf);
}
%%--------------------------------------
\sunmodfunf{N\_VEnableLinearCombination\_Parallel}
{
  This function enables (\id{SUNTRUE}) or disables (\id{SUNFALSE}) the linear
  combination fused operation in the parallel vector. The return value is \id{0} for
  success and \id{-1} if the input vector or its \id{ops} structure are \id{NULL}.
}
{
  int N\_VEnableLinearCombination\_Parallel(N\_Vector v, booleantype tf);
}
%%--------------------------------------
\sunmodfunf{N\_VEnableScaleAddMulti\_Parallel}
{
  This function enables (\id{SUNTRUE}) or disables (\id{SUNFALSE}) the scale and
  add a vector to multiple vectors fused operation in the parallel vector. The
  return value is \id{0} for success and \id{-1} if the input vector or its
  \id{ops} structure are \id{NULL}.
}
{
  int N\_VEnableScaleAddMulti\_Parallel(N\_Vector v, booleantype tf);
}
%%--------------------------------------
\sunmodfunf{N\_VEnableDotProdMulti\_Parallel}
{
  This function enables (\id{SUNTRUE}) or disables (\id{SUNFALSE}) the multiple
  dot products fused operation in the parallel vector. The return value is \id{0}
  for success and \id{-1} if the input vector or its \id{ops} structure are
  \id{NULL}.
}
{
  int N\_VEnableDotProdMulti\_Parallel(N\_Vector v, booleantype tf);
}
%%--------------------------------------
\sunmodfunf{N\_VEnableLinearSumVectorArray\_Parallel}
{
  This function enables (\id{SUNTRUE}) or disables (\id{SUNFALSE}) the linear sum
  operation for vector arrays in the parallel vector. The return value is \id{0} for
  success and \id{-1} if the input vector or its \id{ops} structure are \id{NULL}.
}
{
  int N\_VEnableLinearSumVectorArray\_Parallel(N\_Vector v, booleantype tf);
}
%%--------------------------------------
\sunmodfunf{N\_VEnableScaleVectorArray\_Parallel}
{
  This function enables (\id{SUNTRUE}) or disables (\id{SUNFALSE}) the scale
  operation for vector arrays in the parallel vector. The return value is \id{0} for
  success and \id{-1} if the input vector or its \id{ops} structure are \id{NULL}.
}
{
  int N\_VEnableScaleVectorArray\_Parallel(N\_Vector v, booleantype tf);
}
%%--------------------------------------
\sunmodfunf{N\_VEnableConstVectorArray\_Parallel}
{
  This function enables (\id{SUNTRUE}) or disables (\id{SUNFALSE}) the const
  operation for vector arrays in the parallel vector. The return value is \id{0} for
  success and \id{-1} if the input vector or its \id{ops} structure are \id{NULL}.
}
{
  int N\_VEnableConstVectorArray\_Parallel(N\_Vector v, booleantype tf);
}
%%--------------------------------------
\sunmodfunf{N\_VEnableWrmsNormVectorArray\_Parallel}
{
  This function enables (\id{SUNTRUE}) or disables (\id{SUNFALSE}) the WRMS norm
  operation for vector arrays in the parallel vector. The return value is \id{0} for
  success and \id{-1} if the input vector or its \id{ops} structure are \id{NULL}.
}
{
  int N\_VEnableWrmsNormVectorArray\_Parallel(N\_Vector v, booleantype tf);
}
%%--------------------------------------
\sunmodfunf{N\_VEnableWrmsNormMaskVectorArray\_Parallel}
{
  This function enables (\id{SUNTRUE}) or disables (\id{SUNFALSE}) the masked WRMS
  norm operation for vector arrays in the parallel vector. The return value is
  \id{0} for success and \id{-1} if the input vector or its \id{ops} structure are
  \id{NULL}.
}
{
  int N\_VEnableWrmsNormMaskVectorArray\_Parallel(N\_Vector v, booleantype tf);
}
%%--------------------------------------
\sunmodfun{N\_VEnableScaleAddMultiVectorArray\_Parallel}
{
  This function enables (\id{SUNTRUE}) or disables (\id{SUNFALSE}) the scale and
  add a vector array to multiple vector arrays operation in the parallel vector. The
  return value is \id{0} for success and \id{-1} if the input vector or its
  \id{ops} structure are \id{NULL}.
}
{
  int N\_VEnableScaleAddMultiVectorArray\_Parallel(N\_Vector v,
  \newlinefill{int N\_VEnableScaleAddMultiVectorArray\_Parallel}
  booleantype tf);
}
%%--------------------------------------
\sunmodfun{N\_VEnableLinearCombinationVectorArray\_Parallel}
{
  This function enables (\id{SUNTRUE}) or disables (\id{SUNFALSE}) the linear
  combination operation for vector arrays in the parallel vector. The return value
  is \id{0} for success and \id{-1} if the input vector or its \id{ops} structure
  are \id{NULL}.
}
{
  int N\_VEnableLinearCombinationVectorArray\_Parallel(N\_Vector v,
  \newlinefill{int N\_VEnableLinearCombinationVectorArray\_Parallel}
  booleantype tf);
}
%%
%%------------------------------------
%%
\paragraph{\bf Notes}

\begin{itemize}

\item
  When looping over the components of an \id{N\_Vector} \id{v}, it is
  more efficient to first obtain the local component array via
  \id{v\_data = NV\_DATA\_P(v)} and then access \id{v\_data[i]} within the
  loop than it is to use \id{NV\_Ith\_P(v,i)} within the loop.

\item
  {\warn}\id{N\_VNewEmpty\_Parallel}, \id{N\_VMake\_Parallel},
  and \id{N\_VCloneVectorArrayEmpty\_Parallel} set the field
  {\em own\_data} $=$ \id{SUNFALSE}.
  \id{N\_VDestroy\_Parallel} and \id{N\_VDestroyVectorArray\_Parallel}
  will not attempt to free the pointer {\em data} for any \id{N\_Vector} with
  {\em own\_data} set to \id{SUNFALSE}. In such a case, it is the user's responsibility to
  deallocate the {\em data} pointer.

\item
  {\warn}To maximize efficiency, vector operations in the {\nvecp} implementation
  that have more than one \id{N\_Vector} argument do not check for
  consistent internal representation of these vectors. It is the user's
  responsibility to ensure that such routines are called with \id{N\_Vector}
  arguments that were all created with the same internal representations.

\end{itemize}


% ====================================================================
\subsection{NVECTOR\_PARALLEL Fortran interfaces}
\label{ss:nvec_par_fortran}
% ====================================================================

For solvers that include a {\F} 77 interface module, the {\nvecp} module
also includes a {\F}-callable function
\id{FNVINITP(COMM, code, NLOCAL, NGLOBAL, IER)},
to initialize this {\nvecp} module.  Here \id{COMM} is the MPI communicator,
\id{code} is an input solver id (1 for {\cvode}, 2 for {\ida}, 3 for {\kinsol},
4 for {\arkode}); \id{NLOCAL} and \id{NGLOBAL} are the local and global
vector sizes, respectively (declared so as to match C type \id{long int});
and IER is an error return flag equal 0 for success and -1 for failure.
{\warn}NOTE: If the header file \id{sundials\_config.h} defines
\id{SUNDIALS\_MPI\_COMM\_F2C} to be $1$ (meaning the {\mpi}
implementation used to build {\sundials} includes the
\id{MPI\_Comm\_f2c} function), then \id{COMM} can be any valid
{\mpi} communicator. Otherwise, \id{MPI\_COMM\_WORLD} will be used, so
just pass an integer value as a placeholder.

%% This is a shared SUNDIALS TEX file with a description of the
%% OpenMP nvector implementation
%%
\section{The NVECTOR\_OPENMP implementation}\label{ss:nvec_openmp}

In situations where a user has a multi-core processing unit capable of
running multiple parallel threads with shared memory, {\sundials} provides
an implementation of {\nvector} using OpenMP, called {\nvecopenmp}, and
an implementation using Pthreads, called {\nvecpthreads}.
Testing has shown that vectors should be of length at least $100,000$
before the overhead associated with creating and using the threads is
made up by the parallelism in the vector calculations.

The OpenMP {\nvector} implementation provided with {\sundials},
{\nvecopenmp}, defines the {\em content} field of \id{N\_Vector} to be a structure
containing the length of the vector, a pointer to the beginning of a contiguous
data array, a boolean flag {\em own\_data} which specifies the ownership
of {\em data}, and the number of threads.
Operations on the vector are threaded using OpenMP.
%%
\begin{verbatim}
struct _N_VectorContent_OpenMP {
  sunindextype length;
  booleantype own_data;
  realtype *data;
  int num_threads;
};
\end{verbatim}
%%
%%--------------------------------------------
%%

The header file to include when using this module is \id{nvector\_openmp.h}.
The installed module library to link to is
\id{libsundials\_nvecopenmp.\textit{lib}}
where \id{\em.lib} is typically \id{.so} for shared libraries and \id{.a}
for static libraries.
The {\F} module file to use when using the {\F} 2003 interface to
this module is \id{fnvector\_openmp\_mod.mod}.


% ====================================================================
\subsection{NVECTOR\_OPENMP accessor macros}
\label{ss:nvec_openmp_macros}
% ====================================================================

The following macros are provided to access the content of an {\nvecopenmp}
vector. The suffix \id{\_OMP} in the names denotes the OpenMP version.
%%
\begin{itemize}

\item \ID{NV\_CONTENT\_OMP}

  This routine gives access to the contents of the OpenMP
  vector \id{N\_Vector}.

  The assignment \id{v\_cont} $=$ \id{NV\_CONTENT\_OMP(v)} sets
  \id{v\_cont} to be a pointer to the OpenMP \id{N\_Vector} content
  structure.

  Implementation:

  \verb|#define NV_CONTENT_OMP(v) ( (N_VectorContent_OpenMP)(v->content) )|

\item \ID{NV\_OWN\_DATA\_OMP}, \ID{NV\_DATA\_OMP}, \ID{NV\_LENGTH\_OMP}, \ID{NV\_NUM\_THREADS\_OMP}


  These macros give individual access to the parts of
  the content of a OpenMP \id{N\_Vector}.

  The assignment \id{v\_data = NV\_DATA\_OMP(v)} sets \id{v\_data} to be
  a pointer to the first component of the data for the \id{N\_Vector} \id{v}.
  The assignment \id{NV\_DATA\_OMP(v) = v\_data} sets the component array of \id{v} to
  be \id{v\_data} by storing the pointer \id{v\_data}.

  The assignment \id{v\_len = NV\_LENGTH\_OMP(v)} sets \id{v\_len} to be
  the length of \id{v}. On the other hand, the call \id{NV\_LENGTH\_OMP(v) = len\_v}
  sets the length of \id{v} to be \id{len\_v}.

  The assignment \id{v\_num\_threads = NV\_NUM\_THREADS\_OMP(v)} sets \id{v\_num\_threads} to be
  the number of threads from \id{v}. On the other hand, the call \id{NV\_NUM\_THREADS\_OMP(v) = num\_threads\_v}
  sets the number of threads for \id{v} to be \id{num\_threads\_v}.

  Implementation:

  \verb|#define NV_OWN_DATA_OMP(v) ( NV_CONTENT_OMP(v)->own_data )|

  \verb|#define NV_DATA_OMP(v) ( NV_CONTENT_OMP(v)->data )|

  \verb|#define NV_LENGTH_OMP(v) ( NV_CONTENT_OMP(v)->length )|

  \verb|#define NV_NUM_THREADS_OMP(v) ( NV_CONTENT_OMP(v)->num_threads )|

\item \ID{NV\_Ith\_OMP}

  This macro gives access to the individual components of the data
  array of an \id{N\_Vector}.

  The assignment \id{r = NV\_Ith\_OMP(v,i)} sets \id{r} to be the value of
  the \id{i}-th component of \id{v}. The assignment \id{NV\_Ith\_OMP(v,i) = r}
  sets the value of the \id{i}-th component of \id{v} to be \id{r}.

  Here $i$ ranges from $0$ to $n-1$ for a vector of length $n$.

  Implementation:

  \verb|#define NV_Ith_OMP(v,i) ( NV_DATA_OMP(v)[i] )|

\end{itemize}


% ====================================================================
\subsection{NVECTOR\_OPENMP functions}
\label{ss:nvec_openmp_functions}
% ====================================================================

The {\nvecopenmp} module defines OpenMP implementations of all vector operations listed
in Tables \ref{t:nvecops}, \ref{t:nvecfusedops}, \ref{t:nvecarrayops},
and \ref{t:nveclocalops}. Their names are obtained from those in these
tables by appending the suffix \id{\_OpenMP} (e.g. \id{N\_VDestroy\_OpenMP}).
All the standard vector operations listed in \ref{t:nvecops} with the suffix
\id{\_OpenMP} appended are callable via the {\F} 2003 interface by prepending an
`F' (e.g. \id{FN\_VDestroy\_OpenMP}).

The module {\nvecopenmp} provides the following additional user-callable routines:
%%--------------------------------------
\sunmodfunf{N\_VNew\_OpenMP}
{
 This function creates and allocates memory for a OpenMP \id{N\_Vector}.
 Arguments are the vector length and number of threads.
}
{
 N\_Vector N\_VNew\_OpenMP(sunindextype vec\_length, int num\_threads)
}

%%--------------------------------------

\sunmodfunf{N\_VNewEmpty\_OpenMP}
{
  This function creates a new OpenMP \id{N\_Vector} with an empty (\id{NULL}) data array.
}
{
  N\_Vector N\_VNewEmpty\_OpenMP(sunindextype vec\_length, int num\_threads)
}
%%--------------------------------------
\sunmodfunf{N\_VMake\_OpenMP}
{
 This function creates and allocates memory for a OpenMP vector
 with user-provided data array. This function does {\em not} allocate memory for
 \id{v\_data} itself.
}
{
  N\_Vector N\_VMake\_OpenMP(sunindextype vec\_length, realtype *v\_data,
  \newlinefill{N\_Vector N\_VMake\_OpenMP}
  int num\_threads);
}
%%--------------------------------------
\sunmodfun{N\_VCloneVectorArray\_OpenMP}
{
  This function creates (by cloning) an array of \id{count} OpenMP vectors.
}
{
  N\_Vector *N\_VCloneVectorArray\_OpenMP(int count, N\_Vector w)
}
%%--------------------------------------
\sunmodfun{N\_VCloneVectorArrayEmpty\_OpenMP}
{
  This function creates (by cloning) an array of \id{count} OpenMP vectors, each with an
  empty (\id{NULL}) data array.
}
{
  N\_Vector *N\_VCloneVectorArrayEmpty\_OpenMP(int count, N\_Vector w)
}
%%--------------------------------------
\sunmodfun{N\_VDestroyVectorArray\_OpenMP}
{
  This function frees memory allocated for the array of \id{count} variables of type
  \id{N\_Vector} created with \id{N\_VCloneVectorArray\_OpenMP} or with
  \id{N\_VCloneVectorArrayEmpty\_OpenMP}.
}
{
 void N\_VDestroyVectorArray\_OpenMP(N\_Vector *vs, int count)
}
%%--------------------------------------
\sunmodfunf{N\_VPrint\_OpenMP}
{
  This function prints the content of an OpenMP vector to \id{stdout}.
}
{
  void N\_VPrint\_OpenMP(N\_Vector v)
}
%%--------------------------------------
\sunmodfun{N\_VPrintFile\_OpenMP}
{
  This function prints the content of an OpenMP vector to \id{outfile}.
}
{
  void N\_VPrintFile\_OpenMP(N\_Vector v, FILE *outfile)
}
%%--------------------------------------

By default all fused and vector array operations are disabled in the {\nvecopenmp}
module. The following additional user-callable routines are provided to
enable or disable fused and vector array operations for a specific vector. To
ensure consistency across vectors it is recommended to first create a vector
with \id{N\_VNew\_OpenMP}, enable/disable the desired operations for that vector
with the functions below, and create any additional vectors from that vector
using \id{N\_VClone}. This guarantees the new vectors will have the same
operations enabled/disabled as cloned vectors inherit the same enable/disable
options as the vector they are cloned from while vectors created with
\id{N\_VNew\_OpenMP} will have the default settings for the {\nvecopenmp} module.
%%--------------------------------------
\sunmodfun{N\_VEnableFusedOps\_OpenMP}
{
  This function enables (\id{SUNTRUE}) or disables (\id{SUNFALSE}) all fused and
  vector array operations in the OpenMP vector. The return value is \id{0} for
  success and \id{-1} if the input vector or its \id{ops} structure are \id{NULL}.
}
{
  int N\_VEnableFusedOps\_OpenMP(N\_Vector v, booleantype tf)
}
%%--------------------------------------
\sunmodfun{N\_VEnableLinearCombination\_OpenMP}
{
  This function enables (\id{SUNTRUE}) or disables (\id{SUNFALSE}) the linear
  combination fused operation in the OpenMP vector. The return value is \id{0} for
  success and \id{-1} if the input vector or its \id{ops} structure are \id{NULL}.
}
{
  int N\_VEnableLinearCombination\_OpenMP(N\_Vector v, booleantype tf)
}
%%--------------------------------------
\sunmodfun{N\_VEnableScaleAddMulti\_OpenMP}
{
  This function enables (\id{SUNTRUE}) or disables (\id{SUNFALSE}) the scale and
  add a vector to multiple vectors fused operation in the OpenMP vector. The
  return value is \id{0} for success and \id{-1} if the input vector or its
  \id{ops} structure are \id{NULL}.
}
{
  int N\_VEnableScaleAddMulti\_OpenMP(N\_Vector v, booleantype tf)
}
%%--------------------------------------
\sunmodfun{N\_VEnableDotProdMulti\_OpenMP}
{
  This function enables (\id{SUNTRUE}) or disables (\id{SUNFALSE}) the multiple
  dot products fused operation in the OpenMP vector. The return value is \id{0}
  for success and \id{-1} if the input vector or its \id{ops} structure are
  \id{NULL}.
}
{
  int N\_VEnableDotProdMulti\_OpenMP(N\_Vector v, booleantype tf)
}
%%--------------------------------------
\sunmodfun{N\_VEnableLinearSumVectorArray\_OpenMP}
{
  This function enables (\id{SUNTRUE}) or disables (\id{SUNFALSE}) the linear sum
  operation for vector arrays in the OpenMP vector. The return value is \id{0} for
  success and \id{-1} if the input vector or its \id{ops} structure are \id{NULL}.
}
{
  int N\_VEnableLinearSumVectorArray\_OpenMP(N\_Vector v, booleantype tf)
}
%%--------------------------------------
\sunmodfun{N\_VEnableScaleVectorArray\_OpenMP}
{
  This function enables (\id{SUNTRUE}) or disables (\id{SUNFALSE}) the scale
  operation for vector arrays in the OpenMP vector. The return value is \id{0} for
  success and \id{-1} if the input vector or its \id{ops} structure are \id{NULL}.
}
{
  int N\_VEnableScaleVectorArray\_OpenMP(N\_Vector v, booleantype tf)
}
%%--------------------------------------
\sunmodfun{N\_VEnableConstVectorArray\_OpenMP}
{
  This function enables (\id{SUNTRUE}) or disables (\id{SUNFALSE}) the const
  operation for vector arrays in the OpenMP vector. The return value is \id{0} for
  success and \id{-1} if the input vector or its \id{ops} structure are \id{NULL}.
}
{
  int N\_VEnableConstVectorArray\_OpenMP(N\_Vector v, booleantype tf)
}
%%--------------------------------------
\sunmodfun{N\_VEnableWrmsNormVectorArray\_OpenMP}
{
  This function enables (\id{SUNTRUE}) or disables (\id{SUNFALSE}) the WRMS norm
  operation for vector arrays in the OpenMP vector. The return value is \id{0} for
  success and \id{-1} if the input vector or its \id{ops} structure are \id{NULL}.
}
{
  int N\_VEnableWrmsNormVectorArray\_OpenMP(N\_Vector v, booleantype tf)
}
%%--------------------------------------
\sunmodfun{N\_VEnableWrmsNormMaskVectorArray\_OpenMP}
{
  This function enables (\id{SUNTRUE}) or disables (\id{SUNFALSE}) the masked WRMS
  norm operation for vector arrays in the OpenMP vector. The return value is
  \id{0} for success and \id{-1} if the input vector or its \id{ops} structure are
  \id{NULL}.
}
{
  int N\_VEnableWrmsNormMaskVectorArray\_OpenMP(N\_Vector v, booleantype tf)
}
%%--------------------------------------
\sunmodfun{N\_VEnableScaleAddMultiVectorArray\_OpenMP}
{
  This function enables (\id{SUNTRUE}) or disables (\id{SUNFALSE}) the scale and
  add a vector array to multiple vector arrays operation in the OpenMP vector. The
  return value is \id{0} for success and \id{-1} if the input vector or its
  \id{ops} structure are \id{NULL}.
}
{
  int N\_VEnableScaleAddMultiVectorArray\_OpenMP(N\_Vector v, booleantype tf)
}
%%--------------------------------------
\sunmodfun{N\_VEnableLinearCombinationVectorArray\_OpenMP}
{
  This function enables (\id{SUNTRUE}) or disables (\id{SUNFALSE}) the linear
  combination operation for vector arrays in the OpenMP vector. The return value
  is \id{0} for success and \id{-1} if the input vector or its \id{ops} structure
  are \id{NULL}.
}
{
  int N\_VEnableLinearCombinationVectorArray\_OpenMP(N\_Vector v,
  \newlinefill{int N\_VEnableLinearCombinationVectorArray\_OpenMP}
  booleantype tf)
}
%%
%%------------------------------------
%%
\paragraph{\bf Notes}

\begin{itemize}

\item
  When looping over the components of an \id{N\_Vector} \id{v}, it is
  more efficient to first obtain the component array via
  \id{v\_data = NV\_DATA\_OMP(v)} and then access \id{v\_data[i]} within the
  loop than it is to use \id{NV\_Ith\_OMP(v,i)} within the loop.

\item
  {\warn}\id{N\_VNewEmpty\_OpenMP}, \id{N\_VMake\_OpenMP},
  and \id{N\_VCloneVectorArrayEmpty\_OpenMP} set the field
  {\em own\_data} $=$ \id{SUNFALSE}.
  \id{N\_VDestroy\_OpenMP} and \id{N\_VDestroyVectorArray\_OpenMP}
  will not attempt to free the pointer {\em data} for any \id{N\_Vector} with
  {\em own\_data} set to \id{SUNFALSE}. In such a case, it is the user's responsibility to
  deallocate the {\em data} pointer.

\item
  {\warn}To maximize efficiency, vector operations in the {\nvecopenmp} implementation
  that have more than one \id{N\_Vector} argument do not check for
  consistent internal representation of these vectors. It is the user's
  responsibility to ensure that such routines are called with \id{N\_Vector}
  arguments that were all created with the same internal representations.

\end{itemize}


% ====================================================================
\subsection{NVECTOR\_OPENMP Fortran interfaces}
\label{ss:nvec_openmp_fortran}
% ====================================================================

The {\nvecopenmp} module provides a {\F} 2003 module as well as {\F} 77
style interface functions for use from {\F} applications.

\subsubsection*{FORTRAN 2003 interface module}
The \ID{nvector\_openmp\_mod} {\F} module defines interfaces to most
{\nvecopenmp} {\CC} functions using the intrinsic \id{iso\_c\_binding}
module which provides a standardized mechanism for interoperating with {\CC}. As
noted in the {\CC} function descriptions above, the interface functions are
named after the corresponding {\CC} function, but with a leading `F'. For
example, the function \id{N\_VNew\_OpenMP} is interfaced as
\id{FN\_VNew\_OpenMP}.

The {\F} 2003 {\nvecopenmp} interface module can be accessed with the \id{use}
statement, i.e. \id{use fnvector\_openmp\_mod}, and linking to the library
\id{libsundials\_fnvectoropenmp\_mod}.{\em lib} in addition to the {\CC} library.
For details on where the library and module file
\id{fnvector\_openmp\_mod.mod} are installed see Appendix \ref{c:install}.

\subsubsection*{FORTRAN 77 interface functions}
For solvers that include a {\F} 77 interface module, the {\nvecopenmp}
module also includes a {\F}-callable function
\id{FNVINITOMP(code, NEQ, NUMTHREADS, IER)}, to initialize this
module.  Here \id{code} is an input solver id
(1 for {\cvode}, 2 for {\ida}, 3 for {\kinsol}, 4 for {\arkode}); NEQ is
the problem size (declared so as to match C type \id{long int});
NUMTHREADS is the number of threads; and IER is an error return flag
equal 0 for success and -1 for failure.

%% This is a shared SUNDIALS TEX file with a description of the
%% Pthreads nvector implementation
%%
\section{The NVECTOR\_PTHREADS implementation}\label{ss:nvec_pthreads}

In situations where a user has a multi-core processing unit capable of
running multiple parallel threads with shared memory, {\sundials} provides
an implementation of {\nvector} using OpenMP, called {\nvecopenmp}, and
an implementation using Pthreads, called {\nvecpthreads}.
Testing has shown that vectors should be of length at least $100,000$
before the overhead associated with creating and using the threads is
made up by the parallelism in the vector calculations.

The Pthreads {\nvector} implementation provided with {\sundials}, denoted
{\nvecpthreads}, defines the {\em content} field of \id{N\_Vector} to be a structure
containing the length of the vector, a pointer to the beginning of a contiguous
data array, a boolean flag {\em own\_data} which specifies the ownership
of {\em data}, and the number of threads.
Operations on the vector are threaded using POSIX threads
(Pthreads).
%%
\begin{verbatim}
struct _N_VectorContent_Pthreads {
  sunindextype length;
  booleantype own_data;
  realtype *data;
  int num_threads;
};
\end{verbatim}
%%
%%--------------------------------------------
%%

The header file to include when using this module is \id{nvector\_pthreads.h}.
The installed module library to link to is
\id{libsundials\_nvecpthreads.\textit{lib}}
where \id{\em.lib} is typically \id{.so} for shared libraries and \id{.a}
for static libraries.

% ====================================================================
\subsection{NVECTOR\_PTHREADS accessor macros}
\label{ss:nvec_pthreads_macros}
% ====================================================================

The following macros are provided to access the content of an {\nvecpthreads}
vector. The suffix \id{\_PT} in the names denotes the Pthreads version.
%%
\begin{itemize}

\item \ID{NV\_CONTENT\_PT}

  This routine gives access to the contents of the Pthreads
  vector \id{N\_Vector}.

  The assignment \id{v\_cont} $=$ \id{NV\_CONTENT\_PT(v)} sets
  \id{v\_cont} to be a pointer to the Pthreads \id{N\_Vector} content
  structure.

  Implementation:

  \verb|#define NV_CONTENT_PT(v) ( (N_VectorContent_Pthreads)(v->content) )|

\item \ID{NV\_OWN\_DATA\_PT}, \ID{NV\_DATA\_PT}, \ID{NV\_LENGTH\_PT}, \ID{NV\_NUM\_THREADS\_PT}


  These macros give individual access to the parts of
  the content of a Pthreads \id{N\_Vector}.

  The assignment \id{v\_data = NV\_DATA\_PT(v)} sets \id{v\_data} to be
  a pointer to the first component of the data for the \id{N\_Vector} \id{v}.
  The assignment \id{NV\_DATA\_PT(v) = v\_data} sets the component array of \id{v} to
  be \id{v\_data} by storing the pointer \id{v\_data}.

  The assignment \id{v\_len = NV\_LENGTH\_PT(v)} sets \id{v\_len} to be
  the length of \id{v}. On the other hand, the call \id{NV\_LENGTH\_PT(v) = len\_v}
  sets the length of \id{v} to be \id{len\_v}.

  The assignment \id{v\_num\_threads = NV\_NUM\_THREADS\_PT(v)} sets \id{v\_num\_threads} to be
  the number of threads from \id{v}. On the other hand, the call \id{NV\_NUM\_THREADS\_PT(v) = num\_threads\_v}
  sets the number of threads for \id{v} to be \id{num\_threads\_v}.

  Implementation:

  \verb|#define NV_OWN_DATA_PT(v) ( NV_CONTENT_PT(v)->own_data )|

  \verb|#define NV_DATA_PT(v) ( NV_CONTENT_PT(v)->data )|

  \verb|#define NV_LENGTH_PT(v) ( NV_CONTENT_PT(v)->length )|

  \verb|#define NV_NUM_THREADS_PT(v) ( NV_CONTENT_PT(v)->num_threads )|

\item \ID{NV\_Ith\_PT}

  This macro gives access to the individual components of the data
  array of an \id{N\_Vector}.

  The assignment \id{r = NV\_Ith\_PT(v,i)} sets \id{r} to be the value of
  the \id{i}-th component of \id{v}. The assignment \id{NV\_Ith\_PT(v,i) = r}
  sets the value of the \id{i}-th component of \id{v} to be \id{r}.

  Here $i$ ranges from $0$ to $n-1$ for a vector of length $n$.

  Implementation:

  \verb|#define NV_Ith_PT(v,i) ( NV_DATA_PT(v)[i] )|

\end{itemize}

% ====================================================================
\subsection{NVECTOR\_PTHREADS functions}
\label{ss:nvec_pthreads_functions}
% ====================================================================

The {\nvecpthreads} module defines Pthreads implementations of all vector operations listed
in Tables \ref{t:nvecops}, \ref{t:nvecfusedops}, \ref{t:nvecarrayops},
and \ref{t:nveclocalops}. Their names are
obtained from those in these tables by appending the suffix \id{\_Pthreads}
(e.g. \id{N\_VDestroy\_Pthreads}).
All the standard vector operations listed in \ref{t:nvecops} are callable via
the {\F} 2003 interface by prepending an `F' (e.g. \id{FN\_VDestroy\_Pthreads}).
The module {\nvecpthreads} provides the following additional user-callable routines:
%%--------------------------------------
\sunmodfunf{N\_VNew\_Pthreads}
{
  This function creates and allocates memory for a Pthreads \id{N\_Vector}.
  Arguments are the vector length and number of threads.
}
{
  N\_Vector N\_VNew\_Pthreads(sunindextype vec\_length, int num\_threads)
}
%%--------------------------------------
\sunmodfunf{N\_VNewEmpty\_Pthreads}
{
  This function creates a new Pthreads \id{N\_Vector} with an empty (\id{NULL}) data array.
}
{
  N\_Vector N\_VNewEmpty\_Pthreads(sunindextype vec\_length, int num\_threads)
}
%%--------------------------------------
\sunmodfunf{N\_VMake\_Pthreads}
{
  This function creates and allocates memory for a Pthreads vector
  with user-provided data array. This function does {\em not} allocate memory
  for \id{v\_data} itself.
}
{
  N\_Vector N\_VMake\_Pthreads(sunindextype vec\_length, realtype *v\_data,
  \newlinefill{N\_Vector N\_VMake\_Pthreads}
  int num\_threads);
}
%%--------------------------------------
\sunmodfun{N\_VCloneVectorArray\_Pthreads}
{
  This function creates (by cloning) an array of \id{count} Pthreads vectors.
}
{
  N\_Vector *N\_VCloneVectorArray\_Pthreads(int count, N\_Vector w)
}
%%--------------------------------------
\sunmodfun{N\_VCloneVectorArrayEmpty\_Pthreads}
{
  This function creates (by cloning) an array of \id{count} Pthreads vectors, each with an
  empty (\id{NULL}) data array.
}
{
  N\_Vector *N\_VCloneVectorArrayEmpty\_Pthreads(int count, N\_Vector w)
}
%%--------------------------------------
\sunmodfun{N\_VDestroyVectorArray\_Pthreads}
{
  This function frees memory allocated for the array of \id{count} variables of type
  \id{N\_Vector} created with \id{N\_VCloneVectorArray\_Pthreads} or with \newline
  \id{N\_VCloneVectorArrayEmpty\_Pthreads}.
}
{
  void N\_VDestroyVectorArray\_Pthreads(N\_Vector *vs, int count)
}
%%--------------------------------------
\sunmodfunf{N\_VPrint\_Pthreads}
{
  This function prints the content of a Pthreads vector to \id{stdout}.
}
{
  void N\_VPrint\_Pthreads(N\_Vector v)
}
%%--------------------------------------
\sunmodfun{N\_VPrintFile\_Pthreads}
{
  This function prints the content of a Pthreads vector to \id{outfile}.
}
{
  void N\_VPrintFile\_Pthreads(N\_Vector v, FILE *outfile)
}
%%--------------------------------------

By default all fused and vector array operations are disabled in the {\nvecpthreads}
module. The following additional user-callable routines are provided to
enable or disable fused and vector array operations for a specific vector. To
ensure consistency across vectors it is recommended to first create a vector
with \id{N\_VNew\_Pthreads}, enable/disable the desired operations for that vector
with the functions below, and create any additional vectors from that vector
using \id{N\_VClone}. This guarantees the new vectors will have the same
operations enabled/disabled as cloned vectors inherit the same enable/disable
options as the vector they are cloned from while vectors created with
\id{N\_VNew\_Pthreads} will have the default settings for the {\nvecpthreads} module.
%%--------------------------------------
\sunmodfun{N\_VEnableFusedOps\_Pthreads}
{
  This function enables (\id{SUNTRUE}) or disables (\id{SUNFALSE}) all fused and
  vector array operations in the Pthreads vector. The return value is \id{0} for
  success and \id{-1} if the input vector or its \id{ops} structure are \id{NULL}.
}
{
  int N\_VEnableFusedOps\_Pthreads(N\_Vector v, booleantype tf)
}
%%--------------------------------------
\sunmodfun{N\_VEnableLinearCombination\_Pthreads}
{
  This function enables (\id{SUNTRUE}) or disables (\id{SUNFALSE}) the linear
  combination fused operation in the Pthreads vector. The return value is \id{0} for
  success and \id{-1} if the input vector or its \id{ops} structure are \id{NULL}.
}
{
  int N\_VEnableLinearCombination\_Pthreads(N\_Vector v, booleantype tf)
}
%%--------------------------------------
\sunmodfun{N\_VEnableScaleAddMulti\_Pthreads}
{
  This function enables (\id{SUNTRUE}) or disables (\id{SUNFALSE}) the scale and
  add a vector to multiple vectors fused operation in the Pthreads vector. The
  return value is \id{0} for success and \id{-1} if the input vector or its
  \id{ops} structure are \id{NULL}.
}
{
  int N\_VEnableScaleAddMulti\_Pthreads(N\_Vector v, booleantype tf)
}
%%--------------------------------------
\sunmodfun{N\_VEnableDotProdMulti\_Pthreads}
{
  This function enables (\id{SUNTRUE}) or disables (\id{SUNFALSE}) the multiple
  dot products fused operation in the Pthreads vector. The return value is \id{0}
  for success and \id{-1} if the input vector or its \id{ops} structure are
  \id{NULL}.
}
{
  int N\_VEnableDotProdMulti\_Pthreads(N\_Vector v, booleantype tf)
}
%%--------------------------------------
\sunmodfun{N\_VEnableLinearSumVectorArray\_Pthreads}
{
  This function enables (\id{SUNTRUE}) or disables (\id{SUNFALSE}) the linear sum
  operation for vector arrays in the Pthreads vector. The return value is \id{0} for
  success and \id{-1} if the input vector or its \id{ops} structure are \id{NULL}.
}
{
  int N\_VEnableLinearSumVectorArray\_Pthreads(N\_Vector v, booleantype tf)
}
%%--------------------------------------
\sunmodfun{N\_VEnableScaleVectorArray\_Pthreads}
{
  This function enables (\id{SUNTRUE}) or disables (\id{SUNFALSE}) the scale
  operation for vector arrays in the Pthreads vector. The return value is \id{0} for
  success and \id{-1} if the input vector or its \id{ops} structure are \id{NULL}.
}
{
  int N\_VEnableScaleVectorArray\_Pthreads(N\_Vector v, booleantype tf)
}
%%--------------------------------------
\sunmodfun{N\_VEnableConstVectorArray\_Pthreads}
{
  This function enables (\id{SUNTRUE}) or disables (\id{SUNFALSE}) the const
  operation for vector arrays in the Pthreads vector. The return value is \id{0} for
  success and \id{-1} if the input vector or its \id{ops} structure are \id{NULL}.
}
{
  int N\_VEnableConstVectorArray\_Pthreads(N\_Vector v, booleantype tf)
}
%%--------------------------------------
\sunmodfun{N\_VEnableWrmsNormVectorArray\_Pthreads}
{
  This function enables (\id{SUNTRUE}) or disables (\id{SUNFALSE}) the WRMS norm
  operation for vector arrays in the Pthreads vector. The return value is \id{0} for
  success and \id{-1} if the input vector or its \id{ops} structure are \id{NULL}.
}
{
  int N\_VEnableWrmsNormVectorArray\_Pthreads(N\_Vector v, booleantype tf)
}
%%--------------------------------------
\sunmodfun{N\_VEnableWrmsNormMaskVectorArray\_Pthreads}
{
  This function enables (\id{SUNTRUE}) or disables (\id{SUNFALSE}) the masked WRMS
  norm operation for vector arrays in the Pthreads vector. The return value is
  \id{0} for success and \id{-1} if the input vector or its \id{ops} structure are
  \id{NULL}.
}
{
  int N\_VEnableWrmsNormMaskVectorArray\_Pthreads(N\_Vector v, booleantype tf)
}
%%--------------------------------------
\sunmodfun{N\_VEnableScaleAddMultiVectorArray\_Pthreads}
{
  This function enables (\id{SUNTRUE}) or disables (\id{SUNFALSE}) the scale and
  add a vector array to multiple vector arrays operation in the Pthreads vector. The
  return value is \id{0} for success and \id{-1} if the input vector or its
  \id{ops} structure are \id{NULL}.
}
{
  int N\_VEnableScaleAddMultiVectorArray\_Pthreads(N\_Vector v,
  \newlinefill{int N\_VEnableScaleAddMultiVectorArray\_Pthreads}
  booleantype tf)
}
%%--------------------------------------
\sunmodfun{N\_VEnableLinearCombinationVectorArray\_Pthreads}
{
  This function enables (\id{SUNTRUE}) or disables (\id{SUNFALSE}) the linear
  combination operation for vector arrays in the Pthreads vector. The return value
  is \id{0} for success and \id{-1} if the input vector or its \id{ops} structure
  are \id{NULL}.
}
{
  int N\_VEnableLinearCombinationVectorArray\_Pthreads(N\_Vector v,
  \newlinefill{int N\_VEnableLinearCombinationVectorArray\_Pthreads}
  booleantype tf)
}
%%
%%------------------------------------
%%
\paragraph{\bf Notes}

\begin{itemize}

\item
  When looping over the components of an \id{N\_Vector} \id{v}, it is
  more efficient to first obtain the component array via
  \id{v\_data = NV\_DATA\_PT(v)} and then access \id{v\_data[i]} within the
  loop than it is to use \id{NV\_Ith\_PT(v,i)} within the loop.

\item
  {\warn}\id{N\_VNewEmpty\_Pthreads}, \id{N\_VMake\_Pthreads},
  and \id{N\_VCloneVectorArrayEmpty\_Pthreads} set the field
  {\em own\_data} $=$ \id{SUNFALSE}.
  \id{N\_VDestroy\_Pthreads} and \id{N\_VDestroyVectorArray\_Pthreads}
  will not attempt to free the pointer {\em data} for any \id{N\_Vector} with
  {\em own\_data} set to \id{SUNFALSE}. In such a case, it is the user's responsibility to
  deallocate the {\em data} pointer.

\item
  {\warn}To maximize efficiency, vector operations in the {\nvecpthreads} implementation
  that have more than one \id{N\_Vector} argument do not check for
  consistent internal representation of these vectors. It is the user's
  responsibility to ensure that such routines are called with \id{N\_Vector}
  arguments that were all created with the same internal representations.

\end{itemize}


% ====================================================================
\subsection{NVECTOR\_PTHREADS Fortran interfaces}
\label{ss:nvec_pthreads_fortran}
% ====================================================================

The {\nvecpthreads} module provides a {\F} 2003 module as well as {\F} 77
style interface functions for use from {\F} applications.

\subsubsection*{FORTRAN 2003 interface module}
The \ID{nvector\_pthreads\_mod} {\F} module defines interfaces to most
{\nvecpthreads} {\CC} functions using the intrinsic \id{iso\_c\_binding}
module which provides a standardized mechanism for interoperating with {\CC}. As
noted in the {\CC} function descriptions above, the interface functions are
named after the corresponding {\CC} function, but with a leading `F'. For
example, the function \id{N\_VNew\_Pthreads} is interfaced as
\id{FN\_VNew\_Pthreads}.

The {\F} 2003 {\nvecpthreads} interface module can be accessed with the \id{use}
statement, i.e. \id{use fnvector\_pthreads\_mod}, and linking to the library
\id{libsundials\_fnvectorpthreads\_mod}.{\em lib} in addition to the {\CC} library.
For details on where the library and module file
\id{fnvector\_pthreads\_mod.mod} are installed see Appendix \ref{c:install}.

\subsubsection*{FORTRAN 77 interface functions}
For solvers that include a {\F} interface module, the {\nvecpthreads}
module also includes a {\F}-callable function
\id{FNVINITPTS(code, NEQ, NUMTHREADS, IER)}, to initialize this
module.  Here \id{code} is an input solver id
(1 for {\cvode}, 2 for {\ida}, 3 for {\kinsol}, 4 for {\arkode}); NEQ is
the problem size (declared so as to match C type \id{long int});
NUMTHREADS is the number of threads; and IER is an error return flag
equal 0 for success and -1 for failure.

% This is a shared SUNDIALS TEX file with description of
% the MPI parallel ParHyp hypre nvector implementation
%
\section{The NVECTOR\_PARHYP implementation}\label{ss:nvec_parhyp}

The {\nvecph} implementation of the {\nvector} module provided with
{\sundials} is a wrapper around {\hypre}'s ParVector class. 
Most of the vector kernels simply call {\hypre} vector operations. 
The implementation defines the {\em content} field of \id{N\_Vector} to 
be a structure containing the global and local lengths of the vector, a 
pointer to an object of type \id{HYPRE\_ParVector}, an {\mpi} communicator, 
and a boolean flag {\em own\_parvector} indicating ownership of the
{\hypre} parallel vector object {\em x}.
%%
%%
\begin{verbatim}
struct _N_VectorContent_ParHyp {
  sunindextype local_length;
  sunindextype global_length;
  booleantype own_parvector;
  MPI_Comm comm;
  HYPRE_ParVector x;
};
\end{verbatim}
%%
%%--------------------------------------------
The header file to include when using this module is \id{nvector\_parhyp.h}.
The installed module library to link to is
\id{libsundials\_nvecparhyp.\textit{lib}}
where \id{\em.lib} is typically \id{.so} for shared libraries and \id{.a}
for static libraries.

Unlike native {\sundials} vector types, {\nvecph} does not provide macros 
to access its member variables.
Note that {\nvecph} requires {\sundials} to be built with {\mpi} support.


% ====================================================================
\subsection{NVECTOR\_PARHYP functions}
\label{ss:nvec_parhyp_functions}
% ====================================================================

The {\nvecph} module defines implementations of all vector operations 
listed in Tables \ref{ss:nvecops}, \ref{ss:nvecfusedops},
\ref{ss:nvecarrayops}, and \ref{ss:nveclocalops}, except
for \id{N\_VSetArrayPointer} and \id{N\_VGetArrayPointer}, because accessing raw vector
data is handled by low-level {\hypre} functions.
As such, this vector is not available for use with {\sundials} Fortran interfaces.
When access to raw vector data is needed, one
should extract the {\hypre} vector first, and then use {\hypre}
methods to access the data. Usage examples of {\nvecph} are provided in
the \id{cvAdvDiff\_non\_ph.c} example program for {\cvode} \cite{cvode_ex}
and the \id{ark\_diurnal\_kry\_ph.c} example program for {\arkode} \cite{arkode_ex}.

The names of parhyp methods are obtained from those in Tables \ref{ss:nvecops},
\ref{ss:nvecfusedops}, \ref{ss:nvecarrayops}, and \ref{ss:nveclocalops}
by appending the suffix \id{\_ParHyp} (e.g. \id{N\_VDestroy\_ParHyp}).
The module {\nvecph} provides the following additional user-callable routines:
%%--------------------------------------
\sunmodfun{N\_VNewEmpty\_ParHyp}
{
  This function creates a new parhyp \id{N\_Vector} with the pointer to the {\hypre} 
  vector set to \id{NULL}.
}
{
  N\_Vector N\_VNewEmpty\_ParHyp(MPI\_Comm comm, sunindextype local\_length,
  \newlinefill{N\_Vector N\_VNewEmpty\_ParHyp}
  sunindextype global\_length)
}
%%--------------------------------------
\sunmodfun{N\_VMake\_ParHyp}
{  
  This function creates an \id{N\_Vector} wrapper around an existing
  {\hypre} parallel vector. It does {\em not} allocate memory for \id{x} 
  itself.  
}
{
  N\_Vector N\_VMake\_ParHyp(HYPRE\_ParVector x)
}
%%--------------------------------------
\sunmodfun{N\_VGetVector\_ParHyp}
{  
  This function returns the underlying {\hypre} vector.
}
{
  HYPRE\_ParVector N\_VGetVector\_ParHyp(N\_Vector v)
}
%%--------------------------------------
\sunmodfun{N\_VCloneVectorArray\_ParHyp}
{ 
  This function creates (by cloning) an array of \id{count} parallel vectors.
}
{
  N\_Vector *N\_VCloneVectorArray\_ParHyp(int count, N\_Vector w)
}
%%--------------------------------------
\sunmodfun{N\_VCloneVectorArrayEmpty\_ParHyp}
{
  This function creates (by cloning) an array of \id{count} parallel vectors,
  each with an empty (\id{NULL}) data array.
}
{
  N\_Vector *N\_VCloneVectorArrayEmpty\_ParHyp(int count, N\_Vector w)
}
%%--------------------------------------
\sunmodfun{N\_VDestroyVectorArray\_ParHyp}
{ 
  This function frees memory allocated for the array of \id{count}  variables of
  type \id{N\_Vector} created with \id{N\_VCloneVectorArray\_ParHyp} or with
  \id{N\_VCloneVectorArrayEmpty\_ParHyp}.
}
{
  void N\_VDestroyVectorArray\_ParHyp(N\_Vector *vs, int count)
}
%%--------------------------------------
\sunmodfun{N\_VPrint\_ParHyp}
{  
  This function prints the local content of a parhyp vector to \id{stdout}.
}
{   
  void N\_VPrint\_ParHyp(N\_Vector v)
}
%%--------------------------------------
\sunmodfun{N\_VPrintFile\_ParHyp}
{  
  This function prints the local content of a parhyp vector to \id{outfile}.
}
{   
  void N\_VPrintFile\_ParHyp(N\_Vector v, FILE *outfile)
}

By default all fused and vector array operations are disabled in the {\nvecph}
module. The following additional user-callable routines are provided to
enable or disable fused and vector array operations for a specific vector. To
ensure consistency across vectors it is recommended to first create a vector
with \id{N\_VMake\_ParHyp}, enable/disable the desired operations for that vector
with the functions below, and create any additional vectors from that vector
using \id{N\_VClone}. This guarantees the new vectors will have the same
operations enabled/disabled as cloned vectors inherit the same enable/disable
options as the vector they are cloned from while vectors created with
\id{N\_VMake\_ParHyp} will have the default settings for the {\nvecph} module.
%%--------------------------------------
\sunmodfun{N\_VEnableFusedOps\_ParHyp}
{
  This function enables (\id{SUNTRUE}) or disables (\id{SUNFALSE}) all fused and
  vector array operations in the parhyp vector. The return value is \id{0} for
  success and \id{-1} if the input vector or its \id{ops} structure are \id{NULL}.
}
{
  int N\_VEnableFusedOps\_ParHyp(N\_Vector v, booleantype tf)
}
%%--------------------------------------
\sunmodfun{N\_VEnableLinearCombination\_ParHyp}
{
  This function enables (\id{SUNTRUE}) or disables (\id{SUNFALSE}) the linear
  combination fused operation in the parhyp vector. The return value is \id{0} for
  success and \id{-1} if the input vector or its \id{ops} structure are \id{NULL}.
}
{
  int N\_VEnableLinearCombination\_ParHyp(N\_Vector v, booleantype tf)
}
%%--------------------------------------
\sunmodfun{N\_VEnableScaleAddMulti\_ParHyp}
{
  This function enables (\id{SUNTRUE}) or disables (\id{SUNFALSE}) the scale and
  add a vector to multiple vectors fused operation in the parhyp vector. The
  return value is \id{0} for success and \id{-1} if the input vector or its
  \id{ops} structure are \id{NULL}.
}
{
  int N\_VEnableScaleAddMulti\_ParHyp(N\_Vector v, booleantype tf)
}
%%--------------------------------------
\sunmodfun{N\_VEnableDotProdMulti\_ParHyp}
{
  This function enables (\id{SUNTRUE}) or disables (\id{SUNFALSE}) the multiple
  dot products fused operation in the parhyp vector. The return value is \id{0}
  for success and \id{-1} if the input vector or its \id{ops} structure are
  \id{NULL}.
}
{
  int N\_VEnableDotProdMulti\_ParHyp(N\_Vector v, booleantype tf)
}
%%--------------------------------------
\sunmodfun{N\_VEnableLinearSumVectorArray\_ParHyp}
{
  This function enables (\id{SUNTRUE}) or disables (\id{SUNFALSE}) the linear sum
  operation for vector arrays in the parhyp vector. The return value is \id{0} for
  success and \id{-1} if the input vector or its \id{ops} structure are \id{NULL}.
}
{
  int N\_VEnableLinearSumVectorArray\_ParHyp(N\_Vector v, booleantype tf)
}
%%--------------------------------------
\sunmodfun{N\_VEnableScaleVectorArray\_ParHyp}
{
  This function enables (\id{SUNTRUE}) or disables (\id{SUNFALSE}) the scale
  operation for vector arrays in the parhyp vector. The return value is \id{0} for
  success and \id{-1} if the input vector or its \id{ops} structure are \id{NULL}.
}
{
  int N\_VEnableScaleVectorArray\_ParHyp(N\_Vector v, booleantype tf)
}
%%--------------------------------------
\sunmodfun{N\_VEnableConstVectorArray\_ParHyp}
{
  This function enables (\id{SUNTRUE}) or disables (\id{SUNFALSE}) the const
  operation for vector arrays in the parhyp vector. The return value is \id{0} for
  success and \id{-1} if the input vector or its \id{ops} structure are \id{NULL}.
}
{
  int N\_VEnableConstVectorArray\_ParHyp(N\_Vector v, booleantype tf)
}
%%--------------------------------------
\sunmodfun{N\_VEnableWrmsNormVectorArray\_ParHyp}
{
  This function enables (\id{SUNTRUE}) or disables (\id{SUNFALSE}) the WRMS norm
  operation for vector arrays in the parhyp vector. The return value is \id{0} for
  success and \id{-1} if the input vector or its \id{ops} structure are \id{NULL}.
}
{
  int N\_VEnableWrmsNormVectorArray\_ParHyp(N\_Vector v, booleantype tf)
}
%%--------------------------------------
\sunmodfun{N\_VEnableWrmsNormMaskVectorArray\_ParHyp}
{
  This function enables (\id{SUNTRUE}) or disables (\id{SUNFALSE}) the masked WRMS
  norm operation for vector arrays in the parhyp vector. The return value is
  \id{0} for success and \id{-1} if the input vector or its \id{ops} structure are
  \id{NULL}.
}
{
  int N\_VEnableWrmsNormMaskVectorArray\_ParHyp(N\_Vector v, booleantype tf)
}
%%--------------------------------------
\sunmodfun{N\_VEnableScaleAddMultiVectorArray\_ParHyp}
{
  This function enables (\id{SUNTRUE}) or disables (\id{SUNFALSE}) the scale and
  add a vector array to multiple vector arrays operation in the parhyp vector. The
  return value is \id{0} for success and \id{-1} if the input vector or its
  \id{ops} structure are \id{NULL}.
}
{
  int N\_VEnableScaleAddMultiVectorArray\_ParHyp(N\_Vector v,
  \newlinefill{int N\_VEnableScaleAddMultiVectorArray\_ParHyp}
  booleantype tf)
}
%%--------------------------------------
\sunmodfun{N\_VEnableLinearCombinationVectorArray\_ParHyp}
{
  This function enables (\id{SUNTRUE}) or disables (\id{SUNFALSE}) the linear
  combination operation for vector arrays in the parhyp vector. The return value
  is \id{0} for success and \id{-1} if the input vector or its \id{ops} structure
  are \id{NULL}.
}
{
  int N\_VEnableLinearCombinationVectorArray\_ParHyp(N\_Vector v,
  \newlinefill{int N\_VEnableLinearCombinationVectorArray\_ParHyp}
  booleantype tf)
}
%%
%%------------------------------------
%%
\paragraph{\bf Notes} 
           
\begin{itemize}
                                        
\item
  When there is a need to access components of an \id{N\_Vector\_ParHyp}, \id{v}, 
  it is recommended to extract the {\hypre} vector via       
  \id{x\_vec = N\_VGetVector\_ParHyp(v)} and then access components using 
  appropriate {\hypre} functions.        
                                                               
\item
  {\warn}\id{N\_VNewEmpty\_ParHyp}, \id{N\_VMake\_ParHyp}, 
  and \id{N\_VCloneVectorArrayEmpty\_ParHyp} set the field 
  {\em own\_parvector} to \id{SUNFALSE}. 
  \id{N\_VDestroy\_ParHyp} and \id{N\_VDestroyVectorArray\_ParHyp}
  will not attempt to delete an underlying {\hypre} vector for any \id{N\_Vector} 
  with {\em own\_parvector} set to \id{SUNFALSE}. In such a case, it is the 
  user's responsibility to delete the underlying vector.

\item
  {\warn}To maximize efficiency, vector operations in the {\nvecph} implementation
  that have more than one \id{N\_Vector} argument do not check for
  consistent internal representations of these vectors. It is the user's 
  responsibility to ensure that such routines are called with \id{N\_Vector}
  arguments that were all created with the same internal representations.

\end{itemize}

% This is a shared SUNDIALS TEX file with description of
% the PETSc nvector wrapper implementation
%
\section{The NVECTOR\_PETSC implementation}\label{ss:nvec_petsc}

The {\nvecpetsc} module is an {\nvector} wrapper around the {\petsc} vector.
It defines the {\em content} field of a \id{N\_Vector} to be a structure containing
the global and local lengths of the vector, a pointer to the {\petsc} vector,
an {\mpi} communicator, and a boolean flag {\em own\_data} indicating ownership of 
the wrapped {\petsc} vector.
%%
\begin{verbatim} 
struct _N_VectorContent_Petsc {
  sunindextype local_length;
  sunindextype global_length;
  booleantype own_data;
  Vec *pvec;
  MPI_Comm comm;
};
\end{verbatim}
%%
%%--------------------------------------------
The header file to include when using this module is \id{nvector\_petsc.h}.
The installed module library to link to is
\id{libsundials\_nvecpetsc.\textit{lib}}
where \id{\em.lib} is typically \id{.so} for shared libraries and \id{.a}
for static libraries.

Unlike native {\sundials} vector types, {\nvecpetsc} does not provide macros 
to access its member variables.
Note that {\nvecpetsc} requires {\sundials} to be built with {\mpi} support.


% ====================================================================
\subsection{NVECTOR\_PETSC functions}
\label{ss:nvec_petsc_functions}
% ====================================================================

The {\nvecpetsc} module defines implementations of all vector operations listed 
in Tables \ref{ss:nvecops}, \ref{ss:nvecfusedops}, \ref{ss:nvecarrayops},
and \ref{ss:nveclocalops}, except for
\id{N\_VGetArrayPointer} and \id{N\_VSetArrayPointer}. As such, this vector cannot be
used with {\sundials} Fortran interfaces.
When access to raw vector data is needed, it is 
recommended to extract the {\petsc} vector first, and then use {\petsc} 
methods to access the data. Usage examples of {\nvecpetsc} are provided in
example programs for {\ida} \cite{ida_ex}.

The names of vector operations are obtained from those in 
Tables \ref{ss:nvecops}, \ref{ss:nvecfusedops}, \ref{ss:nvecarrayops}, and
\ref{ss:nveclocalops} by appending the
suffix \id{\_Petsc} (e.g. \id{N\_VDestroy\_Petsc}).
The module {\nvecpetsc}  provides the following additional user-callable routines:
%%--------------------------------------
\sunmodfun{N\_VNewEmpty\_Petsc}
{ 
  This function creates a new {\nvector} wrapper with the pointer to
  the wrapped {\petsc} vector set to (\id{NULL}). It is used by the 
  \id{N\_VMake\_Petsc} and \id{N\_VClone\_Petsc} implementations. 
}
{
  N\_Vector N\_VNewEmpty\_Petsc(MPI\_Comm comm, sunindextype local\_length, 
  \newlinefill{N\_Vector N\_VNewEmpty\_Petsc}
  sunindextype global\_length)
}
%%--------------------------------------
\sunmodfun{N\_VMake\_Petsc}
{  
  This function creates and allocates memory for an {\nvecpetsc}
  wrapper around a user-provided {\petsc} vector. It does {\em not} 
  allocate memory for the vector \id{pvec} itself.
}
{
  N\_Vector N\_VMake\_Petsc(Vec *pvec)
}
%%--------------------------------------
\sunmodfun{N\_VGetVector\_Petsc}
{  
  This function returns a pointer to the underlying {\petsc} vector.
}
{
  Vec *N\_VGetVector\_Petsc(N\_Vector v)
}
%%--------------------------------------
\sunmodfun{N\_VCloneVectorArray\_Petsc}
{ 
  This function creates (by cloning) an array of \id{count} {\nvecpetsc} vectors.
}
{
  N\_Vector *N\_VCloneVectorArray\_Petsc(int count, N\_Vector w)
}
%%--------------------------------------
\sunmodfun{N\_VCloneVectorArrayEmpty\_Petsc}
{ 
  This function creates (by cloning) an array of \id{count} {\nvecpetsc} vectors,
  each with pointers to {\petsc} vectors set to (\id{NULL}).
}
{
  N\_Vector *N\_VCloneVectorArrayEmpty\_Petsc(int count, N\_Vector w)
}
%%--------------------------------------
\sunmodfun{N\_VDestroyVectorArray\_Petsc}
{
  This function frees memory allocated for the array of \id{count} variables of
  type \id{N\_Vector} created with \id{N\_VCloneVectorArray\_Petsc} or with
  \id{N\_VCloneVectorArrayEmpty\_Petsc}.
}
{
  void N\_VDestroyVectorArray\_Petsc(N\_Vector *vs, int count)
}
%%--------------------------------------
\sunmodfun{N\_VPrint\_Petsc}
{
  This function prints the global content of a wrapped {\petsc} vector to \id{stdout}.
}
{
  void N\_VPrint\_Petsc(N\_Vector v)
}
%%--------------------------------------
\sunmodfun{N\_VPrintFile\_Petsc}
{  
  This function prints the global content of a wrapped {\petsc} vector to \id{fname}.
}
{
  void N\_VPrintFile\_Petsc(N\_Vector v, const char fname[])
}
%%--------------------------------------

By default all fused and vector array operations are disabled in the {\nvecpetsc}
module. The following additional user-callable routines are provided to
enable or disable fused and vector array operations for a specific vector. To
ensure consistency across vectors it is recommended to first create a vector
with \id{N\_VMake\_Petsc}, enable/disable the desired operations for that vector
with the functions below, and create any additional vectors from that vector
using \id{N\_VClone}. This guarantees the new vectors will have the same
operations enabled/disabled as cloned vectors inherit the same enable/disable
options as the vector they are cloned from while vectors created with
\id{N\_VMake\_Petsc} will have the default settings for the {\nvecpetsc} module.
%%--------------------------------------
\sunmodfun{N\_VEnableFusedOps\_Petsc}
{
  This function enables (\id{SUNTRUE}) or disables (\id{SUNFALSE}) all fused and
  vector array operations in the {\petsc} vector. The return value is \id{0} for
  success and \id{-1} if the input vector or its \id{ops} structure are \id{NULL}.
}
{
  int N\_VEnableFusedOps\_Petsc(N\_Vector v, booleantype tf)
}
%%--------------------------------------
\sunmodfun{N\_VEnableLinearCombination\_Petsc}
{
  This function enables (\id{SUNTRUE}) or disables (\id{SUNFALSE}) the linear
  combination fused operation in the {\petsc} vector. The return value is \id{0} for
  success and \id{-1} if the input vector or its \id{ops} structure are \id{NULL}.
}
{
  int N\_VEnableLinearCombination\_Petsc(N\_Vector v, booleantype tf)
}
%%--------------------------------------
\sunmodfun{N\_VEnableScaleAddMulti\_Petsc}
{
  This function enables (\id{SUNTRUE}) or disables (\id{SUNFALSE}) the scale and
  add a vector to multiple vectors fused operation in the {\petsc} vector. The
  return value is \id{0} for success and \id{-1} if the input vector or its
  \id{ops} structure are \id{NULL}.
}
{
  int N\_VEnableScaleAddMulti\_Petsc(N\_Vector v, booleantype tf)
}
%%--------------------------------------
\sunmodfun{N\_VEnableDotProdMulti\_Petsc}
{
  This function enables (\id{SUNTRUE}) or disables (\id{SUNFALSE}) the multiple
  dot products fused operation in the {\petsc} vector. The return value is \id{0}
  for success and \id{-1} if the input vector or its \id{ops} structure are
  \id{NULL}.
}
{
  int N\_VEnableDotProdMulti\_Petsc(N\_Vector v, booleantype tf)
}
%%--------------------------------------
\sunmodfun{N\_VEnableLinearSumVectorArray\_Petsc}
{
  This function enables (\id{SUNTRUE}) or disables (\id{SUNFALSE}) the linear sum
  operation for vector arrays in the {\petsc} vector. The return value is \id{0} for
  success and \id{-1} if the input vector or its \id{ops} structure are \id{NULL}.
}
{
  int N\_VEnableLinearSumVectorArray\_Petsc(N\_Vector v, booleantype tf)
}
%%--------------------------------------
\sunmodfun{N\_VEnableScaleVectorArray\_Petsc}
{
  This function enables (\id{SUNTRUE}) or disables (\id{SUNFALSE}) the scale
  operation for vector arrays in the {\petsc} vector. The return value is \id{0} for
  success and \id{-1} if the input vector or its \id{ops} structure are \id{NULL}.
}
{
  int N\_VEnableScaleVectorArray\_Petsc(N\_Vector v, booleantype tf)
}
%%--------------------------------------
\sunmodfun{N\_VEnableConstVectorArray\_Petsc}
{
  This function enables (\id{SUNTRUE}) or disables (\id{SUNFALSE}) the const
  operation for vector arrays in the {\petsc} vector. The return value is \id{0} for
  success and \id{-1} if the input vector or its \id{ops} structure are \id{NULL}.
}
{
  int N\_VEnableConstVectorArray\_Petsc(N\_Vector v, booleantype tf)
}
%%--------------------------------------
\sunmodfun{N\_VEnableWrmsNormVectorArray\_Petsc}
{
  This function enables (\id{SUNTRUE}) or disables (\id{SUNFALSE}) the WRMS norm
  operation for vector arrays in the {\petsc} vector. The return value is \id{0} for
  success and \id{-1} if the input vector or its \id{ops} structure are \id{NULL}.
}
{
  int N\_VEnableWrmsNormVectorArray\_Petsc(N\_Vector v, booleantype tf)
}
%%--------------------------------------
\sunmodfun{N\_VEnableWrmsNormMaskVectorArray\_Petsc}
{
  This function enables (\id{SUNTRUE}) or disables (\id{SUNFALSE}) the masked WRMS
  norm operation for vector arrays in the {\petsc} vector. The return value is
  \id{0} for success and \id{-1} if the input vector or its \id{ops} structure are
  \id{NULL}.
}
{
  int N\_VEnableWrmsNormMaskVectorArray\_Petsc(N\_Vector v, booleantype tf)
}
%%--------------------------------------
\sunmodfun{N\_VEnableScaleAddMultiVectorArray\_Petsc}
{
  This function enables (\id{SUNTRUE}) or disables (\id{SUNFALSE}) the scale and
  add a vector array to multiple vector arrays operation in the {\petsc} vector. The
  return value is \id{0} for success and \id{-1} if the input vector or its
  \id{ops} structure are \id{NULL}.
}
{
  int N\_VEnableScaleAddMultiVectorArray\_Petsc(N\_Vector v, booleantype tf)
}
%%--------------------------------------
\sunmodfun{N\_VEnableLinearCombinationVectorArray\_Petsc}
{
  This function enables (\id{SUNTRUE}) or disables (\id{SUNFALSE}) the linear
  combination operation for vector arrays in the {\petsc} vector. The return value
  is \id{0} for success and \id{-1} if the input vector or its \id{ops} structure
  are \id{NULL}.
}
{
  int N\_VEnableLinearCombinationVectorArray\_Petsc(N\_Vector v,
  \newlinefill{int N\_VEnableLinearCombinationVectorArray\_Petsc}
  booleantype tf)
}
%%
%%------------------------------------
%%
\paragraph{\bf Notes} 
           
\begin{itemize}
                                        
\item
  When there is a need to access components of an \id{N\_Vector\_Petsc}, \id{v}, 
  it is recommeded to extract the {\petsc} vector via       
  \id{x\_vec = N\_VGetVector\_Petsc(v)} and then access components using 
  appropriate {\petsc} functions.        
                                                               
\item
  {\warn}The functions \id{N\_VNewEmpty\_Petsc}, \id{N\_VMake\_Petsc}, and
  \id{N\_VCloneVectorArrayEmpty\_Petsc} set the field {\em own\_data} to \id{SUNFALSE}.   
  \id{N\_VDestroy\_Petsc} and \id{N\_VDestroyVectorArray\_Petsc}
  will not attempt to free the pointer {\em pvec} for any \id{N\_Vector} with
  {\em own\_data} set to \id{SUNFALSE}. In such a case, it is the user's responsibility to
  deallocate the {\em pvec} pointer.

\item
  {\warn}To maximize efficiency, vector operations in the {\nvecpetsc} implementation
  that have more than one \id{N\_Vector} argument do not check for
  consistent internal representations of these vectors. It is the user's 
  responsibility to ensure that such routines are called with \id{N\_Vector}
  arguments that were all created with the same internal representations.

\end{itemize}


% This is a shared SUNDIALS TEX file with description of
% the CUDA nvector implementation
%
\section{The NVECTOR\_CUDA implementation}\label{ss:nvec_cuda}

The {\nveccuda} module is an {\nvector} implementation in the {\cuda} language.
The module allows for {\sundials} vector kernels to run on NVIDIA GPU devices. It is intended
for users who are already familiar with {\cuda} and GPU programming. Building this vector
module requires a CUDA compiler and, by extension, a C++ compiler. The vector content layout
is as follows:

\begin{verbatim}
struct _N_VectorContent_Cuda
{
  sunindextype       length;
  booleantype        own_exec;
  booleantype        own_helper;
  SUNMemory          host_data;
  SUNMemory          device_data;
  SUNCudaExecPolicy* stream_exec_policy;
  SUNCudaExecPolicy* reduce_exec_policy;
  SUNMemoryHelper    mem_helper;
  void*              priv; /* 'private' data */
};

typedef struct _N_VectorContent_Cuda *N_VectorContent_Cuda;
\end{verbatim}

The content members are the vector length (size), ownership flags for the
\id{*\_exec\_policy} fields and the \id{mem\_helper} field, \id{SUNMemory}
objects for the vector data on the host and the device, pointers to
\id{SUNCudaExecPolicy} implementations that control how the CUDA kernels are
launched for streaming and reduction vector kernels, a \id{SUNMemoryHelper}
object, and a private data structure which holds additonal members that should
not be accessed directly.

When instantiated with \id{N\_VNew\_Cuda}, the underlying data will be allocated
memory on both the host and the device. Alternatively, a user can provide host
and device data arrays by using the \id{N\_VMake\_Cuda} constructor. To use {\cuda}
managed memory, the constructors \id{N\_VNewManaged\_Cuda} and \newline
\id{N\_VMakeManaged\_Cuda} are provided. Details on each of these constructors
are provided below.

To use the {\nveccuda} module, the header file to include is \id{nvector\_cuda.h},
and the library to link to is \id{libsundials\_nveccuda.\textit{lib}}. The
extension \id{\textit{.lib}} is typically \id{.so} for shared libraries and \id{.a}
for static libraries.

% ====================================================================
\subsection{NVECTOR\_CUDA functions}
\label{ss:nvec_cuda_functions}
% ====================================================================

Unlike other native {\sundials} vector types, {\nveccuda} does not provide macros
to access its member variables. Instead, user should use the accessor functions:
%%--------------------------------------
\sunmodfun{N\_VGetHostArrayPointer\_Cuda}
{
  This function returns a pointer to the vector data on the host.
}
{
  realtype *N\_VGetHostArrayPointer\_Cuda(N\_Vector v)
}
%%--------------------------------------
\sunmodfun{N\_VGetDeviceArrayPointer\_Cuda}
{
  This function returns a pointer to the vector data on the device.
}
{
  realtype *N\_VGetDeviceArrayPointer\_Cuda(N\_Vector v)
}
%%--------------------------------------
\sunmodfun{N\_VSetHostArrayPointer\_Cuda}
{
  This function sets the pointer to the vector data on the host.
  The existing pointer \textit{will not} be freed first.
}
{
  realtype *N\_VSetHostArrayPointer\_Cuda(N\_Vector v)
}
%%--------------------------------------
\sunmodfun{N\_VSetDeviceArrayPointer\_Cuda}
{
  This function sets pointer to the vector data on the device.
  The existing pointer \textit{will not} be freed first.
}
{
  realtype *N\_VSetDeviceArrayPointer\_Cuda(N\_Vector v)
}
%%--------------------------------------
\sunmodfun{N\_VIsManagedMemory\_Cuda}
{
  This function returns a boolean flag indicating if the vector
  data is allocated in managed memory or not.
}
{
  booleantype *N\_VIsManagedMemory\_Cuda(N\_Vector v)
}
%%--------------------------------------------

The {\nveccuda} module defines implementations of all vector operations listed
in Tables \ref{ss:nvecops}, \ref{ss:nvecfusedops}, \ref{ss:nvecarrayops}
and \ref{ss:nveclocalops}, except for \id{N\_VSetArrayPointer} and
\id{N\_VGetArrayPointer} unless managed memory is used.
As such, this vector can only be used with the {\sundials} Fortran interfaces,
and the {\sundials} direct solvers and preconditioners when using managed memory.
The {\nveccuda} module provides separate functions to access data on the host
and on the device for the unmanaged memory use case. It also provides methods
for copying from the host to the device and vice versa. Usage examples of
{\nveccuda} are provided in some example programs for {\cvode} \cite{cvode_ex}.

The names of vector operations are obtained from those in Tables \ref{ss:nvecops},
\ref{ss:nvecfusedops}, \ref{ss:nvecarrayops}, and \ref{ss:nveclocalops}
by appending the suffix \id{\_Cuda}
(e.g. \id{N\_VDestroy\_Cuda}). The module {\nveccuda} provides the following
functions:
%%--------------------------------------
\sunmodfun{N\_VNew\_Cuda}
{
  This function creates and allocates memory for a {\cuda} \id{N\_Vector}.
  The vector data array is allocated on both the host and device.
}
{
  N\_Vector N\_VNew\_Cuda(sunindextype length)
}
%%--------------------------------------
\sunmodfun{N\_VNewManaged\_Cuda}
{
  This function creates and allocates memory for a {\cuda} \id{N\_Vector}.
  The vector data array is allocated in managed memory.
}
{
  N\_Vector N\_VNewManaged\_Cuda(sunindextype length)
}
%%--------------------------------------
\sunmodfun{N\_VNewWithMemHelp\_Cuda}
{
  This function creates an {\nveccuda} which will use the \id{SUNMemoryHelper}
  object to allocate memory. If \id{use\_managed\_memory} is 0, then unmanaged
  memory is used, otherwise managed memory is used.
}
{
  N\_Vector N\_VNewWithMemHelp\_Cuda(sunindextype length,
                                     booleantype use\_managed\_mem,
                                     SUNMemoryHelper helper);
}
%%--------------------------------------
\sunmodfun{N\_VNewEmpty\_Cuda}
{
  This function creates a new {\nvector} wrapper with the pointer to
  the wrapped {\cuda} vector set to \id{NULL}. It is used by the
  \id{N\_VNew\_Cuda}, \id{N\_VMake\_Cuda}, and \id{N\_VClone\_Cuda}
  implementations.
}
{
  N\_Vector N\_VNewEmpty\_Cuda()
}
%%--------------------------------------
\sunmodfun{N\_VMake\_Cuda}
{
  This function creates an {\nveccuda} with user-supplied vector data arrays
  \id{h\_vdata} and \id{d\_vdata}. This function does not allocate memory for
  data itself.
}
{
  N\_Vector N\_VMake\_Cuda(sunindextype length, realtype *h\_data, realtype *dev\_data)
}
%%--------------------------------------
\sunmodfun{N\_VMakeManaged\_Cuda}
{
  This function creates an {\nveccuda} with a user-supplied managed memory data
  array. This function does not allocate memory for data itself.
}
{
  N\_Vector N\_VMakeManaged\_Cuda(sunindextype length, realtype *vdata)
}
%%--------------------------------------
\sunmodfun{N\_VMakeWithManagedAllocator\_Cuda}
{
  This function creates an {\nveccuda} with a user-supplied memory allocator.
  It requires the user to provide a corresponding free function as well.
  The memory allocated by the allocator function must behave like CUDA managed memory.

  \warn This function is deprecated and will be removed in the next major release.
  Use \id{N\_VNewWithMemHelp\_Cuda} instead.
}
{
  N\_Vector N\_VMakeWithManagedAllocator\_Cuda(sunindextype length,
                                               void* (*allocfn)(size\_t size),
                                               void (*freefn)(void* ptr));
}

The module {\nveccuda} also provides the following user-callable routines:
%%--------------------------------------
\sunmodfun{N\_VSetKernelExecPolicy\_Cuda}
{
  This function sets the execution policies which control the kernel parameters
  utilized when launching the streaming and reduction CUDA kernels. By default
  the vector is setup to use the \id{SUNCudaThreadDirectExecPolicy} and
  \id{SUNCudaBlockReduceExecPolicy}. Any custom execution policy for reductions
  must ensure that the grid dimensions (number of thread blocks) is a multiple of
  the CUDA warp size (32). See section \ref{ss:suncudaexecpolicy} below for more
  information about the \id{SUNCudaExecPolicy} class.

  \textit{Note: All vectors used in a single instance of a {\sundials} solver must
  use the same execution policy. It is \textbf{strongly recommended} that
  this function is called immediately after constructing the vector, and
  any subsequent vector be created by cloning to ensure consistent execution
  policies across vectors.}
}
{
  void N\_VSetKernelExecPolicy\_Cuda(N\_Vector v,
                                    SUNCudaExecPolicy* stream\_exec\_policy,
                                    SUNCudaExecPolicy* reduce\_exec\_policy);
}
%%--------------------------------------
\sunmodfun{N\_VSetCudaStream\_Cuda}
{
  This function sets the {\cuda} stream that all vector kernels will be launched on.
  By default an {\nveccuda} uses the default {\cuda} stream.\\

  \textit{Note: All vectors used in a single instance of a {\sundials} solver must
  use the same {\cuda} stream. It is \textbf{strongly recommended} that
  this function is called immediately after constructing the vector, and
  any subsequent vector be created by cloning to ensure consistent execution
  policies across vectors.}

  \warn This function will be removed in the next major release,
  user should utilize the \id{N\_VSetKernelExecPolicy\_Cuda} function instead.
}
{
  void N\_VSetCudaStream\_Cuda(N\_Vector v, cudaStream\_t *stream)
}
%%--------------------------------------
\sunmodfun{N\_VCopyToDevice\_Cuda}
{
 This function copies host vector data to the device.
}
{
 void N\_VCopyToDevice\_Cuda(N\_Vector v)
}
%%--------------------------------------
\sunmodfun{N\_VCopyFromDevice\_Cuda}
{
 This function copies vector data from the device to the host.
}
{
 void N\_VCopyFromDevice\_Cuda(N\_Vector v)
}
%%--------------------------------------
\sunmodfun{N\_VPrint\_Cuda}
{
  This function prints the content of a {\cuda} vector to \id{stdout}.
}
{
  void N\_VPrint\_Cuda(N\_Vector v)
}
%%--------------------------------------
\sunmodfun{N\_VPrintFile\_Cuda}
{
  This function prints the content of a {\cuda} vector to \id{outfile}.
}
{
  void N\_VPrintFile\_Cuda(N\_Vector v, FILE *outfile)
}
%%--------------------------------------

By default all fused and vector array operations are disabled in the {\nveccuda}
module. The following additional user-callable routines are provided to
enable or disable fused and vector array operations for a specific vector. To
ensure consistency across vectors it is recommended to first create a vector
with \id{N\_VNew\_Cuda}, enable/disable the desired operations for that vector
with the functions below, and create any additional vectors from that vector
using \id{N\_VClone}. This guarantees the new vectors will have the same
operations enabled/disabled as cloned vectors inherit the same enable/disable
options as the vector they are cloned from while vectors created with
\id{N\_VNew\_Cuda} will have the default settings for the {\nveccuda} module.
%%--------------------------------------
\sunmodfun{N\_VEnableFusedOps\_Cuda}
{
  This function enables (\id{SUNTRUE}) or disables (\id{SUNFALSE}) all fused and
  vector array operations in the {\cuda} vector. The return value is \id{0} for
  success and \id{-1} if the input vector or its \id{ops} structure are \id{NULL}.
}
{
  int N\_VEnableFusedOps\_Cuda(N\_Vector v, booleantype tf)
}
%%--------------------------------------
\sunmodfun{N\_VEnableLinearCombination\_Cuda}
{
  This function enables (\id{SUNTRUE}) or disables (\id{SUNFALSE}) the linear
  combination fused operation in the {\cuda} vector. The return value is \id{0} for
  success and \id{-1} if the input vector or its \id{ops} structure are \id{NULL}.
}
{
  int N\_VEnableLinearCombination\_Cuda(N\_Vector v, booleantype tf)
}
%%--------------------------------------
\sunmodfun{N\_VEnableScaleAddMulti\_Cuda}
{
  This function enables (\id{SUNTRUE}) or disables (\id{SUNFALSE}) the scale and
  add a vector to multiple vectors fused operation in the {\cuda} vector. The
  return value is \id{0} for success and \id{-1} if the input vector or its
  \id{ops} structure are \id{NULL}.
}
{
  int N\_VEnableScaleAddMulti\_Cuda(N\_Vector v, booleantype tf)
}
%%--------------------------------------
\sunmodfun{N\_VEnableDotProdMulti\_Cuda}
{
  This function enables (\id{SUNTRUE}) or disables (\id{SUNFALSE}) the multiple
  dot products fused operation in the {\cuda} vector. The return value is \id{0}
  for success and \id{-1} if the input vector or its \id{ops} structure are
  \id{NULL}.
}
{
  int N\_VEnableDotProdMulti\_Cuda(N\_Vector v, booleantype tf)
}
%%--------------------------------------
\sunmodfun{N\_VEnableLinearSumVectorArray\_Cuda}
{
  This function enables (\id{SUNTRUE}) or disables (\id{SUNFALSE}) the linear sum
  operation for vector arrays in the {\cuda} vector. The return value is \id{0} for
  success and \id{-1} if the input vector or its \id{ops} structure are \id{NULL}.
}
{
  int N\_VEnableLinearSumVectorArray\_Cuda(N\_Vector v, booleantype tf)
}
%%--------------------------------------
\sunmodfun{N\_VEnableScaleVectorArray\_Cuda}
{
  This function enables (\id{SUNTRUE}) or disables (\id{SUNFALSE}) the scale
  operation for vector arrays in the {\cuda} vector. The return value is \id{0} for
  success and \id{-1} if the input vector or its \id{ops} structure are \id{NULL}.
}
{
  int N\_VEnableScaleVectorArray\_Cuda(N\_Vector v, booleantype tf)
}
%%--------------------------------------
\sunmodfun{N\_VEnableConstVectorArray\_Cuda}
{
  This function enables (\id{SUNTRUE}) or disables (\id{SUNFALSE}) the const
  operation for vector arrays in the {\cuda} vector. The return value is \id{0} for
  success and \id{-1} if the input vector or its \id{ops} structure are \id{NULL}.
}
{
  int N\_VEnableConstVectorArray\_Cuda(N\_Vector v, booleantype tf)
}
%%--------------------------------------
\sunmodfun{N\_VEnableWrmsNormVectorArray\_Cuda}
{
  This function enables (\id{SUNTRUE}) or disables (\id{SUNFALSE}) the WRMS norm
  operation for vector arrays in the {\cuda} vector. The return value is \id{0} for
  success and \id{-1} if the input vector or its \id{ops} structure are \id{NULL}.
}
{
  int N\_VEnableWrmsNormVectorArray\_Cuda(N\_Vector v, booleantype tf)
}
%%--------------------------------------
\sunmodfun{N\_VEnableWrmsNormMaskVectorArray\_Cuda}
{
  This function enables (\id{SUNTRUE}) or disables (\id{SUNFALSE}) the masked WRMS
  norm operation for vector arrays in the {\cuda} vector. The return value is
  \id{0} for success and \id{-1} if the input vector or its \id{ops} structure are
  \id{NULL}.
}
{
  int N\_VEnableWrmsNormMaskVectorArray\_Cuda(N\_Vector v, booleantype tf)
}
%%--------------------------------------
\sunmodfun{N\_VEnableScaleAddMultiVectorArray\_Cuda}
{
  This function enables (\id{SUNTRUE}) or disables (\id{SUNFALSE}) the scale and
  add a vector array to multiple vector arrays operation in the {\cuda} vector. The
  return value is \id{0} for success and \id{-1} if the input vector or its
  \id{ops} structure are \id{NULL}.
}
{
  int N\_VEnableScaleAddMultiVectorArray\_Cuda(N\_Vector v, booleantype tf)
}
%%--------------------------------------
\sunmodfun{N\_VEnableLinearCombinationVectorArray\_Cuda}
{
  This function enables (\id{SUNTRUE}) or disables (\id{SUNFALSE}) the linear
  combination operation for vector arrays in the {\cuda} vector. The return value
  is \id{0} for success and \id{-1} if the input vector or its \id{ops} structure
  are \id{NULL}.
}
{
  int N\_VEnableLinearCombinationVectorArray\_Cuda(N\_Vector v,
  \newlinefill{int N\_VEnableLinearCombinationVectorArray\_Cuda}
  booleantype tf)
}
%%
%%------------------------------------
%%
\paragraph{\bf Notes}

\begin{itemize}

\item
  When there is a need to access components of an \id{N\_Vector\_Cuda}, \id{v},
  it is recommeded to use functions \id{N\_VGetDeviceArrayPointer\_Cuda} or
  \id{N\_VGetHostArrayPointer\_Cuda}. However, when using managed memory, the
  function \id{N\_VGetArrayPointer} may also be used.

\item
  Performance is better if the \id{SUNMemoryHelper} provided supports \id{SUNMEMTYPE\_PINNED};
  the default \id{SUNMemoryHelper} does provide this support. In the case that it does,
  then the buffers used for reductions will be allocated as pinned memory.

\item
  {\warn}To maximize efficiency, vector operations in the {\nveccuda} implementation
  that have more than one \id{N\_Vector} argument do not check for
  consistent internal representations of these vectors. It is the user's
  responsibility to ensure that such routines are called with \id{N\_Vector}
  arguments that were all created with the same internal representations.

\end{itemize}

%% Eventually we should move this section to a "Using <package> in GPU Environments" section
\subsection{The SUNCudaExecPolicy Class}\label{ss:suncudaexecpolicy}

In order to provide maximum flexibility to users, the CUDA kernel execution parameters used
by kernels within SUNDIALS are defined by objects of the \id{sundials::CudaExecPolicy}
abstract class type (this class can be accessed in the global namespace as \id{SUNCudaExecPolicy}).
Thus, users may provide custom execution policies that fit the needs of their problem. The
\id{sundials::CudaExecPolicy} is defined in the header file \id{sundials\_cuda\_policies.hpp},
and is as follows:

\begin{verbatim}
class CudaExecPolicy
{
public:
  virtual size_t gridSize(size_t numWorkUnits = 0, size_t blockDim = 0) const = 0;
  virtual size_t blockSize(size_t numWorkUnits = 0, size_t gridDim = 0) const = 0;
  virtual cudaStream_t stream() const = 0;
  virtual CudaExecPolicy* clone() const = 0;
  virtual ~CudaExecPolicy() {}
};
\end{verbatim}

To define a custom execution policy, a user simply needs to create a class that inherits from
the abstract class and implements the methods. The {\sundials} provided
\id{sundials::CudaThreadDirectExecPolicy} (aka in the global namespace as
\id{SUNCudaThreadDirectExecPolicy}) class is a good example of a what a custom execution policy
may look like:

\begin{verbatim}
class CudaThreadDirectExecPolicy : public CudaExecPolicy
{
public:
  CudaThreadDirectExecPolicy(const size_t blockDim, const cudaStream_t stream = 0)
    : blockDim_(blockDim), stream_(stream)
  {}

  CudaThreadDirectExecPolicy(const CudaThreadDirectExecPolicy& ex)
    : blockDim_(ex.blockDim_), stream_(ex.stream_)
  {}

  virtual size_t gridSize(size_t numWorkUnits = 0, size_t blockDim = 0) const
  {
    return (numWorkUnits + blockSize() - 1) / blockSize();
  }

  virtual size_t blockSize(size_t numWorkUnits = 0, size_t gridDim = 0) const
  {
    return blockDim_;
  }

  virtual cudaStream_t stream() const
  {
    return stream_;
  }

  virtual CudaExecPolicy* clone() const
  {
    return static_cast<CudaExecPolicy*>(new CudaThreadDirectExecPolicy(*this));
  }

private:
  const cudaStream_t stream_;
  const size_t blockDim_;
};
\end{verbatim}

In total, {\sundials} provides 3 execution policies:

\begin{enumerate}
  \item \id{SUNCudaThreadDirectExecPolicy(const size\_t blockDim, const cudaStream\_t stream = 0)}
    maps each CUDA thread to a work unit. The number of threads per block (blockDim) can be set
    to anything. The grid size will be calculated so that there are enough threads for one
    thread per element. If a CUDA stream is provided, it will be used to execute the kernel.

  \item \id{SUNCudaGridStrideExecPolicy(const size\_t blockDim, const size\_t gridDim, const cudaStream\_t stream = 0)}
    is for kernels that use grid stride loops. The number of threads per block (blockDim)
    can be set to anything. The number of blocks (gridDim) can be set to anything. If a
    CUDA stream is provided, it will be used to execute the kernel.

  \item \id{SUNCudaBlockReduceExecPolicy(const size\_t blockDim, const size\_t gridDim, const cudaStream\_t stream = 0)}
    is for kernels performing a reduction across individual thread blocks. The number of threads
    per block (blockDim) can be set to any valid multiple of the CUDA warp size. The grid size
    (gridDim) can be set to any value greater than 0. If it is set to 0, then the grid size
    will be chosen so that there is enough threads for one thread per work unit. If a
    CUDA stream is provided, it will be used to execute the kernel.
\end{enumerate}

For example, a policy that uses 128 threads per block and a user provided stream can be
created like so:

\begin{verbatim}
  cudaStream_t stream;
  cudaStreamCreate(&stream);
  SUNCudaThreadDirectExecPolicy thread_direct(128, stream);
\end{verbatim}

These default policy objects can be reused for multiple {\sundials} data structures
since they do not hold any modifiable state information.
% This is a shared SUNDIALS TEX file with description of
% the CUDA nvector implementation
%
\section{The NVECTOR\_RAJA implementation}\label{ss:nvec_raja}

The {\nvecraja} module is an experimental {\nvector} implementation using the
\href{https://software.llnl.gov/RAJA/}{\raja} hardware abstraction layer.
In this implementation, {\raja}
allows for {\sundials} vector kernels to run on GPU devices. The module is intended for users
who are already familiar with {\raja} and GPU programming. Building this vector
module requires a C++11 compliant compiler and a CUDA software development toolkit.
Besides the {\cuda} backend, {\raja} has other backends such as serial, OpenMP,
and OpenACC. These backends are not used in this {\sundials} release.
Class \id{Vector} in namespace \id{sunrajavec} manages the vector data layout:
\begin{verbatim}
template <class T, class I>
class Vector {
   I size_;
   I mem_size_;
   I global_size_;
   T* h_vec_;
   T* d_vec_;
   SUNMPI_Comm comm_;
  ...
};
\end{verbatim}
The class members are: vector size (length), size of the vector data
memory block, the global vector size (length), pointers to vector data
on the host and on the device, and the MPI communicator. The class
\id{Vector} inherits from an empty structure
\begin{verbatim}
struct _N_VectorContent_Raja { };
\end{verbatim}
to interface the C++ class with the {\nvector} C code. When instantiated, the class
\id{Vector} will allocate memory on both the host and the device. Due to the rapid
progress of {\raja} development, we expect that the \id{sunrajavec::Vector}
class will change frequently in future {\sundials} releases. The code is
structured so that it can tolerate significant changes in the
\id{sunrajavec::Vector} class without requiring changes to the user API.

%%
%%--------------------------------------------

The {\nvecraja} module can be utilized for single-node parallelism or in a distributed context with MPI.
The header file to include when using this module for single-node parallelism is \id{nvector\_raja.h}.
The header file to include when using this module in the distributed case is \id{nvector\_mpiraja.h}.
The installed module libraries to link to are \id{libsundials\_nvecraja.\textit{lib}} in the single-node case, or \id{libsundials\_nvecmpicudaraja.\textit{lib}} in the distributed case. Only one one of these libraries may be linked to when creating an executable or library. {\sundials} must be built with
MPI support if the distributed library is desired.


% ====================================================================
\subsection{NVECTOR\_RAJA functions}
\label{ss:nvec_raja_functions}
% ====================================================================

Unlike other native {\sundials} vector types, {\nvecraja} does not provide macros
to access its member variables. Instead, user should use the accessor functions:
%%--------------------------------------
\sunmodfun{N\_VGetLocalLength\_Raja}
{
  This function returns the local length of the vector.

  Note: This function is for use in a \textit{distributed context} and
  is defined in the header \id{nvector\_mpiraja.h} and the library
  to link to is \id{libsundials\_nvecmpicudaraja}.\text{lib}.
}
{
  sunindextype N\_VGetLocalLength\_Raja(N\_Vector v)
}
%%--------------------------------------
\sunmodfun{N\_VGetHostArrayPointer\_Raja}
{
  This function returns a pointer to the vector data on the host.
}
{
  realtype *N\_VGetHostArrayPointer\_Raja(N\_Vector v)
}
%%--------------------------------------
\sunmodfun{N\_VGetDeviceArrayPointer\_Raja}
{
  This function returns a pointer to the vector data on the device.
}
{
  realtype *N\_VGetDeviceArrayPointer\_Raja(N\_Vector v)
}
%%--------------------------------------
\sunmodfun{N\_VGetMPIComm\_Raja}
{
  This function returns the MPI communicator for the vector.

  Note: This function is for use in a \textit{distributed context}
  and is defined in the header \id{nvector\_mpiraja.h} and the
  library to link to is \id{libsundials\_nvecmpicudaraja}.\text{lib}.
}
{
  MPI\_Comm N\_VGetMPIComm\_Raja(N\_Vector v)
}
%%--------------------------------------

The {\nvecraja} module defines the implementations of all vector operations listed
in Tables \ref{t:nvecops}, \ref{t:nvecfusedops}, \ref{t:nvecarrayops},
and \ref{t:nveclocalops}, except
for \id{N\_VDotProdMulti}, \id{N\_VWrmsNormVectorArray}, and \\ \noindent
\id{N\_VWrmsNormMaskVectorArray} as support for arrays of reduction vectors is not
yet supported in {\raja}. These function will be added to the {\nvecraja}
implementation in the future. Additionally the vector operations \id{N\_VGetArrayPointer} and
\id{N\_VSetArrayPointer} are not implemented by the {\raja} vector.
As such, this vector cannot be used with the {\sundials} Fortran interfaces,
nor with the {\sundials} direct solvers and preconditioners.
The {\nvecraja} module provides separate functions to access data on the host
and on the device. It also provides methods for copying data from the host to
the device and vice versa. Usage examples of {\nvecraja} are provided in
some example programs for {\cvode} \cite{cvode_ex}.

The names of vector operations are obtained from those in Tables \ref{t:nvecops},
\ref{t:nvecfusedops}, \ref{t:nvecarrayops}, and \ref{t:nveclocalops}
by appending the suffix \id{\_Raja} (e.g. \id{N\_VDestroy\_Raja}).
The module {\nvecraja}  provides the following additional user-callable routines:
%%--------------------------------------
\sunmodfuns{N\_VNew\_Raja}
{
  This function creates and allocates memory for a {\cuda} \id{N\_Vector}.
  The vector data array is allocated on both the host and device.
}
{
  In the \textit{single-node} setting, the only input is the vector length. This
  constructor is defined in the header \id{nvector\_raja.h} and the library to
  link to is \id{libsundials\_nveccudaraja}.\text{lib}.
}
{
  N\_Vector N\_VNew\_Raja(sunindextype length)
}
{
  When used in a \textit{distributed} context with MPI, the arguments are the
  {\mpi} communicator, the local vector lenght, and the global vector length.
  This constructor is defined in the header \id{nvector\_mpiraja.h} and
  the library to link to is \id{libsundials\_nvecmpicudaraja}.\text{lib}.
}
{
  N\_Vector N\_VNew\_Raja(MPI\_Comm comm, sunindextype local\_length,
  \newlinefill{N\_Vector N\_VNew\_Raja}
  sunindextype global\_length)
}
%%--------------------------------------
\sunmodfun{N\_VNewEmpty\_Raja}
{
  This function creates a new {\nvector} wrapper with the pointer to
  the wrapped {\raja} vector set to \id{NULL}. It is used by the
  \id{N\_VNew\_Raja}, \id{N\_VMake\_Raja}, and \id{N\_VClone\_Raja}
  implementations.
}
{
  N\_Vector N\_VNewEmpty\_Raja()
}
%%--------------------------------------
\sunmodfun{N\_VMake\_Raja}
{
  This function creates and allocates memory for an {\nvecraja}
  wrapper around a user-provided \id{sunrajavec::Vector} class.
  Its only argument is of type \newline
  \id{N\_VectorContent\_Raja}, which is the pointer to the class.
}
{
  N\_Vector N\_VMake\_Raja(N\_VectorContent\_Raja c)
}
%%--------------------------------------
\sunmodfun{N\_VCopyToDevice\_Raja}
{
 This function copies host vector data to the device.
}
{
 realtype *N\_VCopyToDevice\_Raja(N\_Vector v)
}
%%--------------------------------------
\sunmodfun{N\_VCopyFromDevice\_Raja}
{
  This function copies vector data from the device to the host.
}
{
  realtype *N\_VCopyFromDevice\_Raja(N\_Vector v)
}
%%--------------------------------------
\sunmodfun{N\_VPrint\_Raja}
{
  This function prints the content of a {\raja} vector to \id{stdout}.
}
{
  void N\_VPrint\_Raja(N\_Vector v)
}
%%--------------------------------------
\sunmodfun{N\_VPrintFile\_Raja}
{
  This function prints the content of a {\raja} vector to \id{outfile}.
}
{
  void N\_VPrintFile\_Raja(N\_Vector v, FILE *outfile)
}
%%--------------------------------------

By default all fused and vector array operations are disabled in the {\nvecraja}
module. The following additional user-callable routines are provided to
enable or disable fused and vector array operations for a specific vector. To
ensure consistency across vectors it is recommended to first create a vector
with \id{N\_VNew\_Raja}, enable/disable the desired operations for that vector
with the functions below, and create any additional vectors from that vector
using \id{N\_VClone}. This guarantees the new vectors will have the same
operations enabled/disabled as cloned vectors inherit the same enable/disable
options as the vector they are cloned from while vectors created with
\id{N\_VNew\_Raja} will have the default settings for the {\nvecraja} module.
%%--------------------------------------
\sunmodfun{N\_VEnableFusedOps\_Raja}
{
  This function enables (\id{SUNTRUE}) or disables (\id{SUNFALSE}) all fused and
  vector array operations in the {\raja} vector. The return value is \id{0} for
  success and \id{-1} if the input vector or its \id{ops} structure are \id{NULL}.
}
{
  int N\_VEnableFusedOps\_Raja(N\_Vector v, booleantype tf)
}
%%--------------------------------------
\sunmodfun{N\_VEnableLinearCombination\_Raja}
{
  This function enables (\id{SUNTRUE}) or disables (\id{SUNFALSE}) the linear
  combination fused operation in the {\raja} vector. The return value is \id{0} for
  success and \id{-1} if the input vector or its \id{ops} structure are \id{NULL}.
}
{
  int N\_VEnableLinearCombination\_Raja(N\_Vector v, booleantype tf)
}
%%--------------------------------------
\sunmodfun{N\_VEnableScaleAddMulti\_Raja}
{
  This function enables (\id{SUNTRUE}) or disables (\id{SUNFALSE}) the scale and
  add a vector to multiple vectors fused operation in the {\raja} vector. The
  return value is \id{0} for success and \id{-1} if the input vector or its
  \id{ops} structure are \id{NULL}.
}
{
  int N\_VEnableScaleAddMulti\_Raja(N\_Vector v, booleantype tf)
}
%%--------------------------------------
\sunmodfun{N\_VEnableLinearSumVectorArray\_Raja}
{
  This function enables (\id{SUNTRUE}) or disables (\id{SUNFALSE}) the linear sum
  operation for vector arrays in the {\raja} vector. The return value is \id{0} for
  success and \id{-1} if the input vector or its \id{ops} structure are \id{NULL}.
}
{
  int N\_VEnableLinearSumVectorArray\_Raja(N\_Vector v, booleantype tf)
}
%%--------------------------------------
\sunmodfun{N\_VEnableScaleVectorArray\_Raja}
{
  This function enables (\id{SUNTRUE}) or disables (\id{SUNFALSE}) the scale
  operation for vector arrays in the {\raja} vector. The return value is \id{0} for
  success and \id{-1} if the input vector or its \id{ops} structure are \id{NULL}.
}
{
  int N\_VEnableScaleVectorArray\_Raja(N\_Vector v, booleantype tf)
}
%%--------------------------------------
\sunmodfun{N\_VEnableConstVectorArray\_Raja}
{
  This function enables (\id{SUNTRUE}) or disables (\id{SUNFALSE}) the const
  operation for vector arrays in the {\raja} vector. The return value is \id{0} for
  success and \id{-1} if the input vector or its \id{ops} structure are \id{NULL}.
}
{
  int N\_VEnableConstVectorArray\_Raja(N\_Vector v, booleantype tf)
}
%%--------------------------------------
\sunmodfun{N\_VEnableScaleAddMultiVectorArray\_Raja}
{
  This function enables (\id{SUNTRUE}) or disables (\id{SUNFALSE}) the scale and
  add a vector array to multiple vector arrays operation in the {\raja} vector. The
  return value is \id{0} for success and \id{-1} if the input vector or its
  \id{ops} structure are \id{NULL}.
}
{
  int N\_VEnableScaleAddMultiVectorArray\_Raja(N\_Vector v, booleantype tf)
}
%%--------------------------------------
\sunmodfun{N\_VEnableLinearCombinationVectorArray\_Raja}
{
  This function enables (\id{SUNTRUE}) or disables (\id{SUNFALSE}) the linear
  combination operation for vector arrays in the {\raja} vector. The return value
  is \id{0} for success and \id{-1} if the input vector or its \id{ops} structure
  are \id{NULL}.
}
{
  int N\_VEnableLinearCombinationVectorArray\_Raja(N\_Vector v,
  \newlinefill{int N\_VEnableLinearCombinationVectorArray\_Raja}
  booleantype tf)
}
%%
%%------------------------------------
%%
\paragraph{\bf Notes}

\begin{itemize}

\item
  When there is a need to access components of an \id{N\_Vector\_Raja}, \id{v},
  it is recommeded to use functions \id{N\_VGetDeviceArrayPointer\_Raja} or
  \id{N\_VGetHostArrayPointer\_Raja}.

% \item
%   {\warn}Unlike in other {\nvector} implementations, vector data will always be
%   deleted when invoking \id{N\_VDestroy\_Raja} and \id{N\_VDestroyVectorArray\_Raja},
%   even when the vector is created using \id{N\_VMake\_Raja}. It is user's responsibility
%   to track memory allocations and deletions when using \id{N\_VMake\_Raja}.

\item
  {\warn}To maximize efficiency, vector operations in the {\nvecraja} implementation
  that have more than one \id{N\_Vector} argument do not check for
  consistent internal representations of these vectors. It is the user's
  responsibility to ensure that such routines are called with \id{N\_Vector}
  arguments that were all created with the same internal representations.

\end{itemize}

%% This is a shared SUNDIALS TEX file with a description of the
%% OpenMPDEV nvector implementation
%%
\section{The NVECTOR\_OPENMPDEV implementation}\label{ss:nvec_openmpdev}

In situations where a user has access to a device such as a GPU for
offloading computation, {\sundials} provides an {\nvector} implementation using
OpenMP device offloading, called {\nvecopenmpdev}.

The {\nvecopenmpdev} implementation defines the \textit{content} field
of the \id{N\_Vector} to be a structure  containing the length of the vector, a pointer
to the beginning of a contiguous  data array on the host, a pointer to the beginning of
a contiguous data array on the device, and a boolean flag \id{own\_data} which specifies
the ownership of host and device data arrays.
%%
\begin{verbatim}
struct _N_VectorContent_OpenMPDEV {
  sunindextype length;
  booleantype own_data;
  realtype *host_data;
  realtype *dev_data;
};
\end{verbatim}
%%
%%--------------------------------------------

The header file to include when using this module is \id{nvector\_openmpdev.h}.
The installed module library to link to is
\id{libsundials\_nvecopenmpdev.\textit{lib}}
where \id{\em.lib} is typically \id{.so} for shared libraries and \id{.a}
for static libraries.


% ====================================================================
\subsection{NVECTOR\_OPENMPDEV accessor macros}
\label{ss:nvec_openmpdev_macros}
% ====================================================================

The following macros are provided to access the content of an {\nvecopenmpdev}
vector.
%%
\begin{itemize}

\item \ID{NV\_CONTENT\_OMPDEV}

  This routine gives access to the contents of the {\nvecopenmpdev}
  vector \id{N\_Vector}.

  The assignment \id{v\_cont} $=$ \id{NV\_CONTENT\_OMPDEV(v)} sets
  \id{v\_cont} to be a pointer to the {\nvecopenmpdev} \id{N\_Vector} content
  structure.

  Implementation:

  \verb|#define NV_CONTENT_OMPDEV(v) ( (N_VectorContent_OpenMPDEV)(v->content) )|

\item \ID{NV\_OWN\_DATA\_OMPDEV}, \ID{NV\_DATA\_HOST\_OMPDEV}, \ID{NV\_DATA\_DEV\_OMPDEV}, \ID{NV\_LENGTH\_OMPDEV}

  These macros give individual access to the parts of
  the content of an {\nvecopenmpdev} \id{N\_Vector}.

  The assignment \id{v\_data = NV\_DATA\_HOST\_OMPDEV(v)} sets \id{v\_data} to be
  a pointer to the first component of the data on the host for the \id{N\_Vector} \id{v}.
  The assignment \id{NV\_DATA\_HOST\_OMPDEV(v) = v\_data} sets the host component array of \id{v} to
  be \id{v\_data} by storing the pointer \id{v\_data}.

  The assignment \id{v\_dev\_data = NV\_DATA\_DEV\_OMPDEV(v)} sets \id{v\_dev\_data} to be
  a pointer to the first component of the data on the device for the \id{N\_Vector} \id{v}.
  The assignment \id{NV\_DATA\_DEV\_OMPDEV(v) = v\_dev\_data} sets the device component array of \id{v} to
  be \id{v\_dev\_data} by storing the pointer \id{v\_dev\_data}.

  The assignment \id{v\_len = NV\_LENGTH\_OMPDEV(v)} sets \id{v\_len} to be
  the length of \id{v}. On the other hand, the call \id{NV\_LENGTH\_OMPDEV(v) = len\_v}
  sets the length of \id{v} to be \id{len\_v}.

  Implementation:

  \verb|#define NV_OWN_DATA_OMPDEV(v) ( NV_CONTENT_OMPDEV(v)->own_data )|

  \verb|#define NV_DATA_HOST_OMPDEV(v) ( NV_CONTENT_OMPDEV(v)->host_data )|

  \verb|#define NV_DATA_DEV_OMPDEV(v) ( NV_CONTENT_OMPDEV(v)->dev_data )|

  \verb|#define NV_LENGTH_OMPDEV(v) ( NV_CONTENT_OMPDEV(v)->length )|

\end{itemize}


% ====================================================================
\subsection{NVECTOR\_OPENMPDEV functions}
\label{ss:nvec_openmpdev_functions}
% ====================================================================%

The {\nvecopenmpdev} module defines OpenMP device offloading implementations of
all vector operations listed in Tables \ref{t:nvecops}, \ref{t:nvecfusedops},
\ref{t:nvecarrayops}, and \ref{t:nveclocalops}, except for \id{N\_VGetArrayPointer} and
\id{N\_VSetArrayPointer}. As such, this vector cannot be used with the
{\sundials} Fortran interfaces, nor with the {\sundials} direct solvers and
preconditioners. It also provides methods for copying from the host to
the device and vice versa.

The names of vector operations are obtained from those in Tables
\ref{t:nvecops}, \ref{t:nvecfusedops}, \ref{t:nvecarrayops}, and
\ref{t:nveclocalops} by appending the
suffix \id{\_OpenMPDEV} (e.g. \id{N\_VDestroy\_OpenMPDEV}). The module
{\nvecopenmpdev} provides the following additional user-callable routines:
%%--------------------------------------
\sunmodfun{N\_VNew\_OpenMPDEV}
{
  This function creates and allocates memory for an {\nvecopenmpdev} \id{N\_Vector}.
}
{
  N\_Vector N\_VNew\_OpenMPDEV(sunindextype vec\_length)
}
%%--------------------------------------
\sunmodfun{N\_VNewEmpty\_OpenMPDEV}
{
  This function creates a new {\nvecopenmpdev} \id{N\_Vector} with an empty
  (\id{NULL}) host and device data arrays.
}
{
  N\_Vector N\_VNewEmpty\_OpenMPDEV(sunindextype vec\_length)
}
%%--------------------------------------
\sunmodfun{N\_VMake\_OpenMPDEV}
{
 This function creates an {\nvecopenmpdev} vector with user-supplied vector data
 arrays \id{h\_vdata} and \id{d\_vdata}. This function does not allocate memory for
 data itself.
}
{
  N\_Vector N\_VMake\_OpenMPDEV(sunindextype vec\_length, realtype *h\_vdata,
  \newlinefill{N\_Vector N\_VMake\_OpenMPDEV}
  realtype *d\_vdata)
}
%%--------------------------------------
\sunmodfun{N\_VCloneVectorArray\_OpenMPDEV}
{
 This function creates (by cloning) an array of \id{count} {\nvecopenmpdev} vectors.
}
{
 N\_Vector *N\_VCloneVectorArray\_OpenMPDEV(int count, N\_Vector w)
}
%%--------------------------------------
\sunmodfun{N\_VCloneVectorArrayEmpty\_OpenMPDEV}
{
 This function creates (by cloning) an array of \id{count} {\nvecopenmpdev} vectors, each with an
 empty (\id{NULL}) data array.
}
{
 N\_Vector *N\_VCloneVectorArrayEmpty\_OpenMPDEV(int count, N\_Vector w)
}
%%--------------------------------------
\sunmodfun{N\_VDestroyVectorArray\_OpenMPDEV}
{
 This function frees memory allocated for the array of \id{count} variables of type
 \id{N\_Vector} created with \id{N\_VCloneVectorArray\_OpenMPDEV} or with \newline
 \id{N\_VCloneVectorArrayEmpty\_OpenMPDEV}.
}
{
 void N\_VDestroyVectorArray\_OpenMPDEV(N\_Vector *vs, int count)
}
%%--------------------------------------
\sunmodfun{N\_VGetHostArrayPointer\_OpenMPDEV}
{
 This function returns a pointer to the host data array.
}
{
 realtype *N\_VGetHostArrayPointer\_OpenMPDEV(N\_Vector v)
}
%%--------------------------------------
\sunmodfun{N\_VGetDeviceArrayPointer\_OpenMPDEV}
{
 This function returns a pointer to the device data array.
}
{
 realtype *N\_VGetDeviceArrayPointer\_OpenMPDEV(N\_Vector v)
}
%%--------------------------------------
\sunmodfun{N\_VPrint\_OpenMPDEV}
{
 This function prints the content of an {\nvecopenmpdev} vector to \id{stdout}.
}
{
 void N\_VPrint\_OpenMPDEV(N\_Vector v)
}
%%--------------------------------------
\sunmodfun{N\_VPrintFile\_OpenMPDEV}
{
 This function prints the content of an {\nvecopenmpdev} vector to \id{outfile}.
}
{
 void N\_VPrintFile\_OpenMPDEV(N\_Vector v, FILE *outfile)
}
%%--------------------------------------
\sunmodfun{N\_VCopyToDevice\_OpenMPDEV}
{
 This function copies the content of an {\nvecopenmpdev} vector's host data array
 to the device data array.
}
{
 void N\_VCopyToDevice\_OpenMPDEV(N\_Vector v)
}
%%--------------------------------------
\sunmodfun{N\_VCopyFromDevice\_OpenMPDEV}
{
 This function copies the content of an {\nvecopenmpdev} vector's device data array
 to the host data array.
}
{
  void N\_VCopyFromDevice\_OpenMPDEV(N\_Vector v)
}
%%--------------------------------------

By default all fused and vector array operations are disabled in the {\nvecopenmpdev}
module. The following additional user-callable routines are provided to
enable or disable fused and vector array operations for a specific vector. To
ensure consistency across vectors it is recommended to first create a vector
with \id{N\_VNew\_OpenMPDEV}, enable/disable the desired operations for that vector
with the functions below, and create any additional vectors from that vector
using \id{N\_VClone}. This guarantees the new vectors will have the same
operations enabled/disabled as cloned vectors inherit the same enable/disable
options as the vector they are cloned from while vectors created with
\id{N\_VNew\_OpenMPDEV} will have the default settings for the {\nvecopenmpdev} module.
%%--------------------------------------
\sunmodfun{N\_VEnableFusedOps\_OpenMPDEV}
{
  This function enables (\id{SUNTRUE}) or disables (\id{SUNFALSE}) all fused and
  vector array operations in the {\nvecopenmpdev} vector. The return value is \id{0} for
  success and \id{-1} if the input vector or its \id{ops} structure are \id{NULL}.
}
{
  int N\_VEnableFusedOps\_OpenMPDEV(N\_Vector v, booleantype tf)
}
%%--------------------------------------
\sunmodfun{N\_VEnableLinearCombination\_OpenMPDEV}
{
  This function enables (\id{SUNTRUE}) or disables (\id{SUNFALSE}) the linear
  combination fused operation in the {\nvecopenmpdev} vector. The return value is \id{0} for
  success and \id{-1} if the input vector or its \id{ops} structure are \id{NULL}.
}
{
  int N\_VEnableLinearCombination\_OpenMPDEV(N\_Vector v, booleantype tf)
}
%%--------------------------------------
\sunmodfun{N\_VEnableScaleAddMulti\_OpenMPDEV}
{
  This function enables (\id{SUNTRUE}) or disables (\id{SUNFALSE}) the scale and
  add a vector to multiple vectors fused operation in the {\nvecopenmpdev} vector. The
  return value is \id{0} for success and \id{-1} if the input vector or its
  \id{ops} structure are \id{NULL}.
}
{
  int N\_VEnableScaleAddMulti\_OpenMPDEV(N\_Vector v, booleantype tf)
}
%%--------------------------------------
\sunmodfun{N\_VEnableDotProdMulti\_OpenMPDEV}
{
  This function enables (\id{SUNTRUE}) or disables (\id{SUNFALSE}) the multiple
  dot products fused operation in the {\nvecopenmpdev} vector. The return value is \id{0}
  for success and \id{-1} if the input vector or its \id{ops} structure are
  \id{NULL}.
}
{
  int N\_VEnableDotProdMulti\_OpenMPDEV(N\_Vector v, booleantype tf)
}
%%--------------------------------------
\sunmodfun{N\_VEnableLinearSumVectorArray\_OpenMPDEV}
{
  This function enables (\id{SUNTRUE}) or disables (\id{SUNFALSE}) the linear sum
  operation for vector arrays in the {\nvecopenmpdev} vector. The return value is \id{0} for
  success and \id{-1} if the input vector or its \id{ops} structure are \id{NULL}.
}
{
  int N\_VEnableLinearSumVectorArray\_OpenMPDEV(N\_Vector v, booleantype tf)
}
%%--------------------------------------
\sunmodfun{N\_VEnableScaleVectorArray\_OpenMPDEV}
{
  This function enables (\id{SUNTRUE}) or disables (\id{SUNFALSE}) the scale
  operation for vector arrays in the {\nvecopenmpdev} vector. The return value is \id{0} for
  success and \id{-1} if the input vector or its \id{ops} structure are \id{NULL}.
}
{
  int N\_VEnableScaleVectorArray\_OpenMPDEV(N\_Vector v, booleantype tf)
}
%%--------------------------------------
\sunmodfun{N\_VEnableConstVectorArray\_OpenMPDEV}
{
  This function enables (\id{SUNTRUE}) or disables (\id{SUNFALSE}) the const
  operation for vector arrays in the {\nvecopenmpdev} vector. The return value is \id{0} for
  success and \id{-1} if the input vector or its \id{ops} structure are \id{NULL}.
}
{
  int N\_VEnableConstVectorArray\_OpenMPDEV(N\_Vector v, booleantype tf)
}
%%--------------------------------------
\sunmodfun{N\_VEnableWrmsNormVectorArray\_OpenMPDEV}
{
  This function enables (\id{SUNTRUE}) or disables (\id{SUNFALSE}) the WRMS norm
  operation for vector arrays in the {\nvecopenmpdev} vector. The return value is \id{0} for
  success and \id{-1} if the input vector or its \id{ops} structure are \id{NULL}.
}
{
  int N\_VEnableWrmsNormVectorArray\_OpenMPDEV(N\_Vector v, booleantype tf)
}
%%--------------------------------------
\sunmodfun{N\_VEnableWrmsNormMaskVectorArray\_OpenMPDEV}
{
  This function enables (\id{SUNTRUE}) or disables (\id{SUNFALSE}) the masked WRMS
  norm operation for vector arrays in the {\nvecopenmpdev} vector. The return value is
  \id{0} for success and \id{-1} if the input vector or its \id{ops} structure are
  \id{NULL}.
}
{
  int N\_VEnableWrmsNormMaskVectorArray\_OpenMPDEV(N\_Vector v,
  \newlinefill{int N\_VEnableWrmsNormMaskVectorArray\_OpenMPDEV}
  booleantype tf)
}
%%--------------------------------------
\sunmodfun{N\_VEnableScaleAddMultiVectorArray\_OpenMPDEV}
{
  This function enables (\id{SUNTRUE}) or disables (\id{SUNFALSE}) the scale and
  add a vector array to multiple vector arrays operation in the {\nvecopenmpdev} vector. The
  return value is \id{0} for success and \id{-1} if the input vector or its
  \id{ops} structure are \id{NULL}.
}
{
  int N\_VEnableScaleAddMultiVectorArray\_OpenMPDEV(N\_Vector v,
  \newlinefill{int N\_VEnableScaleAddMultiVectorArray\_OpenMPDEV}
  booleantype tf)
}
%%--------------------------------------
\sunmodfun{N\_VEnableLinearCombinationVectorArray\_OpenMPDEV}
{
  This function enables (\id{SUNTRUE}) or disables (\id{SUNFALSE}) the linear
  combination operation for vector arrays in the {\nvecopenmpdev} vector. The return value
  is \id{0} for success and \id{-1} if the input vector or its \id{ops} structure
  are \id{NULL}.
}
{
  int N\_VEnableLinearCombinationVectorArray\_OpenMPDEV(N\_Vector v,
  \newlinefill{int N\_VEnableLinearCombinationVectorArray\_OpenMPDEV}
  booleantype tf)
}
%%
%%------------------------------------
%%
\paragraph{\bf Notes}

\begin{itemize}

\item
  When looping over the components of an \id{N\_Vector} \id{v}, it is
  most efficient to first obtain the component array via
  \id{h\_data = NV\_DATA\_HOST\_OMPDEV(v)} for the host array or \newline
  \id{d\_data = NV\_DATA\_DEV\_OMPDEV(v)} for the device array and then access
  \id{h\_data[i]} or \id{d\_data[i]} within the loop.

\item
  When accessing individual components of an \id{N\_Vector} \id{v} on
  the host remember to first copy the array
  back from the device with \id{N\_VCopyFromDevice\_OpenMPDEV(v)}
  to ensure the array is up to date.

\item
  {\warn}\id{N\_VNewEmpty\_OpenMPDEV}, \id{N\_VMake\_OpenMPDEV},
  and \id{N\_VCloneVectorArrayEmpty\_OpenMPDEV} set the field
  \id{own\_data} $=$ \id{SUNFALSE}.
  \id{N\_VDestroy\_OpenMPDEV} and \id{N\_VDestroyVectorArray\_OpenMPDEV}
  will not attempt to free the pointer {\em data} for any \id{N\_Vector} with
  \id{own\_data} set to \id{SUNFALSE}. In such a case, it is the user's responsibility to
  deallocate the {\em data} pointer.

\item
  {\warn}To maximize efficiency, vector operations in the {\nvecopenmpdev} implementation
  that have more than one \id{N\_Vector} argument do not check for
  consistent internal representation of these vectors. It is the user's
  responsibility to ensure that such routines are called with \id{N\_Vector}
  arguments that were all created with the same internal representations.

\end{itemize}

% This is a shared SUNDIALS TEX file with description of
% the Trilinos nvector wrapper implementation
%
\section{The NVECTOR\_TRILINOS implementation}\label{ss:nvec_trilinos}

The {\nvectrilinos} module is an {\nvector} wrapper around the {\trilinos}
\href{https://github.com/trilinos/Trilinos}{Tpetra} vector. The interface
to Tpetra is implemented in the \id{Sundials::TpetraVectorInterface} class. This
class simply stores a reference counting pointer to a Tpetra vector and
inherits from an empty structure
%%
\begin{verbatim}
struct _N_VectorContent_Trilinos {};
\end{verbatim}
%%
%%--------------------------------------------
to interface the C++ class with the {\nvector} C code.
A pointer to an instance of this class is kept in the \id{content} field
of the \id{N\_Vector} object, to ensure that the Tpetra vector
is not deleted for as long as the \id{N\_Vector} object exists.

The Tpetra vector type in the \id{Sundials::TpetraVectorInterface} class is defined
as:
\begin{verbatim}
  typedef Tpetra::Vector<realtype, int, sunindextype> vector_type;
\end{verbatim}
The Tpetra vector will use the {\sundials}-specified \id{realtype} as its scalar
type, \id{int} as its local ordinal type, and \id{sunindextype} as the global ordinal type.
This type definition will use Tpetra's default node type. Available Kokkos node
types in {\trilinos} 12.14 release are serial (single thread), OpenMP, Pthread,
and {\cuda}. The default node type is selected when building the Kokkos package.
For example, the Tpetra vector will use a {\cuda} node if Tpetra was built with
{\cuda} support and the {\cuda} node was selected as the default when Tpetra was
built.

The header file to include when using this module is \id{nvector\_trilinos.h}.
The installed module library to link to is
\id{libsundials\_nvectrilinos.\textit{lib}}
where \id{\em.lib} is typically \id{.so} for shared libraries and \id{.a}
for static libraries.


% ====================================================================
\subsection{NVECTOR\_TRILINOS functions}
\label{ss:nvec_trilinos_functions}
% ====================================================================

The {\nvectrilinos} module defines implementations of all vector operations listed
in Tables \ref{ss:nvecops}, \ref{ss:nveclocalops}, and
\ref{ss:nveclocalops}, except for \verb|N_VGetArrayPointer| and
\verb|N_VSetArrayPointer|. As such, this vector cannot be used with
{\sundials} Fortran interfaces, nor with the {\sundials} direct
solvers and preconditioners. When access to raw vector data is needed, it is
recommended to extract the {\trilinos} Tpetra vector first, and then use Tpetra vector
methods to access the data. Usage examples of {\nvectrilinos} are provided in
example programs for {\ida} \cite{ida_ex}.

The names of vector operations are obtained from those in
Tables \ref{ss:nvecops}, \ref{ss:nveclocalops}, and \ref{ss:nveclocalops} by appending the
suffix \id{\_Trilinos} (e.g. \id{N\_VDestroy\_Trilinos}).
Vector operations call existing \id{Tpetra::Vector} methods when available. Vector
operations specific to {\sundials} are implemented as standalone functions in the namespace
\id{Sundials::TpetraVector}, located in the file \id{SundialsTpetraVectorKernels.hpp}.
The module {\nvectrilinos} provides the following additional user-callable functions:
%%
%%
\begin{itemize}


%%--------------------------------------

\item \ID{N\_VGetVector\_Trilinos}

  This C++ function takes an \id{N\_Vector} as the argument and returns a reference
  counting pointer to the underlying Tpetra vector. This is a standalone function
  defined in the global namespace.

\begin{verbatim}
Teuchos::RCP<vector_type> N_VGetVector_Trilinos(N_Vector v);
\end{verbatim}


%%--------------------------------------

\item \ID{N\_VMake\_Trilinos}

  This C++ function creates and allocates memory for an {\nvectrilinos}
  wrapper around a user-provided Tpetra vector. This is a standalone function
  defined in the global namespace.

\begin{verbatim}
N_Vector N_VMake_Trilinos(Teuchos::RCP<vector_type> v);
\end{verbatim}


\end{itemize}
%%
%%------------------------------------
%%
\paragraph{\bf Notes}

\begin{itemize}

\item

  The template parameter \id{vector\_type} should be set as:\\
  \verb|  typedef Sundials::TpetraVectorInterface::vector_type vector_type|\\
   This will ensure that data types used in Tpetra vector match those in {\sundials}.

\item
  When there is a need to access components of an \id{N\_Vector\_Trilinos}, \id{v},
  it is recommeded to extract the {\trilinos} vector object via
  \id{x\_vec = N\_VGetVector\_Trilinos(v)} and then access components using
  the appropriate {\trilinos} functions.

\item
  The functions \id{N\_VDestroy\_Trilinos} and \id{N\_VDestroyVectorArray\_Trilinos}
  only delete the \id{N\_Vector} wrapper. The underlying Tpetra vector object will exist for as long as
  there is at least one reference to it.

\end{itemize}

% This is a shared SUNDIALS TEX file with description of
% the ManyVector nvector implementation
%
\section{The NVECTOR\_MANYVECTOR implementation}\label{ss:nvec_manyvector}

The {\nvecmanyvector} implementation of the {\nvector} module provided
with {\sundials} is designed to facilitate problems with an inherent
data partitioning for the solution vector.  These data partitions are
entirely user-defined, through construction of distinct {\nvector}
modules for each component, that are then combined together to form
the {\nvecmanyvector}.  We envision three generic use cases for this
implementation:
\begin{itemize}
\item[A.] \emph{Heterogeneous computational architectures (serial or
    parallel)}: for users who wish to partition data on a node between
  different computing resources, they may create architecture-specific
  subvectors for each partition.  For example, a user could create one
  MPI-parallel component based on {\nvecp}, another single-node
  component for GPU accelerators based on {\nveccuda}, and another
  threaded single-node component based on {\nvecopenmp}.
\item[B.] \emph{Process-based multiphysics decompositions (parallel)}: for
  users who wish to combine separate simulations together, e.g., where
  one subvector resides on one subset of MPI processes, while another
  subvector resides on a different subset of MPI processes, and where
  the user has created a MPI \emph{intercommunicator} to connect these
  distinct process sets together.
\item[C.] \emph{Structure of arrays (SOA) data layouts (serial or
    parallel)}: for users who wish to create separate subvectors for
  each solution component, e.g., in a Navier-Stokes simulation they
  could have separate subvectors for density, velocities and
  pressure, which are combined together into a single
  {\nvecmanyvector} for the overall ``solution''. 
\end{itemize}
We note that the above use cases are not mutually exclusive, and the
{\nvecmanyvector} implementation should support arbitrary combinations
of these cases.

The {\nvecmanyvector} implementation is designed to work with any
{\nvector} subvectors that implement the minimum \emph{required} set
of operations, however significant performance benefits may be
obtained when subvectors additionally implement the optional local
reduction operations listed in Table \ref{t:nveclocalops}.

Additionally, {\nvecmanyvector} sets no limit on the number of
subvectors that may be attached (aside from the limitations of using
\id{sunindextype} for indexing, and standard per-node memory
limitations).  However, while this ostensibly supports subvectors
with one entry each (i.e., one subvector for each solution entry), we
anticipate that this extreme situation will hinder performance due to
non-stride-one memory accesses and increased function call overhead.
We therefore recommend a relatively coarse partitioning of the
problem, although actual performance will likely be
problem-dependent.

As a final note, in the coming years we plan to introduce additional
algebraic solvers and time integration modules that will leverage the
problem partitioning enabled by {\nvecmanyvector}.  However, even at
present we anticipate that users will be able to leverage such data
partitioning in their problem-defining ODE right-hand side, DAE
residual, or nonlinear solver residual functions.


% ====================================================================
\subsection{NVECTOR\_MANYVECTOR structure}
\label{ss:nvec_manyvector_functions}
% ====================================================================

The {\nvecmanyvector} implementation defines the {\em content} field
of \id{N\_Vector} to be a structure containing the MPI communicator
(or \id{SUNMPI\_COMM\_NULL} if running in serial), the number of
subvectors comprising the ManyVector, the global length of the
ManyVector (including all subvectors on all MPI tasks), a pointer to
the beginning of the array of subvectors, and a boolean flag
\id{own\_data} indicating ownership of the subvectors that populate
\id{subvec\_array}.
%%
\begin{verbatim} 
struct _N_VectorContent_ManyVector {
  SUNMPI_Comm   comm;            /* overall MPI communicator        */
  sunindextype  num_subvectors;  /* number of vectors attached      */
  sunindextype  global_length;   /* overall manyvector length       */
  N_Vector*     subvec_array;    /* pointer to N_Vector array       */
  booleantype   own_data;        /* flag indicating data ownership  */
};
\end{verbatim}
%%
%%--------------------------------------------

The header file to include when using this module is
\id{nvector\_manyvector.h}. The installed module library to link against is
\id{libsundials\_nvecmanyvector.\textit{lib}} where \id{\em.lib} is typically
\id{.so} for shared libraries and \id{.a} for static libraries.

\warn\textbf{Note:} If {\sundials} is configured with MPI enabled, then the
ManyVector library will be built for the parallel use case and \id{SUNMPI\_Comm}
is set to \id{MPI\_Comm}. As such an MPI-aware compiler will become necessary
even in single node uses of the ManyVector library. If {\sundials} is configured
with MPI disabled, then the ManyVector library is built for the single-node
(serial) use case and \id{SUNMPI\_Comm} is set to \id{int}. As such, users need
not include \id{mpi.h}, nor must executables be built with an MPI-aware
compiler. For more details see Appendix \ref{c:install}.


% ====================================================================
\subsection{NVECTOR\_MANYVECTOR functions}
\label{ss:nvec_manyvector_functions}
% ====================================================================

The {\nvecmanyvector} module implements all vector operations listed 
in Tables \ref{t:nvecops}, \ref{t:nvecfusedops}, \ref{t:nvecarrayops},
and \ref{t:nveclocalops}, except for \id{N\_VGetArrayPointer},
\id{N\_VSetArrayPointer}, \id{N\_VScaleAddMultiVectorArray}, and
\id{N\_VLinearCombinationVectorArray}.  As such, this vector cannot be
used with the {\sundials} Fortran interfaces, nor with the {\sundials}
direct solvers and preconditioners. Instead, the \\
{\nvecmanyvector} module provides functions to access subvectors,
whose data may in turn be accessed according to their {\nvector}
implementations.

The names of vector operations are obtained from those in Tables
\ref{t:nvecops}, \ref{t:nvecfusedops}, \ref{t:nvecarrayops}, and
\ref{t:nveclocalops} by appending the suffix \id{\_ManyVector} 
(e.g. \id{N\_VDestroy\_ManyVector}).
The module {\nvecmanyvector} provides the following additional
user-callable routines:
%%--------------------------------------
\sunmodfun{N\_VNew\_ManyVector}
{
  This function creates a ManyVector from a set of existing {\nvector}
  objects, under the requirement that all MPI-aware subvectors use the
  same MPI communicator (this is checked internally).  If none of the
  subvectors are MPI-aware, then this may equivalently be used to
  describe data partitioning within a single node.  We note that this
  routine is designed to support use cases A and C above.

  This routine will copy all \id{N\_Vector} pointers from the input
  \id{vec\_array}, so the user may modify/free that pointer array
  after calling this function.  However, this routine does \emph{not}
  allocate any new subvectors, so the underlying {\nvector} objects
  themselves should not be destroyed before the ManyVector that
  contains them.

  Upon successful completion, the new ManyVector is returned;
  otherwise this routine returns \id{NULL} (e.g., if two MPI-aware
  subvectors use different MPI communicators).
}
{
  N\_Vector N\_VNew\_ManyVector(sunindextype num\_subvectors,
  \newlinefill{N\_Vector N\_VNew\_ManyVector}
  N\_Vector *vec\_array);
}
%%--------------------------------------
\sunmodfun{N\_VMake\_ManyVector}
{
  This function creates a ManyVector from a set of existing {\nvector}
  objects, and a user-created MPI communicator that ``connects'' these
  subvectors.  Any MPI-aware subvectors may use different MPI
  communicators than the input \id{comm}.  We note that this routine
  is designed to support any combination of the use cases above.

  The input \id{comm} should be the memory reference to this
  user-created MPI communicator.  We note that since many {\mpi}
  implementations \id{\#define} \id{MPI\_COMM\_WORLD} to be a specific
  integer \emph{value} (that has no memory reference), users who wish
  to supply \id{MPI\_COMM\_WORLD} to this routine should first
  duplicate this to a specific \id{MPI\_Comm} variable before passing
  in the reference, e.g.

  \hspace{0.5in} \texttt{MPI\_Comm comm;}\vspace{-0.5em}
  
  \hspace{0.5in} \texttt{N\_Vector x;}\vspace{-0.5em}
  
  \hspace{0.5in} \texttt{MPI\_Comm\_dup(MPI\_COMM\_WORLD, \&comm);}\vspace{-0.5em}
  
  \hspace{0.5in} \texttt{x = N\_VMake\_ManyVector(\&comm, ...);}

  This routine will internally call \id{MPI\_Comm\_dup} to create a
  copy of the input \id{comm}, so the user-supplied \id{comm} argument
  need not be retained after the call to \id{N\_VMake\_ManyVector}.

  If all subvectors are MPI-unaware, then the input \id{comm} argument
  should be \id{NULL}, although in this case, it would be simpler to
  call \Id{N\_VNew\_ManyVector} instead.
  
  This routine will copy all \id{N\_Vector} pointers from the input
  \id{vec\_array}, so the user may modify/free that pointer array
  after calling this function.  However, this routine does \emph{not}
  allocate any new subvectors, so the underlying {\nvector} objects
  themselves should not be destroyed before the ManyVector that
  contains them.

  Upon successful completion, the new ManyVector is returned;
  otherwise this routine returns \id{NULL} (e.g., if the input
  \id{vec\_array} is \id{NULL}).
}
{
  N\_Vector N\_VMake\_ManyVector(void *comm, 
  sunindextype num\_subvectors,
  \newlinefill{N\_Vector N\_VMake\_ManyVector}
  N\_Vector *vec\_array);
}
%%--------------------------------------
\sunmodfun{N\_VGetSubvector\_ManyVector}
{
  This function returns the \id{vec\_num} subvector from the {\nvector}
   array.
}
{
  N\_Vector N\_VGetSubvector\_ManyVector(N\_Vector v, sunindextype vec\_num);
}
%%--------------------------------------
\sunmodfun{N\_VGetNumSubvectors\_ManyVector}
{
  This function returns the overall number of subvectors in the
  ManyVector object.
}
{
  sunindextype N\_VGetNumSubvectors\_ManyVector(N\_Vector v);
}
%%--------------------------------------
By default all fused and vector array operations are disabled in the {\nvecmanyvector}
module, except for \id{N\_VWrmsNormVectorArray} and
\id{N\_VWrmsNormMaskVectorArray}, that are enabled by default. The
following additional user-callable routines are provided to enable or
disable fused and vector array operations for a specific vector. To
ensure consistency across vectors it is recommended to first create a
vector with \id{N\_VNew\_ManyVector} or \id{N\_VMake\_ManyVector},
enable/disable the desired operations for that vector with the
functions below, and create any additional vectors from that vector
using \id{N\_VClone}. This guarantees that the new vectors will have
the same operations enabled/disabled, since cloned vectors inherit
those configuration options from the vector they are cloned from, while
vectors created with \id{N\_VNew\_ManyVector} and
\id{N\_VMake\_ManyVector} will have the default settings for the
{\nvecmanyvector} module.  We note that these routines \emph{do not} 
call the corresponding routines on subvectors, so those should be set up
as desired \emph{before} attaching them to the ManyVector in
\id{N\_VNew\_ManyVector} or \id{N\_VMake\_ManyVector}.
%%--------------------------------------
\sunmodfun{N\_VEnableFusedOps\_ManyVector}
{
  This function enables (\id{SUNTRUE}) or disables (\id{SUNFALSE}) all fused and
  vector array operations in the ManyVector. The return value is \id{0} for
  success and \id{-1} if the input vector or its \id{ops} structure are \id{NULL}.
}
{
  int N\_VEnableFusedOps\_ManyVector(N\_Vector v, booleantype tf);
}
%%--------------------------------------
\sunmodfun{N\_VEnableLinearCombination\_ManyVector}
{
  This function enables (\id{SUNTRUE}) or disables (\id{SUNFALSE}) the linear
  combination fused operation in the ManyVector. The return value is \id{0} for
  success and \id{-1} if the input vector or its \id{ops} structure are \id{NULL}.
}
{
  int N\_VEnableLinearCombination\_ManyVector(N\_Vector v, booleantype tf);
}
%%--------------------------------------
\sunmodfun{N\_VEnableScaleAddMulti\_ManyVector}
{
  This function enables (\id{SUNTRUE}) or disables (\id{SUNFALSE}) the scale and
  add a vector to multiple vectors fused operation in the ManyVector. The
  return value is \id{0} for success and \id{-1} if the input vector or its
  \id{ops} structure are \id{NULL}.
}
{
  int N\_VEnableScaleAddMulti\_ManyVector(N\_Vector v, booleantype tf);
}
%%--------------------------------------
\sunmodfun{N\_VEnableDotProdMulti\_ManyVector}
{
  This function enables (\id{SUNTRUE}) or disables (\id{SUNFALSE}) the multiple
  dot products fused operation in the ManyVector. The return value is \id{0}
  for success and \id{-1} if the input vector or its \id{ops} structure are
  \id{NULL}.
}
{
  int N\_VEnableDotProdMulti\_ManyVector(N\_Vector v, booleantype tf);
}
%%--------------------------------------
\sunmodfun{N\_VEnableLinearSumVectorArray\_ManyVector}
{
  This function enables (\id{SUNTRUE}) or disables (\id{SUNFALSE}) the linear sum
  operation for vector arrays in the ManyVector. The return value is \id{0} for
  success and \id{-1} if the input vector or its \id{ops} structure are \id{NULL}.
}
{
  int N\_VEnableLinearSumVectorArray\_ManyVector(N\_Vector v, booleantype tf);
}
%%--------------------------------------
\sunmodfun{N\_VEnableScaleVectorArray\_ManyVector}
{
  This function enables (\id{SUNTRUE}) or disables (\id{SUNFALSE}) the scale
  operation for vector arrays in the ManyVector. The return value is \id{0} for
  success and \id{-1} if the input vector or its \id{ops} structure are \id{NULL}.
}
{
  int N\_VEnableScaleVectorArray\_ManyVector(N\_Vector v, booleantype tf);
}
%%--------------------------------------
\sunmodfun{N\_VEnableConstVectorArray\_ManyVector}
{
  This function enables (\id{SUNTRUE}) or disables (\id{SUNFALSE}) the const
  operation for vector arrays in the ManyVector. The return value is \id{0} for
  success and \id{-1} if the input vector or its \id{ops} structure are \id{NULL}.
}
{
  int N\_VEnableConstVectorArray\_ManyVector(N\_Vector v, booleantype tf);
}
%%--------------------------------------
\sunmodfun{N\_VEnableWrmsNormVectorArray\_ManyVector}
{
  This function enables (\id{SUNTRUE}) or disables (\id{SUNFALSE}) the WRMS norm
  operation for vector arrays in the ManyVector. The return value is \id{0} for
  success and \id{-1} if the input vector or its \id{ops} structure are \id{NULL}.
}
{
  int N\_VEnableWrmsNormVectorArray\_ManyVector(N\_Vector v, booleantype tf);
}
%%--------------------------------------
\sunmodfun{N\_VEnableWrmsNormMaskVectorArray\_ManyVector}
{
  This function enables (\id{SUNTRUE}) or disables (\id{SUNFALSE}) the masked WRMS
  norm operation for vector arrays in the ManyVector. The return value is
  \id{0} for success and \id{-1} if the input vector or its \id{ops} structure are
  \id{NULL}.
}
{
  int N\_VEnableWrmsNormMaskVectorArray\_ManyVector(N\_Vector v, booleantype tf);
}
%%
%%------------------------------------
%%
\paragraph{\bf Notes} 
           
\begin{itemize}
                                        
\item
  {\warn}\id{N\_VNew\_ManyVector} and \id{N\_VMake\_ManyVector} set
  the field {\em own\_data} $=$ \id{SUNFALSE}.  \\
  \id{N\_VDestroy\_ManyVector} will not attempt to call
  \id{N\_VDestroy} on any subvectors contained in the subvector array
  for any \id{N\_Vector} with {\em own\_data} set to \id{SUNFALSE}. In
  such a case, it is the user's responsibility to deallocate the
  subvectors.

\item
  {\warn}To maximize efficiency, arithmetic vector operations in the
  {\nvecmanyvector} implementation that have more than one
  \id{N\_Vector} argument do not check for consistent internal
  representation of these vectors. It is the user's responsibility to
  ensure that such routines are called with \id{N\_Vector} arguments
  that were all created with the same subvector representations.

\end{itemize}


\section{NVECTOR Examples}\label{ss:nvec_examples}

There are \id{NVector} examples that may be installed for the
implementations provided with {\sundials}. Each
implementation makes use of the functions in \id{test\_nvector.c}.
These example functions show simple usage of the \id{NVector} family
of functions. The input to the examples are the vector length, number
of threads (if threaded implementation), and a print timing flag.

\noindent The following is a list of the example functions in \id{test\_nvector.c}:
\begin{itemize}
\item \id{Test\_N\_VClone}: Creates clone of vector and checks validity of clone.
\item \id{Test\_N\_VCloneEmpty}: Creates clone of empty vector and checks validity of clone.
\item \id{Test\_N\_VCloneVectorArray}: Creates clone of vector array and checks validity of cloned array.
\item \id{Test\_N\_VCloneVectorArray}: Creates clone of empty vector array and checks validity of cloned array.
\item \id{Test\_N\_VGetArrayPointer}: Get array pointer.
\item \id{Test\_N\_VSetArrayPointer}: Allocate new vector, set pointer to new vector array, and check values.
\item \id{Test\_N\_VGetLength}: Compares self-reported length to calculated length.
\item \id{Test\_N\_VGetCommunicator}: Compares self-reported communicator to the one used in constructor; or for MPI-unaware vectors it ensures that NULL is reported.
\item \id{Test\_N\_VLinearSum} Case 1a: Test y =  x + y
\item \id{Test\_N\_VLinearSum} Case 1b: Test y = -x + y
\item \id{Test\_N\_VLinearSum} Case 1c: Test y = ax + y
\item \id{Test\_N\_VLinearSum} Case 2a: Test x =  x + y
\item \id{Test\_N\_VLinearSum} Case 2b: Test x =  x - y
\item \id{Test\_N\_VLinearSum} Case 2c: Test x =  x + by
\item \id{Test\_N\_VLinearSum} Case 3:  Test z =  x + y
\item \id{Test\_N\_VLinearSum} Case 4a: Test z =  x - y
\item \id{Test\_N\_VLinearSum} Case 4b: Test z = -x + y
\item \id{Test\_N\_VLinearSum} Case 5a: Test z =  x + by
\item \id{Test\_N\_VLinearSum} Case 5b: Test z = ax + y
\item \id{Test\_N\_VLinearSum} Case 6a: Test z = -x + by
\item \id{Test\_N\_VLinearSum} Case 6b: Test z = ax - y
\item \id{Test\_N\_VLinearSum} Case 7:  Test z = a(x + y)
\item \id{Test\_N\_VLinearSum} Case 8:  Test z = a(x - y)
\item \id{Test\_N\_VLinearSum} Case 9:  Test z = ax + by
\item \id{Test\_N\_VConst}: Fill vector with constant and check result.
\item \id{Test\_N\_VProd}: Test vector multiply: z = x * y
\item \id{Test\_N\_VDiv}: Test vector division: z = x / y
\item \id{Test\_N\_VScale}: Case 1: scale: x = cx
\item \id{Test\_N\_VScale}: Case 2: copy: z = x
\item \id{Test\_N\_VScale}: Case 3: negate: z = -x
\item \id{Test\_N\_VScale}: Case 4: combination: z = cx
\item \id{Test\_N\_VAbs}: Create absolute value of vector.
\item \id{Test\_N\_VAddConst}: add constant vector: z = c + x
\item \id{Test\_N\_VDotProd}: Calculate dot product of two vectors.
\item \id{Test\_N\_VMaxNorm}: Create vector with known values, find and validate the max norm.
\item \id{Test\_N\_VWrmsNorm}: Create vector of known values, find and validate the weighted root mean square.
\item \id{Test\_N\_VWrmsNormMask}: Create vector of known values, find and validate the weighted root mean square using all elements except one.
\item \id{Test\_N\_VMin}: Create vector, find and validate the min.
\item \id{Test\_N\_VWL2Norm}: Create vector, find and validate the weighted Euclidean L2 norm.
\item \id{Test\_N\_VL1Norm}: Create vector, find and validate the L1 norm.
\item \id{Test\_N\_VCompare}: Compare vector with constant returning and validating comparison vector.
\item \id{Test\_N\_VInvTest}: Test z[i] = 1 / x[i]
\item \id{Test\_N\_VConstrMask}: Test mask of vector x with vector c.
\item \id{Test\_N\_VMinQuotient}: Fill two vectors with known values. Calculate and validate minimum quotient.
\item \id{Test\_N\_VLinearCombination} Case 1a: Test x = a x
\item \id{Test\_N\_VLinearCombination} Case 1b: Test z = a x
\item \id{Test\_N\_VLinearCombination} Case 2a: Test x = a x + b y
\item \id{Test\_N\_VLinearCombination} Case 2b: Test z = a x + b y
\item \id{Test\_N\_VLinearCombination} Case 3a: Test x = x + a y + b z
\item \id{Test\_N\_VLinearCombination} Case 3b: Test x = a x + b y + c z
\item \id{Test\_N\_VLinearCombination} Case 3c: Test w = a x + b y + c z
\item \id{Test\_N\_VScaleAddMulti} Case 1a: y = a x + y
\item \id{Test\_N\_VScaleAddMulti} Case 1b: z = a x + y
\item \id{Test\_N\_VScaleAddMulti} Case 2a: Y[i] = c[i] x + Y[i], i = 1,2,3
\item \id{Test\_N\_VScaleAddMulti} Case 2b: Z[i] = c[i] x + Y[i], i = 1,2,3
\item \id{Test\_N\_VDotProdMulti} Case 1: Calculate the dot product of two vectors
\item \id{Test\_N\_VDotProdMulti} Case 2: Calculate the dot product of one vector with three other vectors in a vector array.
\item \id{Test\_N\_VLinearSumVectorArray} Case 1: z = a x + b y
\item \id{Test\_N\_VLinearSumVectorArray} Case 2a: Z[i] = a X[i] + b Y[i]
\item \id{Test\_N\_VLinearSumVectorArray} Case 2b: X[i] = a X[i] + b Y[i]
\item \id{Test\_N\_VLinearSumVectorArray} Case 2c: Y[i] = a X[i] + b Y[i]
\item \id{Test\_N\_VScaleVectorArray} Case 1a: y = c y
\item \id{Test\_N\_VScaleVectorArray} Case 1b: z = c y
\item \id{Test\_N\_VScaleVectorArray} Case 2a: Y[i] = c[i] Y[i]
\item \id{Test\_N\_VScaleVectorArray} Case 2b: Z[i] = c[i] Y[i]
\item \id{Test\_N\_VScaleVectorArray} Case 1a: z = c
\item \id{Test\_N\_VScaleVectorArray} Case 1b: Z[i] = c
\item \id{Test\_N\_VWrmsNormVectorArray} Case 1a: Create a vector of know values, find and validate the weighted root mean square norm.
\item \id{Test\_N\_VWrmsNormVectorArray} Case 1b: Create a vector array of three vectors of know values, find and validate the weighted root mean square norm of each.
\item \id{Test\_N\_VWrmsNormMaskVectorArray} Case 1a: Create a vector of know values, find and validate the weighted root mean square norm using all elements except one.
\item \id{Test\_N\_VWrmsNormMaskVectorArray} Case 1b: Create a vector array of three vectors of know values, find and validate the weighted root mean square norm of each using all elements except one.
\item \id{Test\_N\_VScaleAddMultiVectorArray} Case 1a: y = a x + y
\item \id{Test\_N\_VScaleAddMultiVectorArray} Case 1b: z = a x + y
\item \id{Test\_N\_VScaleAddMultiVectorArray} Case 2a: Y[j][0] = a[j] X[0] + Y[j][0]
\item \id{Test\_N\_VScaleAddMultiVectorArray} Case 2b: Z[j][0] = a[j] X[0] + Y[j][0]
\item \id{Test\_N\_VScaleAddMultiVectorArray} Case 3a: Y[0][i] = a[0] X[i] + Y[0][i]
\item \id{Test\_N\_VScaleAddMultiVectorArray} Case 3b: Z[0][i] = a[0] X[i] + Y[0][i]
\item \id{Test\_N\_VScaleAddMultiVectorArray} Case 4a: Y[j][i] = a[j] X[i] + Y[j][i]
\item \id{Test\_N\_VScaleAddMultiVectorArray} Case 4b: Z[j][i] = a[j] X[i] + Y[j][i]
\item \id{Test\_N\_VLinearCombinationVectorArray} Case 1a: x = a x
\item \id{Test\_N\_VLinearCombinationVectorArray} Case 1b: z = a x
\item \id{Test\_N\_VLinearCombinationVectorArray} Case 2a: x = a x + b y
\item \id{Test\_N\_VLinearCombinationVectorArray} Case 2b: z = a x + b y
\item \id{Test\_N\_VLinearCombinationVectorArray} Case 3a: x = a x + b y + c z
\item \id{Test\_N\_VLinearCombinationVectorArray} Case 3b: w = a x + b y + c z
\item \id{Test\_N\_VLinearCombinationVectorArray} Case 4a: X[0][i] = c[0] X[0][i]
\item \id{Test\_N\_VLinearCombinationVectorArray} Case 4b: Z[i] = c[0] X[0][i]
\item \id{Test\_N\_VLinearCombinationVectorArray} Case 5a: X[0][i] = c[0] X[0][i] + c[1] X[1][i]
\item \id{Test\_N\_VLinearCombinationVectorArray} Case 5b: Z[i] = c[0] X[0][i] + c[1] X[1][i]
\item \id{Test\_N\_VLinearCombinationVectorArray} Case 6a: X[0][i] = X[0][i] + c[1] X[1][i] + c[2] X[2][i]
\item \id{Test\_N\_VLinearCombinationVectorArray} Case 6b: X[0][i] = c[0] X[0][i] + c[1] X[1][i] + c[2] X[2][i]
\item \id{Test\_N\_VLinearCombinationVectorArray} Case 6c: Z[i] = c[0] X[0][i] + c[1] X[1][i] + c[2] X[2][i]
\item \id{Test\_N\_VDotProdLocal}: Calculate MPI task-local portion of the dot product of two vectors.
\item \id{Test\_N\_VMaxNormLocal}: Create vector with known values, find and validate the MPI task-local portion of the max norm.
\item \id{Test\_N\_VMinLocal}: Create vector, find and validate the MPI task-local min.
\item \id{Test\_N\_VL1NormLocal}: Create vector, find and validate the MPI task-local portion of the L1 norm.
\item \id{Test\_N\_VWSqrSumLocal}: Create vector of known values, find and validate the MPI task-local portion of the weighted squared sum of two vectors.
\item \id{Test\_N\_VWSqrSumMaskLocal}: Create vector of known values, find and validate the MPI task-local portion of the weighted squared sum of two vectors, using all elements except one.
\item \id{Test\_N\_VInvTestLocal}: Test the MPI task-local portion of z[i] = 1 / x[i]
\item \id{Test\_N\_VConstrMaskLocal}: Test the MPI task-local portion of the mask of vector x with vector c.
\item \id{Test\_N\_VMinQuotientLocal}: Fill two vectors with known values. Calculate and validate the MPI task-local minimum quotient.
\end{itemize}

