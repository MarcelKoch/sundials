To support applications which leverage memory pools, or utilize a memory
abstraction layer, {\sundials} provides a set of utilities we will
collectively refer to as the \id{SUNMemoryHelper} API. The goal of this
API is to allow users to leverage operations defined by native {\sundials} data
structures while allowing the user to have finer-grained control of the memory
management.

% ====================================================================
\section{The SUNMemoryHelper API}\label{s:sunmemory}
% ====================================================================

This API consists of three new {\sundials} types: \id{SUNMemoryType},
\id{SUNMemory}, and \id{SUNMemoryHelper}, which we now define.

The \ID{SUNMemory} structure wraps a pointer to actual data. This structure
is defined as
\begin{verbatim}
typedef struct _SUNMemory
{
  void*         ptr;
  SUNMemoryType type;
  booleantype   own;
} *SUNMemory;
\end{verbatim}

The \ID{SUNMemoryType} type is an enumeration that defines the four supported
memory types:
\begin{verbatim}
typedef enum
{
  SUNMEMTYPE_HOST,      /* pageable memory accessible on the host     */
  SUNMEMTYPE_PINNED,    /* page-locked memory accesible on the host   */
  SUNMEMTYPE_DEVICE,    /* memory accessible from the device          */
  SUNMEMTYPE_UVM        /* memory accessible from the host or device  */
} SUNMemoryType;
\end{verbatim}

Finally, the \ID{SUNMemoryHelper} structure is defined as
\begin{verbatim}
struct _SUNMemoryHelper
{
  void*               content;
  SUNMemoryHelper_Ops ops;
} *SUNMemoryHelper;
\end{verbatim}
where \id{SUNMemoryHelper\_Ops} is defined as
\begin{verbatim}
typedef struct _SUNMemoryHelper_Ops
{
  /* operations that implementations are required to provide */
  int             (*alloc)(SUNMemoryHelper, SUNMemory* memptr, size_t mem_size, SUNMemoryType mem_type);
  int             (*dealloc)(SUNMemoryHelper, SUNMemory mem);
  int             (*copy)(SUNMemoryHelper, SUNMemory dst, SUNMemory src, size_t mem_size);

  /* operations that provide default implementations */
  int             (*copyasync)(SUNMemoryHelper, SUNMemory dst, SUNMemory src,
                                size_t mem_size, void* ctx);
  SUNMemoryHelper (*clone)(SUNMemoryHelper);
  int             (*destroy)(SUNMemoryHelper);
} *SUNMemoryHelper_Ops;
\end{verbatim}


%
%
\subsection{Implementation defined operations}\label{ss:sunmemory_ops}

The \id{SUNMemory} API also defines the following operations which do
require a \id{SUNMemoryHelper} instance and \textbf{require} the implementation
to define them:

\ucfunction{SUNMemoryHelper\_Alloc}
{
  retval = SUNMemoryHelper\_Alloc(helper, *memptr, mem\_size, mem\_type);
}
{
  Allocates a \id{SUNMemory} object whose \id{ptr} field is allocated for
  \id{mem\_size} bytes and is of type \id{mem\_type}. The new object will
  have ownership of \id{ptr} and will be deallocated when
  \id{SUNMemoryHelper\_Dealloc} is called.
}
{
  \begin{args}[mem\_type]
  \item[helper] (\id{SUNMemoryHelper}) the \id{SUNMemoryHelper} object
  \item[memptr] (\id{SUNMemory*}) pointer to the allocated \id{SUNMemory}
  \item[mem\_size] (\id{size\_t}) the size in bytes of the \id{ptr}
  \item[mem\_type] (\id{SUNMemoryType}) the \id{SUNMemoryType} of the \id{ptr}
  \end{args}
}
{
  An \id{int} flag indicating success (zero) or failure (non-zero).
}
{}

\ucfunction{SUNMemoryHelper\_Dealloc}
{
  retval = SUNMemoryHelper\_Dealloc(helper, mem);
}
{
  Deallocates the \id{mem->ptr} field if it is owned by \id{mem}, and then
  deallocates the \id{mem} object.
}
{
  \begin{args}[helper]
  \item[helper] (\id{SUNMemoryHelper}) the \id{SUNMemoryHelper} object
  \item[mem] (\id{SUNMemory}) the \id{SUNMemory} object
  \end{args}
}
{
  An \id{int} flag indicating success (zero) or failure (non-zero).
}
{}

\ucfunction{SUNMemoryHelper\_Copy}
{
  retval = SUNMemoryHelper\_Copy(helper, dst, src, mem\_size);
}
{
  Synchronously copies \id{mem\_size} bytes from the the source memory to the
  destination memory.  The copy can be across memory spaces, e.g. host to
  device, or within a memory space, e.g. host to host.  The \id{helper}
  object should use the memory types of \id{dst} and \id{src} to determine
  the appropriate transfer type necessary.
}
{
  \begin{args}[mem\_size]
  \item[helper] (\id{SUNMemoryHelper}) the \id{SUNMemoryHelper} object
  \item[dst] (\id{SUNMemory}) the destination memory to copy to
  \item[src] (\id{SUNMemory}) the source memory to copy from
  \item[mem\_size] (\id{size\_t}) the number of bytes to copy
  \end{args}
}
{
  An \id{int} flag indicating success (zero) or failure (non-zero).
}
{}


%
%
\subsection{Utility Functions}

The \id{SUNMemoryHelper} API defines the following functions which do not
require a \id{SUNMemoryHelper} instance:

\ucfunction{SUNMemoryHelper\_Alias}
{
  mem2 = SUNMemoryHelper\_Alias(mem1);
}
{
  Returns a \id{SUNMemory} object whose \id{ptr} field points to the same address
  as \id{mem1}. The new object \textit{will not} have
  ownership of \id{ptr}, therefore, it will not free \id{ptr} when
  \id{SUNMemoryHelper\_Dealloc} is called.
}
{
  \begin{args}[mem1]
    \item[mem1] (\id{SUNMemory}) a \id{SUNMemory} object
  \end{args}
}
{
  A \id{SUNMemory} object.
}
{}

\ucfunction{SUNMemoryHelper\_Wrap}
{
  mem = SUNMemoryHelper\_Wrap(ptr, mem\_type);
}
{
  Returns a \id{SUNMemory} object whose \id{ptr} field points to the \id{ptr}
  argument passed to the function. The new object \textit{will not} have
  ownership of \id{ptr}, therefore, it will not free \id{ptr} when
  \id{SUNMemoryHelper\_Dealloc} is called.
}
{
  \begin{args}[mem\_type]
  \item[ptr] (\id{SUNMemoryType}) the data pointer to wrap in a \id{SUNMemory} object
  \item[mem\_type] (\id{SUNMemoryType}) the \id{SUNMemoryType} of the \id{ptr}
  \end{args}
}
{
  A \id{SUNMemory} object.
}
{}

\ucfunction{SUNMemoryHelper\_NewEmpty}
{
  helper = SUNMemoryHelper\_NewEmpty();
}
{
  Returns an empty \id{SUNMemoryHelper}.
  This is useful for building custom \id{SUNMemoryHelper} implementations.
}
{
}
{
  A \id{SUNMemoryHelper} object.
}
{}

\ucfunction{SUNMemoryHelper\_CopyOps}
{
  retval = SUNMemoryHelper\_CopyOps(src, dst);
}
{
  Copies the \id{ops} field of \id{src} to the \id{ops} field of \id{dst}.
  This is useful for building custom \id{SUNMemoryHelper} implementations.
}
{
  \begin{args}[dst]
  \item[src] (\id{SUNMemoryHelper}) the object to copy from
  \item[dst] (\id{SUNMemoryHelper}) the object to copy to
  \end{args}
}
{
  An \id{int} flag indicating success (zero) or failure (non-zero).
}
{}


%
%
\subsection{Implementation overridable operations with defaults}\label{ss:sunmemory_overridable_ops}

In addition, the \id{SUNMemoryHelper} API defines the following \textit{optionally
overridable} operations which do require a \id{SUNMemoryHelper} instance:

\ucfunction{SUNMemoryHelper\_CopyAsync}
{
  retval = SUNMemoryHelper\_CopyAsync(helper, dst, src, mem\_size, ctx);
}
{
  Asynchronously copies \id{mem\_size} bytes from the the source memory to the
  destination memory.  The copy can be across memory spaces, e.g. host to
  device, or within a memory space, e.g. host to host.  The \id{helper} object
  should use the memory types of \id{dst} and \id{src} to determine the
  appropriate transfer type necessary.  The \id{ctx} argument is used when a
  different execution ``stream'' needs to be provided to perform the copy in,
  e.g. with \id{CUDA} this would be a \id{cudaStream\_t}.
}
{
  \begin{args}[mem\_size]
  \item[helper] (\id{SUNMemoryHelper}) the \id{SUNMemoryHelper} object
  \item[dst] (\id{SUNMemory}) the destination memory to copy to
  \item[src] (\id{SUNMemory}) the source memory to copy from
  \item[mem\_size] (\id{size\_t}) the number of bytes to copy
  \item[ctx] (\id{void *}) typically a handle for an object representing an
  alternate execution stream, but it can be any implementation specific data
  \end{args}
}
{
  An \id{int} flag indicating success (zero) or failure (non-zero).
}
{
  \warn If this operation is not defined by the implementation, then
  \id{SUNMemoryHelper\_Copy} will be used.
}

\ucfunction{SUNMemoryHelper\_Clone}
{
  helper2 = SUNMemoryHelper\_Clone(helper);
}
{
  Clones the \id{SUNMemoryHelper} object itself.
}
{
  \begin{args}[helper]
  \item[helper] (\id{SUNMemoryHelper}) the \id{SUNMemoryHelper} object to clone
  \end{args}
}
{
  A \id{SUNMemoryHelper} object.
}
{
  \warn If this operation is not defined by the implementation, then the default
  clone will only copy the \id{SUNMemoryHelper\_Ops} structure stored in
  \id{helper->ops}, and not the \id{helper->content} field.
}

\ucfunction{SUNMemoryHelper\_Destroy}
{
  retval = SUNMemoryHelper\_Destroy(helper);
}
{
  Destroys (frees) the \id{SUNMemoryHelper} object itself.
}
{
  \begin{args}[helper]
  \item[helper] (\id{SUNMemoryHelper}) the \id{SUNMemoryHelper} object to destroy
  \end{args}
}
{
  An \id{int} flag indicating success (zero) or failure (non-zero).
}
{
  \warn If this operation is not defined by the implementation, then the default
  destroy will only free the \id{helper->ops} field and the \id{helper} itself.
  The \id{helper->content} field will not be freed.
}


%
%
\subsection{Implementing a custom \id{SUNMemoryHelper}}\label{ss:sunmemory_custom}

A particular implementation of the \id{SUNMemoryHelper} API must:
\begin{itemize}
\item Define and implement the required operations.
  Note that the names of these routines should be unique to that implementation
  in order to permit using more than one \id{SUNMemoryHelper} module in the
  same code.
\item Optionally, specify the {\em content} field of \id{SUNMemoryHelper}.
\item Optionally, define and implement additional user-callable routines
  acting on the newly defined \id{SUNMemoryHelper}.
\end{itemize}
An example of a custom \id{SUNMemoryHelper} is given in
\id{examples/utilities/custom\_memory\_helper.h}.
