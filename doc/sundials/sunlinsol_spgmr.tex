% ====================================================================
\section{The SUNLinearSolver\_SPGMR implementation}
\label{ss:sunlinsol_spgmr}
% ====================================================================

This section describes the {\sunlinsol} implementation of the {\spgmr}
(Scaled, Preconditioned, Generalized Minimum Residual \cite{SaSc:86})
iterative linear solver. The {\sunlinsolspgmr} module is designed to be
compatible with any {\nvector} implementation that supports a minimal subset
of operations (\id{N\_VClone}, \id{N\_VDotProd}, \id{N\_VScale},
\id{N\_VLinearSum}, \id{N\_VProd}, \id{N\_VConst}, \id{N\_VDiv}, and
\id{N\_VDestroy}). When using Classical Gram-Schmidt, the optional function
\id{N\_VDotProdMulti} may be supplied for increased efficiency.

To access the {\sunlinsolspgmr} module, include the header file
\id{sunlinsol/sunlinsol\_spgmr.h}. We note that the {\sunlinsolspgmr} module is
accessible from {\sundials} packages \textit{without} separately linking to
the \id{libsundials\_sunlinsolspgmr} module library.


% ====================================================================
\subsection{SUNLinearSolver\_SPGMR description}
\label{ss:sunlinsol_spgmr_description}
% ====================================================================

This solver is constructed to perform the following operations:
\begin{itemize}
\item During construction, the \id{xcor} and \id{vtemp} arrays are
  cloned from a template {\nvector} that is input, and default solver
  parameters are set.
\item User-facing ``set'' routines may be called to modify default
  solver parameters.
\item Additional ``set'' routines are called by the {\sundials} solver
  that interfaces with {\sunlinsolspgmr} to supply the
  \id{ATimes}, \id{PSetup}, and \id{Psolve} function pointers and
  \id{s1} and \id{s2} scaling vectors.
\item In the ``initialize'' call, the remaining solver data is
  allocated (\id{V}, \id{Hes}, \id{givens}, and \id{yg} )
\item In the ``setup'' call, any non-\id{NULL}
  \id{PSetup} function is called.  Typically, this is provided by
  the {\sundials} solver itself, that translates between the
  generic \id{PSetup} function and the
  solver-specific routine (solver-supplied or user-supplied).
\item In the ``solve'' call, the GMRES iteration is performed.  This
  will include scaling, preconditioning, and restarts if those options
  have been supplied.
\end{itemize}


% ====================================================================
\subsection{SUNLinearSolver\_SPGMR functions}
\label{ss:sunlinsol_spgmr_functions}
% ====================================================================

The {\sunlinsolspgmr} module provides the following user-callable constructor
for creating a \newline \id{SUNLinearSolver} object.
%
% --------------------------------------------------------------------
%
\ucfunctiondf{SUNLinSol\_SPGMR}
{
  LS = SUNLinSol\_SPGMR(y, pretype, maxl);
}
{
  The function \ID{SUNLinSol\_SPGMR} creates and allocates memory for
  a {\spgmr} \newline \id{SUNLinearSolver} object.
}
{
  \begin{args}[pretype]
  \item[y] (\id{N\_Vector})
    a template for cloning vectors needed within the solver
  \item[pretype] (\id{int})
    flag indicating the desired type of preconditioning, allowed
    values are:
    \begin{itemize}
    \item \id{PREC\_NONE} (0)
    \item \id{PREC\_LEFT} (1)
    \item \id{PREC\_RIGHT} (2)
    \item \id{PREC\_BOTH} (3)
    \end{itemize}
    Any other integer input will result in the default (no
    preconditioning).
  \item[maxl] (\id{int})
    the number of Krylov basis vectors to use. Values $\le0$ will
    result in the default value (5).
  \end{args}
}
{
  This returns a \id{SUNLinearSolver} object.  If either \id{y} is
  incompatible then this routine will return \id{NULL}.
}
{
  This routine will perform consistency checks to ensure that it is
  called with a consistent {\nvector} implementation (i.e.~that it
  supplies the requisite vector operations).  If \id{y} is
  incompatible, then this routine will return \id{NULL}.

  We note that some {\sundials} solvers are designed to only work
  with left preconditioning ({\ida} and {\idas}) and others with only
  right preconditioning ({\kinsol}). While it is possible to configure
  a {\sunlinsolspgmr} object to use any of the preconditioning options
  with these solvers, this use mode is not supported and may result in
  inferior performance.
}
{SUNSPGMR}
%
% --------------------------------------------------------------------
%
The {\sunlinsolspgmr} module defines implementations of all
``iterative'' linear solver operations listed in Sections
\ref{ss:sunlinsol_CoreFn} -- \ref{ss:sunlinsol_GetFn}:
\begin{itemize}
\item \id{SUNLinSolGetType\_SPGMR}
\item \id{SUNLinSolInitialize\_SPGMR}
\item \id{SUNLinSolSetATimes\_SPGMR}
\item \id{SUNLinSolSetPreconditioner\_SPGMR}
\item \id{SUNLinSolSetScalingVectors\_SPGMR}
\item \id{SUNLinSolSetZeroGuess\_SPGMR} -- note the solver assumes a non-zero
  guess by default and the zero guess flag is reset to \id{SUNFALSE} after each
  call to \id{SUNLinSolSolve\_SPGMR}.
\item \id{SUNLinSolSetup\_SPGMR}
\item \id{SUNLinSolSolve\_SPGMR}
\item \id{SUNLinSolNumIters\_SPGMR}
\item \id{SUNLinSolResNorm\_SPGMR}
\item \id{SUNLinSolResid\_SPGMR}
\item \id{SUNLinSolLastFlag\_SPGMR}
\item \id{SUNLinSolSpace\_SPGMR}
\item \id{SUNLinSolFree\_SPGMR}
\end{itemize}
All of the listed operations are callable via the {\F} 2003 interface module
by prepending an `F' to the function name.

The {\sunlinsolspgmr} module also defines the following additional
user-callable functions.
%
% --------------------------------------------------------------------
%
\ucfunctiondf{SUNLinSol\_SPGMRSetPrecType}
{
  retval = SUNLinSol\_SPGMRSetPrecType(LS, pretype);
}
{
  The function \ID{SUNLinSol\_SPGMRSetPrecType} updates the type of
  preconditioning to use in the {\sunlinsolspgmr} object.
}
{
  \begin{args}[pretype]
  \item[LS] (\id{SUNLinearSolver})
    the {\sunlinsolspgmr} object to update
  \item[pretype] (\id{int})
    flag indicating the desired type of preconditioning, allowed
    values match those discussed in \id{SUNLinSol\_SPGMR}.
  \end{args}
}
{
  This routine will return with one of the error codes
  \id{SUNLS\_ILL\_INPUT} (illegal \id{pretype}), \id{SUNLS\_MEM\_NULL}
  (\id{S} is \id{NULL}) or \id{SUNLS\_SUCCESS}.
}
{}
{SUNSPGMRSetPrecType}
%
% --------------------------------------------------------------------
%
\ucfunctiondf{SUNLinSol\_SPGMRSetGSType}
{
  retval = SUNLinSol\_SPGMRSetGSType(LS, gstype);
}
{
  The function \ID{SUNLinSol\_SPGMRSetPrecType} sets the type of
  Gram-Schmidt orthogonalization to use in the {\sunlinsolspgmr}
  object.
}
{
  \begin{args}[gstype]
  \item[LS] (\id{SUNLinearSolver})
    the {\sunlinsolspgmr} object to update
  \item[gstype] (\id{int})
    flag indicating the desired orthogonalization algorithm; allowed
    values are:
    \begin{itemize}
    \item \id{MODIFIED\_GS} (1)
    \item \id{CLASSICAL\_GS} (2)
    \end{itemize}
    Any other integer input will result in a
    failure, returning error code \newline \id{SUNLS\_ILL\_INPUT}.
  \end{args}
}
{
  This routine will return with one of the error codes
  \id{SUNLS\_ILL\_INPUT} (illegal \id{pretype}), \id{SUNLS\_MEM\_NULL}
  (\id{S} is \id{NULL}) or \id{SUNLS\_SUCCESS}.
}
{}
{SUNSPGMRSetGSType}
%
% --------------------------------------------------------------------
%
\ucfunctiondf{SUNLinSol\_SPGMRSetMaxRestarts}
{
  retval = SUNLinSol\_SPGMRSetMaxRestarts(LS, maxrs);
}
{
  The function \ID{SUNLinSol\_SPGMRSetMaxRestarts} sets the number of
  GMRES restarts to allow in the {\sunlinsolspgmr} object.
}
{
  \begin{args}[maxrs]
  \item[LS] (\id{SUNLinearSolver})
    the {\sunlinsolspgmr} object to update
  \item[maxrs] (\id{int})
    integer indicating number of restarts to allow.  A negative input
    will result in the default of 0.
  \end{args}
}
{
  This routine will return with one of the error codes
  \id{SUNLS\_MEM\_NULL} (\id{S} is \id{NULL}) or \id{SUNLS\_SUCCESS}.
}
{}
{SUNSPGMRSetMaxRestarts}
%
% --------------------------------------------------------------------
%
\ucfunctionf{SUNLinSolSetInfoFile\_SPGMR}
{
  retval = SUNLinSolSetInfoFile\_SPGMR(LS, info\_file);
}
{
  The function \ID{SUNLinSolSetInfoFile\_SPGMR} sets the
  output file where all informative (non-error) messages should be directed.
}
{
  \begin{args}[info\_file]
    \item[LS] (\id{SUNLinearSolver})
      a {\sunnonlinsol} object
    \item[info\_file] (\id{FILE*}) pointer to output file (\id{stdout} by default);
      a \id{NULL} input will disable output
  \end{args}
}
{
  The return value is
  \begin{itemize}
    \item \id{SUNLS\_SUCCESS} if successful
    \item \id{SUNLS\_MEM\_NULL} if the SUNLinearSolver memory was \id{NULL}
    \item \id{SUNLS\_ILL\_INPUT} if {\sundials} was not built with monitoring enabled
  \end{itemize}
}
{
  This function is intended for users that wish to monitor the linear
  solver progress. By default, the file pointer is set to \id{stdout}.

  \textbf{{\sundials} must be built with the CMake option
  \id{SUNDIALS\_BUILD\_WITH\_MONITORING}, to utilize this function.}
  See section \ref{ss:configuration_options_nix} for more information.
}
%
% --------------------------------------------------------------------
%
\ucfunctionf{SUNLinSolSetPrintLevel\_SPGMR}
{
  retval = SUNLinSolSetPrintLevel\_SPGMR(NLS, print\_level);
}
{
  The function \ID{SUNLinSolSetPrintLevel\_SPGMR} specifies the level
  of verbosity of the output.
}
{
  \begin{args}[print\_level]
  \item[LS] (\id{SUNLinearSolver})
    a {\sunnonlinsol} object
  \item[print\_level] (\id{int}) flag indicating level of verbosity;
    must be one of:
    \begin{itemize}
      \item 0, no information is printed (default)
      \item 1, for each linear iteration the residual norm is printed
    \end{itemize}
  \end{args}
}
{
  The return value is
  \begin{itemize}
    \item \id{SUNLS\_SUCCESS} if successful
    \item \id{SUNLS\_MEM\_NULL} if the SUNLinearSolver memory was \id{NULL}
    \item \id{SUNLS\_ILL\_INPUT} if {\sundials} was not built with monitoring enabled,
      or the print level value was invalid
  \end{itemize}
}
{
  This function is intended for users that wish to monitor the linear
  solver progress. By default, the print level is 0.

  \textbf{{\sundials} must be built with the CMake option
  \id{SUNDIALS\_BUILD\_WITH\_MONITORING}, to utilize this function.}
  See section \ref{ss:configuration_options_nix} for more information.
}


% ====================================================================
\subsection{SUNLinearSolver\_SPGMR Fortran interfaces}
\label{ss:sunlinsol_spgmr_fortran}
% ====================================================================

The {\sunlinsolspgmr} module provides a {\F} 2003 module as well as {\F} 77
style interface functions for use from {\F} applications.

\subsubsection*{FORTRAN 2003 interface module}
The \ID{fsunlinsol\_spgmr\_mod} {\F} module defines interfaces to all
{\sunlinsolspgmr} {\CC} functions using the intrinsic \id{iso\_c\_binding}
module which provides a standardized mechanism for interoperating with {\CC}. As
noted in the {\CC} function descriptions above, the interface functions are
named after the corresponding {\CC} function, but with a leading `F'. For
example, the function \id{SUNLinSol\_SPGMR} is interfaced as
\id{FSUNLinSol\_SPGMR}.

The {\F} 2003 {\sunlinsolspgmr} interface module can be accessed with the \id{use}
statement, i.e. \id{use fsunlinsol\_spgmr\_mod}, and linking to the library
\id{libsundials\_fsunlinsolspgmr\_mod}.{\em lib} in addition to the {\CC} library.
For details on where the library and module file
\id{fsunlinsol\_spgmr\_mod.mod} are installed see Appendix \ref{c:install}.
We note that the module is accessible from the {\F} 2003 {\sundials} integrators
\textit{without} separately linking to the
\id{libsundials\_fsunlinsolspgmr\_mod} library.

\subsubsection*{FORTRAN 77 interface functions}
For solvers that include a {\F} 77 interface module, the
{\sunlinsolspgmr} module also includes a Fortran-callable function
for creating a \id{SUNLinearSolver} object.
%
% --------------------------------------------------------------------
%
\ucfunction{FSUNSPGMRINIT}
{
  FSUNSPGMRINIT(code, pretype, maxl, ier)
}
{
  The function \ID{FSUNSPGMRINIT} can be called for Fortran programs
  to create a {\sunlinsolspgmr} object.
}
{
  \begin{args}[pretype]
  \item[code] (\id{int*})
    is an integer input specifying the solver id (1 for {\cvode}, 2
    for {\ida}, 3 for {\kinsol}, and 4 for {\arkode}).
  \item[pretype] (\id{int*})
    flag indicating desired preconditioning type
  \item[maxl] (\id{int*})
    flag indicating Krylov subspace size
  \end{args}
}
{
  \id{ier} is a return completion flag equal to \id{0} for a success
  return and \id{-1} otherwise. See printed message for details in case
  of failure.
}
{
  This routine must be called \emph{after} the {\nvector} object has
  been initialized.

  Allowable values for \id{pretype} and \id{maxl} are the same as for
  the {\CC} function \newline \ID{SUNLinSol\_SPGMR}.
}
Additionally, when using {\arkode} with a non-identity
mass matrix, the {\sunlinsolspgmr} module includes a Fortran-callable
function for creating a \id{SUNLinearSolver} mass matrix solver
object.
%
% --------------------------------------------------------------------
%
\ucfunction{FSUNMASSSPGMRINIT}
{
  FSUNMASSSPGMRINIT(pretype, maxl, ier)
}
{
  The function \ID{FSUNMASSSPGMRINIT} can be called for Fortran programs
  to create a {\sunlinsolspgmr} object for mass matrix linear systems.
}
{
  \begin{args}[pretype]
  \item[pretype] (\id{int*})
    flag indicating desired preconditioning type
  \item[maxl] (\id{int*})
    flag indicating Krylov subspace size
  \end{args}
}
{
  \id{ier} is a \id{int} return completion flag equal to \id{0} for a success
  return and \id{-1} otherwise. See printed message for details in case
  of failure.
}
{
  This routine must be called \emph{after} the {\nvector} object has
  been initialized.

  Allowable values for \id{pretype} and \id{maxl} are the same as for
  the {\CC} function \newline \ID{SUNLinSol\_SPGMR}.
}
%
% --------------------------------------------------------------------
%
The \id{SUNLinSol\_SPGMRSetPrecType}, \id{SUNLinSol\_SPGMRSetGSType}
and \newline \id{SUNLinSol\_SPGMRSetMaxRestarts} routines also
support Fortran interfaces for the system and mass matrix solvers.
%
% --------------------------------------------------------------------
%
\ucfunction{FSUNSPGMRSETGSTYPE}
{
  FSUNSPGMRSETGSTYPE(code, gstype, ier)
}
{
  The function \ID{FSUNSPGMRSETGSTYPE} can be called for Fortran
  programs to change the Gram-Schmidt orthogonaliation algorithm.
}
{
  \begin{args}[gstype]
  \item[code] (\id{int*})
    is an integer input specifying the solver id (1 for {\cvode}, 2
    for {\ida}, 3 for {\kinsol}, and 4 for {\arkode}).
  \item[gstype] (\id{int*})
    flag indicating the desired orthogonalization algorithm.
  \end{args}
}
{
  \id{ier} is a \id{int} return completion flag equal to \id{0} for a success
  return and \id{-1} otherwise. See printed message for details in case
  of failure.
}
{
  See \id{SUNLinSol\_SPGMRSetGSType} for complete further documentation of
  this routine.
}
%
% --------------------------------------------------------------------
%
\ucfunction{FSUNMASSSPGMRSETGSTYPE}
{
  FSUNMASSSPGMRSETGSTYPE(gstype, ier)
}
{
  The function \ID{FSUNMASSSPGMRSETGSTYPE} can be called for Fortran
  programs to change the Gram-Schmidt orthogonaliation algorithm for
  mass matrix linear systems.
}
{
  The arguments are identical to \id{FSUNSPGMRSETGSTYPE} above, except that
  \id{code} is not needed since mass matrix linear systems only arise
  in {\arkode}.
}
{
  \id{ier} is a \id{int} return completion flag equal to \id{0} for a success
  return and \id{-1} otherwise. See printed message for details in case
  of failure.
}
{
  See \id{SUNLinSol\_SPGMRSetGSType} for complete further documentation of
  this routine.
}
%
% --------------------------------------------------------------------
%
\ucfunction{FSUNSPGMRSETPRECTYPE}
{
  FSUNSPGMRSETPRECTYPE(code, pretype, ier)
}
{
  The function \ID{FSUNSPGMRSETPRECTYPE} can be called for Fortran
  programs to change the type of preconditioning to use.
}
{
  \begin{args}[pretype]
  \item[code] (\id{int*})
    is an integer input specifying the solver id (1 for {\cvode}, 2
    for {\ida}, 3 for {\kinsol}, and 4 for {\arkode}).
  \item[pretype] (\id{int*})
    flag indicating the type of preconditioning to use.
  \end{args}
}
{
  \id{ier} is a \id{int} return completion flag equal to \id{0} for a success
  return and \id{-1} otherwise. See printed message for details in case
  of failure.
}
{
  See \id{SUNLinSol\_SPGMRSetPrecType} for complete further documentation of
  this routine.
}
%
% --------------------------------------------------------------------
%
\ucfunction{FSUNMASSSPGMRSETPRECTYPE}
{
  FSUNMASSSPGMRSETPRECTYPE(pretype, ier)
}
{
  The function \ID{FSUNMASSSPGMRSETPRECTYPE} can be called for Fortran
  programs to change the type of preconditioning for mass matrix
  linear systems.
}
{
  The arguments are identical to \id{FSUNSPGMRSETPRECTYPE} above, except that
  \id{code} is not needed since mass matrix linear systems only arise
  in {\arkode}.
}
{
  \id{ier} is a \id{int} return completion flag equal to \id{0} for a success
  return and \id{-1} otherwise. See printed message for details in case
  of failure.
}
{
  See \id{SUNLinSol\_SPGMRSetPrecType} for complete further documentation of
  this routine.
}
%
% --------------------------------------------------------------------
%
\ucfunction{FSUNSPGMRSETMAXRS}
{
  FSUNSPGMRSETMAXRS(code, maxrs, ier)
}
{
  The function \ID{FSUNSPGMRSETMAXRS} can be called for Fortran programs
  to change the maximum number of restarts allowed for {\spgmr}.
}
{
  \begin{args}[maxrs]
  \item[code] (\id{int*})
    is an integer input specifying the solver id (1 for {\cvode}, 2
    for {\ida}, 3 for {\kinsol}, and 4 for {\arkode}).
  \item[maxrs] (\id{int*})
    maximum allowed number of restarts.
  \end{args}
}
{
  \id{ier} is a \id{int} return completion flag equal to \id{0} for a success
  return and \id{-1} otherwise. See printed message for details in case
  of failure.
}
{
  See \id{SUNLinSol\_SPGMRSetMaxRestarts} for complete further
  documentation of this routine.
}
%
% --------------------------------------------------------------------
%
\ucfunction{FSUNMASSSPGMRSETMAXRS}
{
  FSUNMASSSPGMRSETMAXRS(maxrs, ier)
}
{
  The function \ID{FSUNMASSSPGMRSETMAXRS} can be called for Fortran
  programs to change the maximum number of restarts allowed for
  {\spgmr} for mass matrix linear systems.
}
{
  The arguments are identical to \id{FSUNSPGMRSETMAXRS} above, except that
  \id{code} is not needed since mass matrix linear systems only arise
  in {\arkode}.
}
{
  \id{ier} is a \id{int} return completion flag equal to \id{0} for a success
  return and \id{-1} otherwise. See printed message for details in case
  of failure.
}
{
  See \id{SUNLinSol\_SPGMRSetMaxRestarts} for complete further
  documentation of this routine.
}
%
% --------------------------------------------------------------------
%

% ====================================================================
\subsection{SUNLinearSolver\_SPGMR content}
\label{ss:sunlinsol_spgmr_content}
% ====================================================================

The {\sunlinsolspgmr} module defines the \textit{content} field of a
\id{SUNLinearSolver} as the following structure:
%%
\begin{verbatim}
struct _SUNLinearSolverContent_SPGMR {
  int maxl;
  int pretype;
  int gstype;
  int max_restarts;
  booleantype zeroguess;
  int numiters;
  realtype resnorm;
  int last_flag;
  ATimesFn ATimes;
  void* ATData;
  PSetupFn Psetup;
  PSolveFn Psolve;
  void* PData;
  N_Vector s1;
  N_Vector s2;
  N_Vector *V;
  realtype **Hes;
  realtype *givens;
  N_Vector xcor;
  realtype *yg;
  N_Vector vtemp;
  int      print_level;
  FILE*    info_file;
};
\end{verbatim}
%%
These entries of the \emph{content} field contain the following
information:
\begin{args}[max\_restarts]
  \item[maxl] - number of GMRES basis vectors to use (default is 5),
  \item[pretype] - flag for type of preconditioning to employ
    (default is none),
  \item[gstype] - flag for type of Gram-Schmidt orthogonalization
    (default is modified Gram-Schmidt),
  \item[max\_restarts] - number of GMRES restarts to allow
    (default is 0),
  \item[numiters] - number of iterations from the most-recent solve,
  \item[resnorm] - final linear residual norm from the most-recent solve,
  \item[last\_flag] - last error return flag from an internal function,
  \item[ATimes] - function pointer to perform $Av$ product,
  \item[ATData] - pointer to structure for \id{ATimes},
  \item[Psetup] - function pointer to preconditioner setup routine,
  \item[Psolve] - function pointer to preconditioner solve routine,
  \item[PData] - pointer to structure for \id{Psetup} and \id{Psolve},
  \item[s1, s2] - vector pointers for supplied scaling matrices
    (default is \id{NULL}),
  \item[V] - the array of Krylov basis vectors
    $v_1, \ldots, v_{\text{\id{maxl}}+1}$, stored in \id{V[0]},
    \ldots, \id{V[maxl]}. Each $v_i$ is a vector of type {\nvector}.,
  \item[Hes] - the $(\text{\id{maxl}}+1)\times\text{\id{maxl}}$
    Hessenberg matrix. It is stored row-wise so that the (i,j)th
    element is given by \id{Hes[i][j]}.,
  \item[givens] - a length \id{2*maxl} array which represents the
    Givens rotation matrices that arise in the GMRES algorithm. These
    matrices are $F_0, F_1, \ldots, F_j$, where
    \begin{equation*}
    F_i = \begin{bmatrix}
      1 &        &   &     &      &   &        &   \\
        & \ddots &   &     &      &   &        &   \\
        &        & 1 &     &      &   &        &   \\
        &        &   & c_i & -s_i &   &        &   \\
        &        &   & s_i &  c_i &   &        &   \\
        &        &   &     &      & 1 &        &   \\
        &        &   &     &      &   & \ddots &   \\
        &        &   &     &      &   &        & 1\end{bmatrix},
    \end{equation*}
    are represented in the \id{givens} vector as \id{givens[0] =}
    $c_0$, \id{givens[1] = } $s_0$, \id{givens[2] = } $c_1$,
    \id{givens[3] = } $s_1$, \ldots \id{givens[2j] = } $c_j$,
    \id{givens[2j+1] = } $s_j$.,
  \item[xcor] - a vector which holds the scaled, preconditioned
    correction to the initial guess,
  \item[yg] - a length \id{(maxl+1)} array of \id{realtype} values
    used to hold ``short'' vectors (e.g. $y$ and $g$),
  \item[vtemp] - temporary vector storage.
  \item[print\_level] - controls the amount of information to be printed to the info file
  \item[info\_file]   - the file where all informative (non-error) messages will be directed
\end{args}
