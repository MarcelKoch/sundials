% This is a shared SUNDIALS TEX file with description of
% types used in llntyps.h
%
\index{portability}
The \ID{sundials\_types.h} file contains the definition of the type \ID{realtype},
which is used by the {\sundials} solvers for all floating-point data, the definition
of the integer type \ID{sunindextype}, which is used for vector and matrix indices,
and \ID{booleantype}, which is used for certain logic operations within {\sundials}.


\subsection{Floating point types}

The type \id{realtype} can be \id{float}, \id{double}, or \id{long double}, with
the default being \id{double}.
The user can change the precision of the {\sundials} solvers arithmetic at the
configuration stage (see \S\ref{ss:configuration_options_nix}).

Additionally, based on the current precision, \id{sundials\_types.h} defines
\Id{BIG\_REAL} to be the largest value representable as a \id{realtype},
\Id{SMALL\_REAL} to be the smallest value representable as a \id{realtype}, and
\Id{UNIT\_ROUNDOFF} to be the difference between $1.0$ and the minimum \id{realtype}
greater than $1.0$.

Within {\sundials}, real constants are set by way of a macro called
\Id{RCONST}.  It is this macro that needs the ability to branch on the
definition \id{realtype}.  In ANSI {\CC}, a floating-point constant with no
suffix is stored as a \id{double}.  Placing the suffix ``F'' at the
end of a floating point constant makes it a \id{float}, whereas using the suffix
``L'' makes it a \id{long double}.  For example,
\begin{verbatim}
#define A 1.0
#define B 1.0F
#define C 1.0L
\end{verbatim}
defines \id{A} to be a \id{double} constant equal to $1.0$, \id{B} to be a
\id{float} constant equal to $1.0$, and \id{C} to be a \id{long double} constant
equal to $1.0$.  The macro call \id{RCONST(1.0)} automatically expands to \id{1.0}
if \id{realtype} is \id{double}, to \id{1.0F} if \id{realtype} is \id{float},
or to \id{1.0L} if \id{realtype} is \id{long double}.  {\sundials} uses the
\id{RCONST} macro internally to declare all of its floating-point constants.

Additionally, {\sundials} defines several macros for common mathematical
functions \textit{e.g.}, \id{fabs}, \id{sqrt}, \id{exp}, etc. in
\id{sundials\_math.h}. The macros are prefixed with \id{SUNR} and expand to the
appropriate \id{C} function based on the \id{realtype}. For example, the macro
\id{SUNRabs} expands to the \id{C} function \id{fabs} when \id{realtype} is
\id{double}, \id{fabsf} when \id{realtype} is \id{float}, and \id{fabsl} when
\id{realtype} is \id{long double}.

A user program which uses the type \id{realtype}, the \id{RCONST} macro, and the
\id{SUNR} mathematical function macros is precision-independent except for any
calls to  precision-specific library functions. Our example programs use
\id{realtype}, \id{RCONST}, and the \id{SUNR} macros. Users can, however, use
the type \id{double}, \id{float}, or \id{long double} in their code (assuming
that this usage is consistent with the typedef for \id{realtype}) and call the
appropriate math library functions directly. Thus, a previously existing piece
of ANSI {\CC} code can use {\sundials} without modifying the code to use
\id{realtype}, \id{RCONST}, or the \id{SUNR} macros so long as the {\sundials}
libraries use the correct precision (for details see
\S\ref{ss:configuration_options_nix}).


\subsection{Integer types used for indexing}

The type \id{sunindextype} is used for indexing array entries in {\sundials}
modules (\textit{e.g.}, vectors lengths and matrix sizes) as well as for storing
the total problem size. During configuration \id{sunindextype} may be selected
to be either a 32- or 64-bit \emph{signed} integer with the default being
64-bit. See \S\ref{ss:configuration_options_nix} for the configuration option
to select the desired size of \id{sunindextype}. When using a 32-bit integer the
total problem size is limited to $2^{31}-1$ and with 64-bit integers the limit
is $2^{63}-1$. For users with problem sizes that exceed the 64-bit limit an
advanced configuration option is available to specify the type used for
\id{sunindextype}.

A user program which uses \id{sunindextype} to handle indices will work with
both index storage types except for any calls to index storage-specific
external libraries. Our \id{C} and \id{C++} example programs
use \id{sunindextype}. Users can, however, use any compatible type
(\textit{e.g.}, \id{int}, \id{long int}, \id{int32\_t}, \id{int64\_t}, or
\id{long long int}) in their code, assuming that this usage is consistent with
the typedef for \id{sunindextype} on their architecture. Thus, a previously
existing piece of ANSI {\CC} code can use {\sundials} without modifying the code
to use \id{sunindextype}, so long as the {\sundials} libraries use the
appropriate index storage type (for details see
\S\ref{ss:configuration_options_nix}).
