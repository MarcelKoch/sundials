%% This is a shared SUNDIALS TEX file with a description of the
%% SuperLU_DIST SLUNRloc SUNMatrix implementation
%%
\section{The SUNMatrix\_SLUNRloc implementation}\label{ss:sunmat_slunrloc}

The {\sunmatslunrloc} implementation of the {\sunmatrix} module provided with
{\sundials} is an adapter for the \id{SuperMatrix} structure provided by the
{\superludist} sparse matrix factorization and solver library written by
X. Sherry Li \cite{SuperLUDIST_site,GDL:07,LD:03,SLUUG:99}.
It is designed to be used with the {\sunlinsolsludist} linear solver
discussed in Section~\ref{ss:sunlinsol_sludist}. To this end, it defines the
{\em content} field of \id{SUNMatrix} to be the following structure:
%%
\begin{verbatim}
struct _SUNMatrixContent_SLUNRloc {
  booleantype   own_data;
  gridinfo_t    *grid;
  sunindextype  *row_to_proc;
  pdgsmv_comm_t *gsmv_comm;
  SuperMatrix   *A_super;
  SuperMatrix   *ACS_super;
};
\end{verbatim}
%%

A more complete description of the this \emph{content} field is given below:

\begin{description}
  \item[own\_data] - a flag which indicates if the SUNMatrix is responsible for freeing
    \id{A\_super}
  \item[grid] - pointer to the {\superludist} structure that stores the 2D process grid
  \item[row\_to\_proc] - a mapping between the rows in the matrix and the process it
    resides on; will be \id{NULL} until the \id{SUNMatMatvecSetup} routine is called
  \item[gsmv\_comm] - pointer to the {\superludist} structure that stores the
    communication information needed for matrix-vector multiplication; will be
    \id{NULL} until the \id{SUNMatMatvecSetup} routine is called
  \item[A\_super] - pointer to the underlying {\superludist} \id{SuperMatrix} with
      \id{Stype = SLU\_NR\_loc, Dtype = SLU\_D, Mtype = SLU\_GE}; must have the
      full diagonal present to be used with \id{SUNMatScaleAddI} routine
  \item[ACS\_super] - a column-sorted version of the matrix needed to perform matrix-vector
    multiplication; will be \id{NULL} until the routine \id{SUNMatMatvecSetup}
    routine is called
\end{description}

\noindent The header file to include when using this module
is \id{sunmatrix/sunmatrix\_slunrloc.h}. The installed module
library to link to is \id{libsundials\_sunmatrixslunrloc.\textit{lib}}
where \id{\em.lib} is typically \id{.so} for shared libraries and
\id{.a} for static libraries.


% ====================================================================
\subsection{SUNMatrix\_SLUNRloc functions}
\label{ss:sunmat_slunrloc_functions}
% ====================================================================

The module {\sunmatslunrloc} provides the following user-callable routines:
%%--------------------------------------
%%
\ucfunction{SUNMatrix\_SLUNRloc}
{
  A = SUNMatrix\_SLUNRloc(Asuper, grid);
}
{
  The function \ID{SUNMatrix\_SLUNRloc} creates and allocates memory for a
  {\sunmatslunrloc} object.
}
{
  \begin{args}
  \item[Asuper] (\id{SuperMatrix*})
      a fully-allocated {\superludist} \id{SuperMatrix} that the SUNMatrix will
      wrap; must have \id{Stype = SLU\_NR\_loc, Dtype = SLU\_D, Mtype = SLU\_GE}
      to be compatible
  \item[grid] (\id{gridinfo\_t*}) the initialized {\superludist} 2D process grid structure
  \end{args}
}
{
  a \id{SUNMatrix} object if \id{Asuper} is compatible else \id{NULL}
}
{
}

\ucfunction{SUNMatrix\_SLUNRloc\_Print}
{
  SUNMatrix\_SLUNRloc\_Print(A, fp);
}
{
  The function \ID{SUNMatrix\_SLUNRloc\_Print} prints the underlying
  \id{SuperMatrix} content.
}
{
  \begin{args}
  \item[A] (\id{SUNMatrix}) the matrix to print
  \item[fp] (\id{FILE}) the file pointer used for printing
  \end{args}
}
{
  \id{void}
}
{
}

\ucfunction{SUNMatrix\_SLUNRloc\_SuperMatrix}
{
  Asuper = SUNMatrix\_SLUNRloc\_SuperMatrix(A);
}
{
  The function \ID{SUNMatrix\_SLUNRloc\_SuperMatrix} provides access
  to the underlying {\superludist} \id{SuperMatrix} of \id{A}.
}
{
  \begin{args}
  \item[A] (\id{SUNMatrix}) the matrix to access
  \end{args}
}
{
  \id{SuperMatrix*}
}
{
}

\ucfunction{SUNMatrix\_SLUNRloc\_ProcessGrid}
{
  grid = SUNMatrix\_SLUNRloc\_ProcessGrid(A);
}
{
  The function \ID{SUNMatrix\_SLUNRloc\_ProcessGrid} provides access
  to the {\superludist} \id{gridinfo\_t} structure associated with \id{A}.
}
{
  \begin{args}
  \item[A] (\id{SUNMatrix}) the matrix to access
  \end{args}
}
{
  \id{gridinfo\_t*}
}
{
}

\ucfunction{SUNMatrix\_SLUNRloc\_OwnData}
{
  does\_own\_data = SUNMatrix\_SLUNRloc\_OwnData(A);
}
{
  The function \ID{SUNMatrix\_SLUNRloc\_OwnData} returns true if the \id{SUNMatrix}
  object is responsible for freeing \id{A\_super}, otherwise it returns false.
}
{
  \begin{args}
  \item[A] (\id{SUNMatrix}) the matrix to access
  \end{args}
}
{
  \id{booleantype}
}
{
}

The {\sunmatslunrloc} module defines implementations of all generic \id{SUNMatrix} operations
listed in Section \ref{ss:sunmatrix_functions}:

\begin{itemize}
  \item \id{SUNMatGetID\_SLUNRloc} - returns \id{SUNMATRIX\_SLUNRLOC}
  \item \id{SUNMatClone\_SLUNRloc}
  \item \id{SUNMatDestroy\_SLUNRloc}
  \item \id{SUNMatSpace\_SLUNRloc} - this only returns information for the storage within the
    matrix interface, i.e. storage for \id{row\_to\_proc}
  \item \id{SUNMatZero\_SLUNRloc}
  \item \id{SUNMatCopy\_SLUNRloc}
  \item \id{SUNMatScaleAdd\_SLUNRloc} - performs $A = cA + B$, but $A$ and $B$ must have the same sparsity pattern 
  \item \id{SUNMatScaleAddI\_SLUNRloc} - performs $A = cA + I$, but the diagonal of $A$ must be present
  \item \id{SUNMatMatvecSetup\_SLUNRloc} - initializes the {\superludist} parallel communication
    structures needed to perform a matrix-vector product; only needs to be called before the
    first call to \id{SUNMatMatvec} or if the matrix changed since the last setup
  \item \id{SUNMatMatvec\_SLUNRloc}
\end{itemize}

%%------------------------------------
%%
{\warn} The {\sunmatslunrloc} module requires that the complete diagonal, i.e. nonzeros and zeros,
is present in order to use the \id{SUNMatScaleAddI} operation.

% % ====================================================================
% \subsection{SUNMatrix\_SLUNRloc Fortran interfaces}
% \label{ss:sunmat_slunrloc_fortran}
% % ====================================================================

% The {\sunmatslunrloc} module provides a {\F} 2003 module as well as {\F} 77
% style interface functions for use from {\F} applications.

% \subsubsection*{FORTRAN 2003 interface module}
% The \ID{fsunmatrix\_slunrloc\_mod} {\F} module defines interfaces to most
% {\sunmatslunrloc} {\CC} functions using the intrinsic \id{iso\_c\_binding}
% module which provides a standardized mechanism for interoperating with {\CC}. As
% noted in the {\CC} function descriptions above, the interface functions are
% named after the corresponding {\CC} function, but with a leading `F'. For
% example, the function \id{SUNMatrix\_SLUNRloc} is interfaced as
% \id{FSUNMatrix\_SLUNRloc}.

% The {\F} 2003 {\sunmatslunrloc} interface module can be accessed with the \id{use}
% statement, i.e. \id{use fsunmatrix\_slunrloc\_mod}, and linking to the library
% \id{libsundials\_fsunmatrixslunrloc\_mod}.{\em lib} in addition to the {\CC} library.
% For details on where the library and module file \id{fsunmatrix\_slunrloc\_mod.mod}
% are installed see Appendix \ref{c:install}.

