% ====================================================================
\section{The SUNLinearSolver\_SuperLUMT implementation}
\label{ss:sunlinsol_superlumt}
% ====================================================================

This section describes the {\sunlinsol} implementation for solving sparse linear
systems with SuperLU\_MT. The {\superlumt} module is designed to be used with the
corresponding {\sunmatsparse} matrix type, and one of the serial or
shared-memory {\nvector} implementations ({\nvecs}, {\nvecopenmp}, or
{\nvecpthreads}). While these are compatible, it is not recommended
to use a threaded vector module with {\sunlinsolslumt} unless it is
the {\nvecopenmp} module and the {\superlumt} library has also been
compiled with OpenMP.

The header file to include when using this module
is \id{sunlinsol/sunlinsol\_superlumt.h}. The installed module
library to link to is
\id{libsundials\_sunlinsolsuperlumt.\textit{lib}}
where \id{\em.lib} is typically \id{.so} for shared libraries and
\id{.a} for static libraries.

The {\sunlinsolslumt} module is a {\sunlinsol} wrapper for
the {\superlumt} sparse matrix factorization and solver library
written by X. Sherry Li \cite{SuperLUMT_site,Li:05,DGL:99}.  The
package performs matrix factorization using threads to enhance
efficiency in shared memory parallel environments.  It should be noted
that threads are only used in the factorization step.  In
order to use the {\sunlinsolslumt} interface to {\superlumt}, it is
assumed that {\superlumt} has been installed on the system prior to
installation of {\sundials}, and that {\sundials} has been configured
appropriately to link with {\superlumt} (see Appendix \ref{c:install}
for details).  Additionally, this wrapper only supports single- and
double-precision calculations, and therefore cannot be compiled if
{\sundials} is configured to have \id{realtype} set to \id{extended}
(see Section \ref{s:types}).  Moreover, since the {\superlumt} library
may be installed to support either 32-bit or 64-bit integers, it is
assumed that the {\superlumt} library is installed using the same
integer precision as the {\sundials} \id{sunindextype} option. {\warn}

% ====================================================================
\subsection{SUNLinearSolver\_SuperLUMT description}
\label{ss:sunlinsol_slumt_usage}
% ====================================================================

The {\superlumt} library has a symbolic factorization routine that
computes the permutation of the linear system matrix to reduce fill-in
on subsequent $LU$ factorizations (using COLAMD, minimal degree
ordering on $A^T*A$, minimal degree ordering on $A^T+A$, or natural
ordering).  Of these ordering choices, the default value in the
{\sunlinsolslumt} module is the COLAMD ordering.

Since the linear systems that arise within the context of {\sundials}
calculations will typically have identical sparsity patterns, the
{\sunlinsolslumt} module is constructed to perform the
following operations:
\begin{itemize}
\item The first time that the ``setup'' routine is called, it
  performs the symbolic factorization, followed by an initial
  numerical factorization.
\item On subsequent calls to the ``setup'' routine, it skips the
  symbolic factorization, and only refactors the input matrix.
\item The ``solve'' call performs pivoting and forward and
  backward substitution using the stored {\superlumt} data
  structures.  We note that in this solve {\superlumt} operates on the
  native data arrays for the right-hand side and solution vectors,
  without requiring costly data copies.
\end{itemize}


% ====================================================================
\subsection{SUNLinearSolver\_SuperLUMT functions}
\label{ss:sunlinsol_slumt_functions}
% ====================================================================

The module {\sunlinsolslumt} provides the following user-callable constructor
for creating a \newline \id{SUNLinearSolver} object.
%
% --------------------------------------------------------------------
%
\ucfunctiond{SUNLinSol\_SuperLUMT}
{
  LS = SUNLinSol\_SuperLUMT(y, A, num\_threads);
}
{
  The function \ID{SUNLinSol\_SuperLUMT} creates and allocates memory for
  a \newline SuperLU\_MT-based \id{SUNLinearSolver} object.
}
{
  \begin{args}[num\_threads]
  \item[y] (\id{N\_Vector})
    a template for cloning vectors needed within the solver
  \item[A] (\id{SUNMatrix})
    a {\sunmatsparse} matrix template for cloning matrices needed
    within the solver
  \item[num\_threads] (\id{int})
    desired number of threads (OpenMP or Pthreads, depending on how
    {\superlumt} was installed) to use during the factorization steps
  \end{args}
}
{
  This returns a \id{SUNLinearSolver} object.  If either \id{A} or
  \id{y} are incompatible then this routine will return \id{NULL}.
}
{
  This routine analyzes the input matrix and vector to determine the
  linear system size and to assess compatibility with the {\superlumt}
  library.

  This routine will perform consistency checks to ensure that it is
  called with consistent {\nvector} and {\sunmatrix} implementations.
  These are currently limited to the {\sunmatsparse} matrix type
  (using either CSR or CSC storage formats) and the {\nvecs},
  {\nvecopenmp}, and {\nvecpthreads} vector types.  As additional
  compatible matrix and vector implementations are added to
  {\sundials}, these will be included within this compatibility
  check.

  The \id{num\_threads} argument is not checked and is passed directly
  to {\superlumt} routines.
}
{SUNSuperLUMT}
%
% --------------------------------------------------------------------
%
\noindent The {\sunlinsolslumt} module defines implementations of all
``direct'' linear solver operations listed in Sections
\ref{ss:sunlinsol_CoreFn} -- \ref{ss:sunlinsol_GetFn}:
\begin{itemize}
\item \id{SUNLinSolGetType\_SuperLUMT}
\item \id{SUNLinSolInitialize\_SuperLUMT} -- this sets the
  \id{first\_factorize} flag to 1 and resets the internal {\superlumt}
  statistics variables.
\item \id{SUNLinSolSetup\_SuperLUMT} -- this performs either a $LU$
  factorization or refactorization of the input matrix.
\item \id{SUNLinSolSolve\_SuperLUMT} -- this calls the appropriate
  {\superlumt} solve routine to utilize the $LU$ factors to solve the
  linear system.
\item \id{SUNLinSolLastFlag\_SuperLUMT}
\item \id{SUNLinSolSpace\_SuperLUMT} -- this only returns information for
  the storage within the solver \emph{interface}, i.e.~storage for the
  integers \id{last\_flag} and \id{first\_factorize}.  For additional
  space requirements, see the {\superlumt} documentation.
\item \id{SUNLinSolFree\_SuperLUMT}
\end{itemize}

The {\sunlinsolslumt} module also defines the following additional
user-callable function.
%
% --------------------------------------------------------------------
%
\ucfunctiond{SUNLinSol\_SuperLUMTSetOrdering}
{
  retval = SUNLinSol\_SuperLUMTSetOrdering(LS, ordering);
}
{
  This function sets the ordering used by {\superlumt} for reducing fill in
  the linear solve.
}
{
  \begin{args}[ordering]
  \item[LS] (\id{SUNLinearSolver})
    the {\sunlinsolslumt} object
  \item[ordering] (\id{int})
    a flag indicating the ordering algorithm to use, the options are:
    \begin{itemize}
    \item[0] natural ordering
    \item[1] minimal degree ordering on $A^TA$
    \item[2] minimal degree ordering on $A^T+A$
    \item[3] COLAMD ordering for unsymmetric matrices
    \end{itemize}
    The default is 3 for COLAMD.
  \end{args}
}
{
  The return values from this function are \id{SUNLS\_MEM\_NULL}
  (\id{S} is \id{NULL}), \newline \id{SUNLS\_ILL\_INPUT}
  (invalid ordering choice), or \id{SUNLS\_SUCCESS}.
}
{}
{SUNSuperLUMTSetOrdering}

% ====================================================================
\subsection{SUNLinearSolver\_SuperLUMT Fortran interfaces}
\label{ss:sunlinsol_slumt_fortran}
% ====================================================================

For solvers that include a Fortran interface module, the
{\sunlinsolslumt} module also includes a Fortran-callable function
for creating a \id{SUNLinearSolver} object.
%
% --------------------------------------------------------------------
%
\ucfunction{FSUNSUPERLUMTINIT}
{
  FSUNSUPERLUMTINIT(code, num\_threads, ier)
}
{
  The function \ID{FSUNSUPERLUMTINIT} can be called for Fortran programs
  to create a {\sunlinsolklu} object.
}
{
  \begin{args}[num\_threads]
  \item[code] (\id{int*})
    is an integer input specifying the solver id (1 for {\cvode}, 2
    for {\ida}, 3 for {\kinsol}, and 4 for {\arkode}).
  \item[num\_threads] (\id{int*})
    desired number of threads (OpenMP or Pthreads, depending on how
    {\superlumt} was installed) to use during the factorization steps
  \end{args}
}
{
  \id{ier} is a return completion flag equal to \id{0} for a success
  return and \id{-1} otherwise. See printed message for details in case
  of failure.
}
{
  This routine must be
  called \emph{after} both the {\nvector} and {\sunmatrix} objects have
  been initialized.
}
Additionally, when using {\arkode} with a non-identity
mass matrix, the {\sunlinsolslumt} module includes a Fortran-callable
function for creating a \id{SUNLinearSolver} mass matrix solver
object.
%
% --------------------------------------------------------------------
%
\ucfunction{FSUNMASSSUPERLUMTINIT}
{
  FSUNMASSSUPERLUMTINIT(num\_threads, ier)
}
{
  The function \ID{FSUNMASSSUPERLUMTINIT} can be called for Fortran programs
  to create a SuperLU\_MT-based \id{SUNLinearSolver} object for mass matrix linear
  systems.
}
{
  \begin{args}[num\_threads]
  \item[num\_threads] (\id{int*})
    desired number of threads (OpenMP or Pthreads, depending on how
    {\superlumt} was installed) to use during the factorization steps.
  \end{args}
}
{
  \id{ier} is a \id{int} return completion flag equal to \id{0} for a success
  return and \id{-1} otherwise. See printed message for details in case
  of failure.
}
{
  This routine must be
  called \emph{after} both the {\nvector} and {\sunmatrix} mass-matrix
  objects have been initialized.
}
The \ID{SUNLinSol\_SuperLUMTSetOrdering} routine also supports Fortran
interfaces for the system and mass matrix solvers:
%
% --------------------------------------------------------------------
%
\ucfunction{FSUNSUPERLUMTSETORDERING}
{
  FSUNSUPERLUMTSETORDERING(code, ordering, ier)
}
{
  The function \ID{FSUNSUPERLUMTSETORDERING} can be called for Fortran programs
  to update the ordering algorithm in a {\sunlinsolslumt} object.
}
{
  \begin{args}[ordering]
  \item[code] (\id{int*})
    is an integer input specifying the solver id (1 for {\cvode}, 2
    for {\ida}, 3 for {\kinsol}, and 4 for {\arkode}).
  \item[ordering] (\id{int*})
    a flag indicating the ordering algorithm, options are:
    \begin{itemize}
    \item[0] natural ordering
    \item[1] minimal degree ordering on $A^TA$
    \item[2] minimal degree ordering on $A^T+A$
    \item[3] COLAMD ordering for unsymmetric matrices
    \end{itemize}
    The default is 3 for COLAMD.
  \end{args}
}
{
  \id{ier} is a \id{int} return completion flag equal to \id{0} for a success
  return and \id{-1} otherwise. See printed message for details in case
  of failure.
}
{
  See \id{SUNLinSol\_SuperLUMTSetOrdering} for complete further
  documentation of this routine.
}
%
% --------------------------------------------------------------------
%
\ucfunction{FSUNMASSUPERLUMTSETORDERING}
{
  FSUNMASSUPERLUMTSETORDERING(ordering, ier)
}
{
  The function \ID{FSUNMASSUPERLUMTSETORDERING} can be called for Fortran
  programs to update the ordering algorithm in a {\sunlinsolslumt}
  object for mass matrix linear systems.
}
{
  \begin{args}[ordering]
  \item[ordering] (\id{int*})
    a flag indicating the ordering algorithm, options are:
    \begin{itemize}
    \item[0] natural ordering
    \item[1] minimal degree ordering on $A^TA$
    \item[2] minimal degree ordering on $A^T+A$
    \item[3] COLAMD ordering for unsymmetric matrices
    \end{itemize}
    The default is 3 for COLAMD.
  \end{args}
}
{
  \id{ier} is a \id{int} return completion flag equal to \id{0} for a success
  return and \id{-1} otherwise. See printed message for details in case
  of failure.
}
{
  See \id{SUNLinSol\_SuperLUMTSetOrdering} for complete further
  documentation of this routine.
}


% ====================================================================
\subsection{SUNLinearSolver\_SuperLUMT content}
\label{ss:sunlinsol_slumt_content}
% ====================================================================

The {\sunlinsolslumt} module defines the \textit{content} field of a
\id{SUNLinearSolver} as the following structure:
%%
\begin{verbatim}
struct _SUNLinearSolverContent_SuperLUMT {
  int          last_flag;
  int          first_factorize;
  SuperMatrix  *A, *AC, *L, *U, *B;
  Gstat_t      *Gstat;
  sunindextype *perm_r, *perm_c;
  sunindextype N;
  int          num_threads;
  realtype     diag_pivot_thresh;
  int          ordering;
  superlumt_options_t *options;
};
\end{verbatim}
%%
These entries of the \emph{content} field contain the following
information:
\begin{args}[diag\_pivot\_thresh]
  \item[last\_flag] - last error return flag from internal function evaluations,
  \item[first\_factorize] - flag indicating whether the factorization
    has ever been performed,
  \item[A, AC, L, U, B] - \id{SuperMatrix} pointers used in solve,
  \item[Gstat] - \id{GStat\_t} object used in solve,
  \item[perm\_r, perm\_c] - permutation arrays used in solve,
  \item[N] - size of the linear system,
  \item[num\_threads] - number of OpenMP/Pthreads threads to use,
  \item[diag\_pivot\_thresh] - threshold on diagonal pivoting,
  \item[ordering] - flag for which reordering algorithm to use,
  \item[options] - pointer to {\superlumt} options structure.
\end{args}
