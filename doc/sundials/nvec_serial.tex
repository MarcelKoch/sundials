% This is a shared SUNDIALS TEX file with a description of the
%% serial nvector implementation
%%
\section{The NVECTOR\_SERIAL implementation}\label{ss:nvec_ser}

The serial implementation of the {\nvector} module provided with {\sundials},
{\nvecs}, defines the {\em content} field of \id{N\_Vector} to be a structure
containing the length of the vector, a pointer to the beginning of a contiguous
data array, and a boolean flag {\em own\_data} which specifies the ownership
of {\em data}.
%%
\begin{verbatim}
struct _N_VectorContent_Serial {
  sunindextype length;
  booleantype own_data;
  realtype *data;
};
\end{verbatim}
%%
%%--------------------------------------------
%%

The header file to include when using this module is \id{nvector\_serial.h}.
The installed module library to link to is
\id{libsundials\_nvecserial.\textit{lib}}
where \id{\em.lib} is typically \id{.so} for shared libraries and \id{.a}
for static libraries.


% ====================================================================
\subsection{NVECTOR\_SERIAL accessor macros}
\label{ss:nvec_ser_macros}
% ====================================================================

The following macros are provided to access the content of an {\nvecs}
vector. The suffix \id{\_S} in the names denotes the serial version.
%%
\begin{itemize}

\item \ID{NV\_CONTENT\_S}

  This routine gives access to the contents of the serial
  vector \id{N\_Vector}.

  The assignment \id{v\_cont} $=$ \id{NV\_CONTENT\_S(v)} sets
  \id{v\_cont} to be a pointer to the serial \id{N\_Vector} content
  structure.

  Implementation:

  \verb|#define NV_CONTENT_S(v) ( (N_VectorContent_Serial)(v->content) )|

\item \ID{NV\_OWN\_DATA\_S}, \ID{NV\_DATA\_S}, \ID{NV\_LENGTH\_S}


  These macros give individual access to the parts of
  the content of a serial \id{N\_Vector}.

  The assignment \id{v\_data = NV\_DATA\_S(v)} sets \id{v\_data} to be
  a pointer to the first component of the data for the \id{N\_Vector} \id{v}.
  The assignment \id{NV\_DATA\_S(v) = v\_data} sets the component array of \id{v} to
  be \id{v\_data} by storing the pointer \id{v\_data}.

  The assignment \id{v\_len = NV\_LENGTH\_S(v)} sets \id{v\_len} to be
  the length of \id{v}. On the other hand, the call \id{NV\_LENGTH\_S(v) = len\_v}
  sets the length of \id{v} to be \id{len\_v}.

  Implementation:

  \verb|#define NV_OWN_DATA_S(v) ( NV_CONTENT_S(v)->own_data )|

  \verb|#define NV_DATA_S(v) ( NV_CONTENT_S(v)->data )|

  \verb|#define NV_LENGTH_S(v) ( NV_CONTENT_S(v)->length )|

\item \ID{NV\_Ith\_S}

  This macro gives access to the individual components of the data
  array of an \id{N\_Vector}.

  The assignment \id{r = NV\_Ith\_S(v,i)} sets \id{r} to be the value of
  the \id{i}-th component of \id{v}. The assignment \id{NV\_Ith\_S(v,i) = r}
  sets the value of the \id{i}-th component of \id{v} to be \id{r}.

  Here $i$ ranges from $0$ to $n-1$ for a vector of length $n$.

  Implementation:

  \verb|#define NV_Ith_S(v,i) ( NV_DATA_S(v)[i] )|

\end{itemize}


% ====================================================================
\subsection{NVECTOR\_SERIAL functions}
\label{ss:nvec_ser_functions}
% ====================================================================

The {\nvecs} module defines serial implementations of all vector operations listed
in Tables \ref{t:nvecops}, \ref{t:nvecfusedops}, \ref{t:nvecarrayops}
and \ref{t:nveclocalops}. Their
names are obtained from those in these tables by appending the suffix \id{\_Serial}
(e.g. \id{N\_VDestroy\_Serial}).
All the standard vector operations listed in \ref{t:nvecops} with the suffix
\id{\_Serial} appended are callable via the {\F} 2003 interface by prepending an
`F' (e.g. \id{FN\_VDestroy\_Serial}).

The module {\nvecs} provides the following additional user-callable routines:
%%--------------------------------------
\sunmodfunf{N\_VNew\_Serial}
{
  This function creates and allocates memory for a serial \id{N\_Vector}.
  Its only argument is the vector length.
}
{
  N\_Vector N\_VNew\_Serial(sunindextype vec\_length);
}
%%--------------------------------------
\sunmodfunf{N\_VNewEmpty\_Serial}
{
  This function creates a new serial \id{N\_Vector} with an empty (\id{NULL})
  data array.
}
{
  N\_Vector N\_VNewEmpty\_Serial(sunindextype vec\_length);
}
%%--------------------------------------
\sunmodfunf{N\_VMake\_Serial}
{
  This function creates and allocates memory for a serial vector
  with user-provided data array.

  (This function does {\em not} allocate memory for \id{v\_data} itself.)
}
{
  N\_Vector N\_VMake\_Serial(sunindextype vec\_length, realtype *v\_data);
}
%%--------------------------------------
\sunmodfun{N\_VCloneVectorArray\_Serial}
{
  This function creates (by cloning) an array of \id{count} serial vectors.
}
{
  N\_Vector *N\_VCloneVectorArray\_Serial(int count, N\_Vector w);
}
%%--------------------------------------
\sunmodfun{N\_VCloneVectorArrayEmpty\_Serial}
{
  This function creates (by cloning) an array of \id{count} serial vectors,
  each with an empty (\id{NULL}) data array.
}
{
  N\_Vector *N\_VCloneVectorArrayEmpty\_Serial(int count, N\_Vector w);
}
%%--------------------------------------
\sunmodfun{N\_VDestroyVectorArray\_Serial}
{
  This function frees memory allocated for the array of \id{count} variables of type
  \id{N\_Vector} created with \id{N\_VCloneVectorArray\_Serial} or with \newline
  \id{N\_VCloneVectorArrayEmpty\_Serial}.
}
{
  void N\_VDestroyVectorArray\_Serial(N\_Vector *vs, int count);
}
%%--------------------------------------
\sunmodfunf{N\_VPrint\_Serial}
{
  This function prints the content of a serial vector to \id{stdout}.
}
{
  void N\_VPrint\_Serial(N\_Vector v);
}
%%--------------------------------------
\sunmodfun{N\_VPrintFile\_Serial}
{
  This function prints the content of a serial vector to \id{outfile}.
}
{
  void N\_VPrintFile\_Serial(N\_Vector v, FILE *outfile);
}
%%--------------------------------------

By default all fused and vector array operations are disabled in the {\nvecs}
module. The following additional user-callable routines are provided to
enable or disable fused and vector array operations for a specific vector. To
ensure consistency across vectors it is recommended to first create a vector
with \id{N\_VNew\_Serial}, enable/disable the desired operations for that vector
with the functions below, and create any additional vectors from that vector
using \id{N\_VClone}. This guarantees the new vectors will have the same
operations enabled/disabled as cloned vectors inherit the same enable/disable
options as the vector they are cloned from while vectors created with
\id{N\_VNew\_Serial} will have the default settings for the {\nvecs} module.
%%--------------------------------------
\sunmodfun{N\_VEnableFusedOps\_Serial}
{
  This function enables (\id{SUNTRUE}) or disables (\id{SUNFALSE}) all fused and
  vector array operations in the serial vector. The return value is \id{0} for
  success and \id{-1} if the input vector or its \id{ops} structure are \id{NULL}.
}
{
  int N\_VEnableFusedOps\_Serial(N\_Vector v, booleantype tf);
}
%%--------------------------------------
\sunmodfun{N\_VEnableLinearCombination\_Serial}
{
  This function enables (\id{SUNTRUE}) or disables (\id{SUNFALSE}) the linear
  combination fused operation in the serial vector. The return value is \id{0} for
  success and \id{-1} if the input vector or its \id{ops} structure are \id{NULL}.
}
{
  int N\_VEnableLinearCombination\_Serial(N\_Vector v, booleantype tf);
}
%%--------------------------------------
\sunmodfun{N\_VEnableScaleAddMulti\_Serial}
{
  This function enables (\id{SUNTRUE}) or disables (\id{SUNFALSE}) the scale and
  add a vector to multiple vectors fused operation in the serial vector. The
  return value is \id{0} for success and \id{-1} if the input vector or its
  \id{ops} structure are \id{NULL}.
}
{
  int N\_VEnableScaleAddMulti\_Serial(N\_Vector v, booleantype tf);
}
%%--------------------------------------
\sunmodfun{N\_VEnableDotProdMulti\_Serial}
{
  This function enables (\id{SUNTRUE}) or disables (\id{SUNFALSE}) the multiple
  dot products fused operation in the serial vector. The return value is \id{0}
  for success and \id{-1} if the input vector or its \id{ops} structure are
  \id{NULL}.
}
{
  int N\_VEnableDotProdMulti\_Serial(N\_Vector v, booleantype tf);
}
%%--------------------------------------
\sunmodfun{N\_VEnableLinearSumVectorArray\_Serial}
{
  This function enables (\id{SUNTRUE}) or disables (\id{SUNFALSE}) the linear sum
  operation for vector arrays in the serial vector. The return value is \id{0} for
  success and \id{-1} if the input vector or its \id{ops} structure are \id{NULL}.
}
{
  int N\_VEnableLinearSumVectorArray\_Serial(N\_Vector v, booleantype tf);
}
%%--------------------------------------
\sunmodfun{N\_VEnableScaleVectorArray\_Serial}
{
  This function enables (\id{SUNTRUE}) or disables (\id{SUNFALSE}) the scale
  operation for vector arrays in the serial vector. The return value is \id{0} for
  success and \id{-1} if the input vector or its \id{ops} structure are \id{NULL}.
}
{
  int N\_VEnableScaleVectorArray\_Serial(N\_Vector v, booleantype tf);
}
%%--------------------------------------
\sunmodfun{N\_VEnableConstVectorArray\_Serial}
{
  This function enables (\id{SUNTRUE}) or disables (\id{SUNFALSE}) the const
  operation for vector arrays in the serial vector. The return value is \id{0} for
  success and \id{-1} if the input vector or its \id{ops} structure are \id{NULL}.
}
{
  int N\_VEnableConstVectorArray\_Serial(N\_Vector v, booleantype tf);
}
%%--------------------------------------
\sunmodfun{N\_VEnableWrmsNormVectorArray\_Serial}
{
  This function enables (\id{SUNTRUE}) or disables (\id{SUNFALSE}) the WRMS norm
  operation for vector arrays in the serial vector. The return value is \id{0} for
  success and \id{-1} if the input vector or its \id{ops} structure are \id{NULL}.
}
{
  int N\_VEnableWrmsNormVectorArray\_Serial(N\_Vector v, booleantype tf);
}
%%--------------------------------------
\sunmodfun{N\_VEnableWrmsNormMaskVectorArray\_Serial}
{
  This function enables (\id{SUNTRUE}) or disables (\id{SUNFALSE}) the masked WRMS
  norm operation for vector arrays in the serial vector. The return value is
  \id{0} for success and \id{-1} if the input vector or its \id{ops} structure are
  \id{NULL}.
}
{
  int N\_VEnableWrmsNormMaskVectorArray\_Serial(N\_Vector v, booleantype tf);
}
%%--------------------------------------
\sunmodfun{N\_VEnableScaleAddMultiVectorArray\_Serial}
{
  This function enables (\id{SUNTRUE}) or disables (\id{SUNFALSE}) the scale and
  add a vector array to multiple vector arrays operation in the serial vector. The
  return value is \id{0} for success and \id{-1} if the input vector or its
  \id{ops} structure are \id{NULL}.
}
{
  int N\_VEnableScaleAddMultiVectorArray\_Serial(N\_Vector v,
  \newlinefill{int N\_VEnableScaleAddMultiVectorArray\_Serial}
  booleantype tf);
}
%%--------------------------------------
\sunmodfun{N\_VEnableLinearCombinationVectorArray\_Serial}
{
  This function enables (\id{SUNTRUE}) or disables (\id{SUNFALSE}) the linear
  combination operation for vector arrays in the serial vector. The return value
  is \id{0} for success and \id{-1} if the input vector or its \id{ops} structure
  are \id{NULL}.
}
{
  int N\_VEnableLinearCombinationVectorArray\_Serial(N\_Vector v,
  \newlinefill{int N\_VEnableLinearCombinationVectorArray\_Serial}
  booleantype tf);
}
%%
%%------------------------------------
%%
\paragraph{\bf Notes}

\begin{itemize}

\item
  When looping over the components of an \id{N\_Vector} \id{v}, it is
  more efficient to first obtain the component array via
  \id{v\_data = NV\_DATA\_S(v)} and then access \id{v\_data[i]} within the
  loop than it is to use \id{NV\_Ith\_S(v,i)} within the loop.

\item
  {\warn}\id{N\_VNewEmpty\_Serial}, \id{N\_VMake\_Serial},
  and \id{N\_VCloneVectorArrayEmpty\_Serial} set the field
  {\em own\_data} $=$ \id{SUNFALSE}.
  \id{N\_VDestroy\_Serial} and \id{N\_VDestroyVectorArray\_Serial}
  will not attempt to free the pointer {\em data} for any \id{N\_Vector} with
  {\em own\_data} set to \id{SUNFALSE}. In such a case, it is the user's responsibility to
  deallocate the {\em data} pointer.

\item
  {\warn}To maximize efficiency, vector operations in the {\nvecs} implementation
  that have more than one \id{N\_Vector} argument do not check for
  consistent internal representation of these vectors. It is the user's
  responsibility to ensure that such routines are called with \id{N\_Vector}
  arguments that were all created with the same internal representations.

\end{itemize}


% ====================================================================
\subsection{NVECTOR\_SERIAL Fortran interfaces}
\label{ss:nvec_ser_fortran}
% ====================================================================

The {\nvecs} module provides a {\F} 2003 module as well as {\F} 77
style interface functions for use from {\F} applications.

\subsubsection*{FORTRAN 2003 interface module}
The \ID{fnvector\_serial\_mod} {\F} module defines interfaces to all
{\nvecs} {\CC} functions using the intrinsic \id{iso\_c\_binding}
module which provides a standardized mechanism for interoperating with {\CC}. As
noted in the {\CC} function descriptions above, the interface functions are
named after the corresponding {\CC} function, but with a leading `F'. For
example, the function \id{N\_VNew\_Serial} is interfaced as
\id{FN\_VNew\_Serial}.

The {\F} 2003 {\nvecs} interface module can be accessed with the \id{use}
statement, i.e. \id{use fnvector\_serial\_mod}, and linking to the library
\id{libsundials\_fnvectorserial\_mod}.{\em lib} in addition to the {\CC} library.
For details on where the library and module file
\id{fnvector\_serial\_mod.mod} are installed see Appendix \ref{c:install}.
We note that the module is accessible from the {\F} 2003 {\sundials} integrators
\textit{without} separately linking to the
\id{libsundials\_fnvectorserial\_mod} library.

\subsubsection*{FORTRAN 77 interface functions}
For solvers that include a {\F} 77 interface module, the {\nvecs} module
also includes a {\F}-callable function \id{FNVINITS(code, NEQ, IER)},
to initialize this {\nvecs} module.  Here \id{code} is an input solver id
(1 for {\cvode}, 2 for {\ida}, 3 for {\kinsol}, 4 for {\arkode}); NEQ is
the problem size (declared so as to match C type \id{long int}); and
IER is an error return flag equal 0 for success and -1 for failure.
